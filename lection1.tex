% \section{Лекция 1 -- 2024-02-07 -- Механика сплошных сред}\label{sec:lec1}

Програмные комплексы:

%сеня
\begin{enumerate}
	\item \texttt{ANSYS}
	\item \texttt{ЛОГОС}
	\item \texttt{Abaqus}
	\item \texttt{Манипула}
	\item \texttt{SolidWorks}
	\item \texttt{Компас 3D}
\end{enumerate}

План курса МСС:
\begin{enumerate}
  \item Введение:
    \begin{enumerate}
      \item Объекты и методы МСС;
      \item Основные задачи МСС;
    \end{enumerate}

  \item Элементы тензорного анализа;
  \item Основополагающие аксиомы МСС;
  \item Кинематика МСС;
  \item Законы сохранения МСС:
    \begin{enumerate}
      \item закон сохранения массы;
      \item закон изменения количества движения (закон сохранения импульса);
      \item закон изменения момента количества движения;
      \item первый закон термодинамики;
      \item второй закон термодинамики;
      \item нулевой закон термодинамики;
    \end{enumerate}
  \item Определяющие соотношения;
  \item Замкнутые системы уравнений МСС;
  \item Соотношения на поверхностях сильных разрывов;

    След сем:
  \item Основы механики деформируемого твердого тела (МДТТ);
  \item Основы механики жидкостей и газов (МЖГ).
\end{enumerate}

\section{Введение}

\subsection{Объекты МСС}

\paragraph{Что изучает МСС?}
Рассматривается физические (материальные) объекты. Выберем \underline{геометрический способ}
описания объектов:
\begin{itemize}
  \item $L$ --- характерный размер (длина) материального объекта (м);
    
	\begin{figure}[H]
		\centering
		\includesvg[scale = 0.8]{lec01_scales}
	\end{figure}
    
    Объекты:
    \begin{itemize}
      \item естественного происхождения;
      \item искусственного происхождения;
    \end{itemize}

    Верхний предел применимости МСС где-то между $10^7$ и $10^9$ -- там нарушается первый принцип
    МСС, т.е. исчезает сплошность -- между планетками пустота. Однако после этой границы всё равно 
    возникает возможность применения МСС для релятивистких явлений (это такое обобщение МСС).

    Композиты -- особый вид твердых сред (материя).

    Итоги:
    \begin{enumerate}
      \item $L_{min} \leqslant L \leqslant L_{max}$ -- границы применимости МСС. При больших
        длинах -- астрофизика, при меньших -- физика микромира.

      \item Сплошность: $\exists$ самоподобных характерных объемов в которых $\exists$ <<много>>
        вещества.

      \item Основные области применения законов МСС:
        \begin{itemize}
          \item автомобилестроение;
          \item двигателестроение;
          \item авиастроение;
          \item ракетостроение;
          \item атомо техника;
          \item строительство;
          \item биомеханика (движение крови, работа сердца, трансплантация и т.д.);
          \item геомеханика (прогнозирование климата, тектоника и т.д.);
          \item композиты (прогнозирование свойств композитов)
		  \newline
		  \begin{quotation}
				Тервер говно, потому что использует грубые методы, не применяя информацию о внутренних
				законах исследуемого объекта. Поэтому он не применим в композитах -- углеродные волокна 
				валяются просто, есть что-то жидкое (связующее вещество) и оно не валяется и из этого 
				всего возникает крыло -- вот как это предказать.
				\flushright---Димитриенко
			\end{quotation}
        \end{itemize}
    \end{enumerate}
\end{itemize}

\subsection{Методы МСС}

Рассматриваются следующие науки: МСС, физика, химия, математика.

\begin{enumerate}
  \item По объектам, которые изучает МСС -- это просто \underline{часть физики}.

  \item Если изучается движение сред с химическими реакциями, то это раздел МСС -- механика
    многокомпонентых сред.

  \item Методы построения законов -- МСС - это раздел математики. Математика построена на аксиомах.
    МСС близка к математике, потому что она основана на аксиомах:
    \begin{itemize}
      \item аксиома сплошности;
      \item Евклидово пространство $\mathcal{E}_3^a$;
      \item $\exists$ абсолютное время.
    \end{itemize}
\end{enumerate}


\section{Элементы тензорного анализа}

Основные пространства в МСС:
\begin{enumerate}
  \item $\mathcal{L}_n$ -- векторное (линейное пространство) пространство;
  \item $\mathcal{E}_n$ -- евклидово простраство;
  \item $\mathcal{E}_n^a$ -- точечно-евклидово простраство (афинное);
  \item $\chi$ -- метрическое пространство.
\end{enumerate}

\subsection{Векторное (линейное) пространство}

\begin{definition}
  $\mathcal{L}_n$ -- множество, в котором введены 2 операции:
  \begin{itemize}
    \item сложение
    \item умножение на число
  \end{itemize}
\end{definition}

\begin{example}
  Пространство элементарной геометрии $\mathbb{E}_2$ и $\mathbb{E}_3$.

  Множество геометрических объектов, состоящих из точек, прямых, плосткостей и из ... Геометрический
  вектор -- направленный отрезок.

  На множестве геометрических векторов введём отношение эквивалентности, такое что два вектора
  эквивалентны, если один получаются параллельным переносом. Тогда сложение векторов определим
  как сложение классов эквивалентости по правилу параллелограма. 

  $V_2$ -- пространство свободных векторов в $\mathbb E_2$.
  $V_3$ -- пространство свободных векторов в $\mathbb E_3$
\end{example}

\begin{example}
  Арифметическое пространство координатных столбцов $\mathbb{R}_n$.
  \[
    \begin{pmatrix}
      x_1 \\
      \vdots \\
      x_n
    \end{pmatrix} \in \mathbb{R}_n; \quad
    \begin{pmatrix}
      x_1 \\
      \vdots \\
      x_n
    \end{pmatrix} + \begin{pmatrix}
      y_1 \\
      \vdots \\
      y_n
    \end{pmatrix} = 
    \begin{pmatrix}
      x_1 + y_1 \\
      \vdots \\
      x_n + y_n
    \end{pmatrix}; \quad
    \lambda \begin{pmatrix}
      x_1 \\
      \vdots \\
      x_n
    \end{pmatrix} 
  \]
\end{example}

В любом $\mathcal{L}_n$ $\exists$ \underline{базис} (конечный), т.е. такая система векторов, что
\[
  \exists \vec{e_1}, \dots, \vec{e_n}: \forall \vec{a} \in \mathcal L_n: \exists a^i :  \vec{a} = a^i \vec{e}_i;
\]
Не существует такого набора чисел $s^1, \dots, s^n$, такого что не все $s^i = 0$, так что
$s^i \vec{e}_i = 0$.

\begin{example}
  Базис в $\mathbb E_2$ ( $V_2$ ). 
\end{example}

\paragraph{Замена базиса}

Рассмотрим базис $\vec{e}_i$ и $\vec{e}_i' = Q_i^j \vec{e}_j = Q^k_i \vec{e}_k$. $Q^j_i$ -- матрица 
замены базисов ($\det Q^j_i \neq 0$), к которой $\exists P^j_i$ -- обратная матрица, т.е. $P^i_j Q^j_k = \delta^i_k$; 

Преобразование компоненты вектора при замене базиса
\[
  \vec{a} = a^i \vec{e}_j = {a'}^i \vec{e}_j = ({a'}^j Q^i_j) \vec{e}_i
\]

\subsection{Евклидово пространство}

\begin{definition}
  $\mathcal{L}_n$, в котором дополнительно введена операция скалярного умножения: $\varphi: \mathcal{E}_n^2 \to \mathbb{R}$. 
  Свойства:
  \begin{itemize}
    \item $\vec{a} \cdot \vec{b} = \vec{b} \cdot \vec{a}$;
    \item $(\vec{a}+\vec{b}) \cdot \vec{c} = \vec{a} \cdot \vec{c} + \vec{b} \cdot \vec{c}$;
    \item $\vec{a} \cdot \vec{a} \leqslant 0$.
  \end{itemize}
  
  Факты:
  \begin{itemize}
    \item $|\vec{a}| = \sqrt{ \vec{a} \cdot \vec{a}}$;
    \item $g_{ij} = \vec{e}_i \cdot \vec{e}_j$ -- метрическая матрица;
    \item $g^{ij} g_{jk} = \delta^i_k$;
    \item $\vec{e}^i = g^{ik} \vec{e}_k$ -- векторы взаимного базиса;
    \item $\vec{e}^i \cdot \vec{e}_j = (g^{ik} \vec{e}_k) \cdot \vec{e}_j = g^{ik} (\vec{e}_k \cdot \vec{e}_j) = g^{ik} g_{kj} = \delta^i_j$;
    \item Частный случай: если $\bar{\vec{e}}_i$ -- ортонормированный, т.е.
      $\vec{e}_i \cdot \vec{e}_j = \delta_{ij}$. Тогда $g_{ij} = \delta_{ij}$, тогда
      $g^{ij} g_{jk} = \delta^i_k$ и $g^{ij} = \delta^{ij}$. Тогда $\bar{\vec{e}}^i = \delta^{ik} \bar{\vec{e}}_k$
  \end{itemize}
\end{definition}

Векторное произведение в $\mathcal{E}_3$:
\[
  \vec{a} \times \vec{b} = \dfrac{1}{\sqrt{g}} \varepsilon^{ijk} a_i b_j \vec{e}_k
  = \sqrt{g} \varepsilon_{ijk} a^i b^j \skew{-7}\vec{e}^k.
\]
