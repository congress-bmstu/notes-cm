\paragraph{Програмные комплексы.}
\begin{enumerate}
	\item \texttt{ANSYS},
	\item \texttt{ЛОГОС},
	\item \texttt{Abaqus},
	\item \texttt{Манипула},
	\item \texttt{SolidWorks},
	\item \texttt{Компас 3D}.
\end{enumerate}

\paragraph{План курса МСС.} \textsc{Первый семестр}:
\begin{enumerate}
  \item Введение:
    \begin{enumerate}
      \item Объекты и методы МСС,
      \item Основные задачи МСС;
    \end{enumerate}
  \item Элементы тензорного анализа;
  \item Основополагающие аксиомы МСС;
  \item Кинематика МСС;
  \item Законы сохранения МСС:
    \begin{enumerate}
      \item закон сохранения массы,
      \item закон изменения количества движения (закон сохранения импульса),
      \item закон изменения момента количества движения,
      \item первый закон термодинамики,
      \item второй закон термодинамики,
      \item нулевой закон термодинамики;
    \end{enumerate}
  \item Определяющие соотношения;
  \item Замкнутые системы уравнений МСС;
  \item Соотношения на поверхностях сильных разрывов;
\end{enumerate}

\textsc{Второй семестр}:
\begin{enumerate}
  \item Основы механики деформируемого твердого тела (МДТТ);
  \item Основы механики жидкостей и газов (МЖГ).
\end{enumerate}



\section{Введение}
\subsection{Объекты МСС}
\paragraph{Что изучает МСС?}
Рассматривается физические (материальные) объекты. Выберем \textsl{геометрический способ}
описания объектов. Пусть $ L $ есть характерный размер (порядок длины) материального
объекта в метрах.
    
	\begin{figure}[H]
		\centering
		\includesvg[scale = 0.8]{lec01_scales}
	\end{figure}
    
    Объекты делятся на 
объекты естественного происхождения и искусственного происхождения.

    Верхний предел применимости МСС где-то между $10^7$\,м и $10^9$\,м. Там нарушается первый принцип
    МСС --- исчезает сплошность, поскольку между планетами пустота. Однако и
    здесь есть возможность применения обобщения классического МСС для
    релятивистких явлений (не рассматривается в курсе).

    Композиты --- особый вид твердых сред (материя).

    \paragraph{Итоги.}
    \begin{enumerate}
      \item \textsc{Границы применения} МСС $L_{\min} \leqslant L \leqslant
        L_{\max}$. При б\'{о}льших
        длинах астрофизика, при меньших --- физика микромира.

      \item \textsc{Сплошность.} Существование самоподобных характерных объемов
        в которых есть <<много>>
        вещества.

      \item \textsc{Области применения} законов МСС:
        \begin{itemize}[label=--]
          \item автомобилестроение,
          \item двигателестроение,
          \item авиастроение,
          \item ракетостроение,
          \item атомо техника,
          \item строительство,
          \item биомеханика (движение крови, работа сердца, трансплантация и
            т.д.),
          \item геомеханика (прогнозирование климата, тектоника и т.д.),
          \item композиты (прогнозирование свойств композитов).
        \end{itemize}
        \item
		  % \newline
		  \begin{quotation}
				\emph{Тервер говно, потому что использует грубые методы, не применяя информацию о внутренних
				законах исследуемого объекта. Поэтому он не применим в композитах. Углеродные волокна 
				валяются просто, есть что-то жидкое (связующее вещество) и оно не валяется и из этого 
      всего возникает крыло --- вот как это предказать?}
				\flushright--- Димитриенко Ю.\,И.
			\end{quotation}
    \end{enumerate}


\subsection{Методы МСС}
Рассмотрим следующие науки: МСС, физика, химия, математика.

\begin{enumerate}
  \item По объектам, которые изучает МСС, она просто \textsl{часть физики}.
  \item Если изучается движение сред с химическими реакциями, то это раздел МСС
    под названием
    \emph{механика
  многокомпонентых сред}.
  \item По методам построения законов МСС является математикой. Математика построена на аксиомах.
    МСС близка к математике, потому что она основана на аксиомах (подробнее см.
    далее):
  \begin{enumerate}[label=$\mathscr{A}_{\arabic*}$.]
      \item Аксиома сплошности.
      \item Рассматривается евклидово пространство $\mathcal{E}_3^a$.
      \item Аксиома существования абсолютного времени.
\end{enumerate}
\end{enumerate}



\section{Элементы тензорного анализа}
Основные пространства в МСС:
\begin{enumerate}
  \item Векторное (линейное) пространство $\mathcal{L}_n$.  
  \item Евклидово пространство $\mathcal{E}_n$.
  \item Точечно-евклидово (аффинное) пространство $\mathcal{E}_n^a$.
  \item Метрическое пространство $ \chi $.
\end{enumerate}


\subsection{Векторное (линейное) пространство}
\begin{definition}
  \emph{Векторным пространством} $\mathcal{L}_n$ мы называем множество, в котором введены две операции: сложение и
  умножение на число.
\end{definition}

\begin{example}
  Пространство элементарной геометрии $\mathbb{E}_2$ и $\mathbb{E}_3$.

  Множество геометрических объектов, состоящих из точек, прямых, плосткостей и из ... Геометрический
  вектор --- направленный отрезок.

  На множестве геометрических векторов введём отношение эквивалентности, такое что два вектора
  эквивалентны, если один получаются параллельным переносом. Тогда сложение векторов определим
  как сложение классов эквивалентости по правилу параллелограма. 
   Обозначим символом $V_2$ пространство свободных векторов в $\mathbb E_2$,
  и символом $V_3$ пространство свободных векторов в $\mathbb E_3$
\end{example}

\begin{example}
  Арифметическое пространство координатных столбцов $\mathbb{R}_n$.
  \[
    \begin{pmatrix}
      x_1 \\
      \vdots \\
      x_n
    \end{pmatrix} \in \mathbb{R}_n, \qquad
    \begin{pmatrix}
      x_1 \\
      \vdots \\
      x_n
    \end{pmatrix} + \begin{pmatrix}
      y_1 \\
      \vdots \\
      y_n
    \end{pmatrix} = 
    \begin{pmatrix}
      x_1 + y_1 \\
      \vdots \\
      x_n + y_n
    \end{pmatrix}, \qquad
    \lambda \begin{pmatrix}
      x_1 \\
      \vdots \\
      x_n
      \end{pmatrix} = \begin{pmatrix} \lambda x_1 \\ \vdots \\ \lambda x_n
    \end{pmatrix}.
  \]
\end{example}

В любом $\mathcal{L}_n$ существует (конечный) \textsl{базис}, то есть такая система векторов, что
\[
  \exists \mathbf{e_1}, \dots, \mathbf{e_n}\colon \forall \mathbf{a} \in
  \mathcal L_n\colon \exists
  a^i\colon  \mathbf{a} = a^i \mathbf{e}_i.
\]
Не существует такого набора чисел $s^1, \dots, s^n$, такого что не все $s^i = 0$
и при этом $s^i \mathbf{e}_i = 0$.

\begin{example}
  Базис в $\mathbb E_2$ ($V_2$). 
\end{example}

\paragraph{Замена базиса.} Рассмотрим базис $\mathbf{e}_i$ и $\mathbf{e}_i' =
Q_i^j \mathbf{e}_j = Q^k_i \mathbf{e}_k$. Здесь $Q^j_i$ --- матрица 
замены базисов ($\det Q^j_i \neq 0$), к которой существует обратная матрица
$P^j_i$, то есть $P^i_j Q^j_k = \delta^i_k$; 

Преобразование компоненты вектора при замене базиса производится по закону
\[
  \mathbf{a} = a^i \mathbf{e}_i = {a'}^j \mathbf{e'}_j = ({a'}^j Q^i_j)
  \mathbf{e}_i.
\]

\subsection{Евклидово пространство}

\begin{definition}
  \emph{Евклидовым пространством} $ \mathcal E_n $ называется линейное
  пространство $\mathcal{L}_n$, в котором дополнительно введена билинейная операция
  скалярного умножения $(\cdot)\colon \mathcal{E}_n^2 \to \mathbb{R}$ со
  свойствами
  \begin{enumerate}
    \item \textsc{Симметричность:} $\mathbf{a} \cdot \mathbf{b} = \mathbf{b} \cdot \mathbf{a}$,
    \item \textsc{Дистрибутивность:} $(\mathbf{a}+\mathbf{b}) \cdot \mathbf{c} = \mathbf{a} \cdot
      \mathbf{c} + \mathbf{b} \cdot \mathbf{c}$,
    \item \textsc{Положительная определённость:} $\mathbf{a} \cdot \mathbf{a} \geqslant 0$, причём $ \mathbf a \cdot
      \mathbf a = 0 $ тогда и только тогда, когда $ \mathbf a = 0 $.
  \end{enumerate}

  Перечислим ещё некоторые факты, известные из курса линейной алгебры.
  \begin{itemize}[label=--]
    \item \emph{длиной вектора} мы называем скаляр $|\mathbf{a}| = \sqrt{
      \mathbf{a} \cdot \mathbf{a}}$,
    \item \emph{метрическая матрица} определяется выражением $g_{ij} =
      \mathbf{e}_i \cdot \mathbf{e}_j$,
    \item символом $ g^{ij} $ обозначается матрица, обратная к метрической:
      $g^{ij} g_{jk} = \delta^i_k$,
    \item векторы $\mathbf{e}^i = g^{ik} \mathbf{e}_k$ называют \emph{векторами
      взаимного базиса},
    \item $\mathbf{e}^i \cdot \mathbf{e}_j = (g^{ik} \mathbf{e}_k) \cdot
      \mathbf{e}_j = g^{ik} (\mathbf{e}_k \cdot \mathbf{e}_j) = g^{ik} g_{kj} =
      \delta^i_j$,
    \item обозначив ортонормированный базис как $\bar{\mathbf{e}}_i$ (т.е.
      $\bar{\mathbf{e}}_i \cdot \bar{\mathbf{e}}_j = \delta_{ij}$), заметим, что
      для него $g_{ij} =
      \delta_{ij}$, а значит,
      $g^{ij} g_{jk} = \delta^i_k$ и $g^{ij} = \delta^{ij}$. В этом случае
      $\bar{\mathbf{e}}^i = \delta^{ik} \bar{\mathbf{e}}_k$.
  \end{itemize}
\end{definition}

Векторное произведение в $\mathcal{E}_3$ мы определим как
\[
  \mathbf{a} \times \mathbf{b} = \frac{1}{\sqrt{g}} \varepsilon^{ijk} a_i b_j \mathbf{e}_k
  = \sqrt{g} \varepsilon_{ijk} a^i b^j \skew{-7}\mathbf{e}^k,
\]
где $ g := \det (g_{ij}) $, а \emph{символ Леви -- Чевиты} $ \varepsilon_{ijk} $
равен чётности перестановки $ (ijk) $ (напр., $ \varepsilon_{123} =
\varepsilon_{231} = \varepsilon_{312} = 1 $, $
\varepsilon_{132} = -1 $, а $ \varepsilon_{122} = 0 $).
