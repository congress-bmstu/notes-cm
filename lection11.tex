\paragraph{Рассмотрим следующий вопрос:}
при рассмотрении кинематики (то есть движения сплошной среды)
мы предполагали существование закона движения сплошной среды $\vec{x} = \vec{x} (X^i, t)$, 
причём $\vec{x}(X^i, t)$ -- непрерывно-дифференцируемая (иногда дважды), а при $t=0$ она ещё
и взаимно-однозначная. Возможна ли такая ситуация, когда в $\mathcal{K}$ радиус-вектор
перестанет быть взаимно-однозначной функцией.

Цель этого раздела: вывести дифференциальные уравнения, которые гарантируют, если они
выполнены, взаимную однозначность радиус-вектора материальных точек. Иначе говоря, мы хотим
получить уравнения, которые гарантируют сохранение сплошности тела в $\mathcal{K}$.

\begin{definition}
  Необходимые и достаточные условия существования однозначной функции или, что тоже самое,
  вектора перемещений $\vec{u} = \vec{x} - \mathring{\vec{x}}$ называются
  \emph{условиями соместности деформаций} сплошной среды.
\end{definition}

\begin{utv}
  Условия совместности деформации сплошной среды $\Leftrightarrow$
  сплошная среда принадлежит точечно-евклидову пространству $\mathring{\varepsilon}_3^a$.
\end{utv}
\begin{proof}
  \begin{enumerate}
    \item необходимость. Пусть точка $M \in \mathring{\varepsilon}_3^a$ в $\mathcal{K}$,
      тогда по условию евклидовости для точки М существует радиус-вектор, но это и
      означает, что выполнены условия совместности.
    \item достаточность. Аналогично.
  \end{enumerate}
\end{proof}

Из этого утверждения следует, что если условия совместности деформаций не выполнены, то такая
сплошная среда не принадлежит точечно-евклидовому пространству.



\subsection{Условия интегрируемости дифференциальной формы}

\begin{definition}
  Дифференциальной формой называется выражение вида $\sum_{\alpha=1}^3 A_\alpha dX^\alpha$,
  где $A_\alpha$ -- некоторая функция $A_\alpha = A_\alpha(X^i)$.
\end{definition}

\begin{theorem}
  Функция $A$, 
\end{theorem}


\subsection{Первая форма условий совместности деформаций}

В формулах для дифференциальных форм вместо $A_\alpha$ могут быть и вектор-функции.
Тогда:
\[
  d\vec{x} = \vec{r}_i \, dX^i
\]

\begin{utv}
  Условия совместности деформаций мыполнены тогда и только тогда, когда в $\mathcal{K}$
  существуют локальные векторы базиса $\vec{r}_i$, то есть система вектор-функций
  $\vec{r}_i = \vec{r}_i(X^j, t)$, обладающие следующим свойствам:
  \begin{enumerate}
    \item линейно независимы;
    \item однозначны и гладкие в рассматриваемой области;
    \item 
  \end{enumerate}
\end{utv}
\begin{proof}
  \begin{enumerate}
    \item необходимость. Пусть существуют локальные векторы базиса, удовлетворяющие свойствам.
      Тогда можно вычислить производную от $r_\alpha$ по $X^\beta$:
      \[
        \dfrac{\partial r_\alpha}{\partial X^\beta} = \dfrac{\partial^2 \vec{x}}{\partial X^\alpha \partial X^\beta} = \dfrac{\partial r_\beta}{\partial X^\alpha},
      \]
      иначе говоря, выполнены условия взаимности, тогда $\dots$ $\vec{x} = \vec{x}(X^i, t)$ --
      взаимно-однозначная, то есть выполнены условия совместности деформаций.

    \item Пусть существует взаимно-однозначная функция $\vec{x} = \vec{x}(X^i, t)$,
      тогда последовательно вычисляем и проверяем вектора локального базиса.
  \end{enumerate}
\end{proof}

У этом утверждении условие взаимности имеет вид: $ \dfrac{\partial \vec{r}_\alpha}{\partial X^\beta} = \dfrac{\partial \vec{r}_\beta}{\partial X^\alpha} $.

\subsection{Вторая форма условия совместности деформаций}

Выберем теперь в качестве функции $A$ локальные векторы базиса.
% TODO дописать

\begin{utv}
  УСД выполнены тогда и только тогда, когда существуют символы Кристоффеля $\Gamma_{ij}^k$,
  удовлетворяющие уравнению $\tensor{R}{_n_i_j^k} = 0$.
\end{utv}

% TODO здесь ещё чото

\begin{definition}
  Тензор $\tensor[^4]{R} = R_{nijk} r^n \otimes r^i \otimes r^j \otimes r^k$ называется
  тензором \emph{Римана-Кристоффеля}.
\end{definition}
\begin{theorem}
  Тензор Римана-Кристоффеля -- действительно тензор 4-го ранга.
\end{theorem}
\begin{proof}
  Необходимо доказать, что $R'_{i_1 i_2 i_3 i_4} = R_{j_1 j_2 j_3 j_4} \tensor{Q}{^{j_1}_{i_1}}
  \tensor{Q}{^{j_2}_{i_2}} \tensor{Q}{^{j_3}_{i_3}} \tensor{Q}{^{j_4}_{i_4}}$, но доказать
  это будет достаточно трудоёмко. Поэтому применим \emph{косвенный признак тензора}:
  если в некотором выражении учавствуют некоторые тензоры и наш неизвестный индексный объект,
  например, $a_i = b_i + T_{ik} c^k$, где $a, b, c$ -- тензоры, тогда $T_{ik}$ -- тензор по этому
  признаку.

  Расммотрим $\forall \vec{a} = a^k \vec{r}_k$ -- вектор, его вторая ковариантная производная:
  \[
    \nabla_j \left( \nabla_i a^k \right) = \dots
  \]
  Тогда:
  \[
    \nabla_j (\nabla_i a^k) - \nabla_i (\nabla_j a^k) = \left(\dfrac{\partial \Gamma^k_{is}}{\partial X^j} - \dfrac{\partial \Gamma^k_{js}}{\partial X^i} + \Gamma^{k}_{jm} \Gamma^{m}_{is} - \Gamma^k_{im} \Gamma^m_{js}\right) a^s = \tensor{R}{_{jis}^k} a^s.
  \]

  Выводы:
  \begin{enumerate}
    \item т.к. $\tensor{R}{_{jis}^k} \equiv 0$, то
      $\nabla_j (\nabla_i a^k) - \nabla_i (\nabla_j a^k) = 0$ --
      \emph{переставимость ковариантных производных при повторном дифференцировании};
    \item $\tensor{R}{_{jis}^k} \equiv 0$ -- условие евклидовости пространства $\mathcal{E}_3^a$;
    \item так как в выражении
      $\nabla_j (\nabla_i a^k) - \nabla_i (\nabla_j a^k) = \tensor{R}{_{jis}^k} a^s$
      повторные ковариантные производные являются компонентами тензоров 3-го ранга, $a$ -- вектор,
      тогда $\tensor{R}{_{jis}^k}$ -- компоненты тензора 4-го ранга.
  \end{enumerate}
\end{proof}
