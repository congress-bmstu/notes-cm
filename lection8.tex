\subsection{Физический смысл компонент тензора напряжений Коши}

\paragraph{Физический смысл самого тензора Коши}
Тензор напряжений Коши $T$ вводится в актуальной конфигурации $\mathcal{K}$.
Какую бы площадку $d\Sigma$ мы не взяли в теле вокруг некоторой точки
-- на неё всегда действует сила ($d\mathcal{F}$) -- внутренняя поверхностная сила. 
Построим объект $\mathbf{t}_n = \dfrac{d\mathcal{F}}{d\Sigma}$ -- вектор напряжений.
Теорема №2 Коши говорит о том, что $\mathbf{t}_n = \mathbf{n}\cdot T$,
где тензор $T$ -- тензор напряжений Коши $T = \mathbf{r}_i \otimes \mathbf{t}^i$.

\paragraph{Физический смысл компонент}
Рассмотрим базис $\mathbf{r}_i$, по нему образуем базис $\hat{\mathbf{r}}_i$ --
процессом ортогонализации и ортонормирования. Тогда $T$ можно представить в этом
базисе $T = T^{ij} \mathbf{r}_i \otimes \mathbf{r}_j
= \hat{T}^{ij} \hat{\mathbf{r}}_i \otimes \hat{\mathbf{r}}_j$.

Для раскрытия физического смысла, рассмотрим элементарный объём $dV$ в форме куба,
рёбра которого ориентированы по $\hat{\mathbf{r}}_\alpha$. Тогда на каждой
грани $d\Sigma_\alpha$ ($d\Sigma_\alpha$ -- грань, ортогональная ребру $\hat{\mathbf{r}}_\alpha$)
этого куба действует внутренняя поверхностная сила $d\mathbf{\mathcal{F}}_\alpha$.
Тогда можно ввести вектор напряжений $\mathbf{t}_\alpha$ на $d\Sigma_\alpha$:
$\mathbf{t}_\alpha = \dfrac{d\mathbf{\mathcal{F}}_\alpha}{d\Sigma_\alpha}$.
По формуле Коши, $\mathbf{t}_\alpha = \mathbf{n}_\alpha \cdot T$, тогда:
$\dfrac{d\mathbf{\mathcal{F}}_\alpha}{d\Sigma_\alpha} = \mathbf{n}_\alpha \cdot T$.

Запишем это соотношение в базисе $\hat{\mathbf{r}}_\alpha$:
$d\mathbf{\mathcal{F}}_\alpha = d\hat{\mathbf{\mathcal{F}}}^i_\alpha \cdot\hat{\mathbf{r}}_i, \mathbf{n}_\alpha = \hat{\mathbf{n}}^i_\alpha \cdot \hat{\mathbf{r}}_i$.

Тогда:
\[
  \dfrac{d\hat{\mathbf{\mathcal{F}}}^i_\alpha \hat{\mathbf{r}}_i }{d\Sigma_\alpha}
= \mathbf{n}^i_\alpha \cdot \hat{\mathbf{r}}_i \cdot \hat{T}^{kj} \hat{\mathbf{r}}_j \otimes \hat{\mathbf{r}}_j.
\]
\[
  \dfrac{d\hat{\mathbf{\mathcal{F}}}^i_\alpha \hat{\mathbf{r}}_i }{d\Sigma_\alpha}
  = \hat{\mathbf{n}}^i_\alpha ^{kj} \delta_{ik} \hat{\mathbf{r}}_j
\]
\[
  \dfrac{d\hat{\mathbf{\mathcal{F}}}^j_\alpha \hat{\mathbf{r}}_i }{d\Sigma_\alpha}
  = \hat{\mathbf{n}}_{k\alpha} \hat{T}^{kj}
\]
-- выражение компонент тензора деформаций в данном базисе.
Так как $dV$ -- куб, ориентированный вдоль базиса $\hat{\mathbf{r}}_\alpha$,
поэтому компоненты нормалей $\hat{\mathbf{n}}_{k\alpha} = \delta_{k\alpha}$.
\[
  \delta_{k\alpha} \hat{T}^{kj} = \dfrac{d\mathbf{\mathcal{F}}^j_\alpha}{d\Sigma_\alpha}
\]
\[
  \hat{T}^{\alpha j} = \dfrac{d\mathbf{\mathcal{F}}^j_\alpha}{d\Sigma_\alpha}
\]

Рассмотрим отдельно диагональные компоненты:
\[
  \hat{T}^{\alpha \alpha} = \dfrac{d\mathbf{\mathcal{F}}^\alpha_\alpha}{d\Sigma_\alpha}
\]
$d\mathbf{\mathcal{F}}^\alpha_\alpha$ -- проекции на нормаль к грани $d\Sigma_\alpha$
-- $\mathbf{n}_\alpha$, поэтому $\hat{T}^{\alpha\alpha}$ -- \emph{нормальные напряжения}
-- проекция силы на нормаль, делённая на площадь площадки, перпендикулярной 
к этой нормали.

% TODO рисунок-иллюстрация, поясняющая почему это проекции и какие именно это
% проекции

Рассмотрим теперь $\hat{T}^{\alpha\beta} = \dfrac{d\hat{\mathbf{\mathcal{F}}}^\beta_\alpha}{d\Sigma_\alpha}$.
\[
  \begin{cases}
    \hat{T}^{\alpha\beta} = \dfrac{d\hat{\mathbf{\mathcal{F}}}^{\beta_\alpha}}{d\Sigma_\alpha},\\
    \hat{T}^{\alpha\gamma} = \dfrac{dd\hat{\mathbf{\mathcal{F}}}^{\gamma}_\alpha}{d\Sigma_\alpha}.
  \end{cases}
\]
-- касательные напряжения.

Всего для такого кубика получилось
3 нормальных напряжения (на каждую площадку) и 6 касательных напряжений (по две
на каждую площадку). На данном этапе. ничего не известно про симметрию (к сожалению).
В общем случае тензор $T$ -- не симметричен.

\paragraph{Размерность}
Рекомендация: почитайте второй том МСС, если интересуетесь размерностями.
Вообще есть 4 размерности: секунды, метры, килограммы, кельвины -- почемы их 4?
Оказывается, что их 4, потому что они все следуют из математики! То есть
оказывается, что есть ровно 4 масштабных коэффициента, которые допускают 
масштабирование системы координат. Оказывается, что всех размерностей 
в электродинамике не существует, они все выводятся из механических размерностей.
(правда все они в дробных степенях, чего другие обычные механические размерности
никогда не делают)

Понятие размерности относят только к скалярам и компонентам векторов, но не к тензорам.
Те размерности величин компонент векторов и тензоров обычно рассматриваются в 
ортонормированном базисе.

У компонент тензора напряжения Коши в декартовом ортонормированном базисе:
\[
  \left[ \hat{T}^{\alpha\beta} \right] = \dfrac{\text{Н}}{\text{м}^2} = \text{Па}.
\]
Также часто используется $\text{кгс} / \text{мм}^2 = \dots$.


\subsection{Физический смысл компонент тензора напряжений Пиолы-Кирхгофа}

Напомним, что данный тензор вводиться для некоторой площадки $d\mathring{\Sigma}$
в отсчётной конфигурации, которая переходит в $d\Sigma$. Тогда:
$\mathbf{n} \cdot T d\Sigma = \mathbf{t}_n \cdot d\Sigma =
d\mathbf{\mathcal{F}}
= \mathring{\mathbf{t}}_n d\mathring{\Sigma} = \mathring{\mathbf{n}} \cdot P \cdot d\mathring{\Sigma},$
где тензор $P = \sqrt{\dfrac{g}{\mathring{g}}} F^{-1} \cdot T$ -- тензор Пиолы-Кирхгофа.

Рассмотрим в $\mathring{\mathcal{K}}$ элементарный объем в виде куба, 
ориентированного по векторам базиса $\hat{\mathring{\mathbf{r}}}_\alpha$ --
ортонормированный базис, в актуальной конфигурации этот объём перейдет в
параллелограм образованный векторами $\mathbf{r}_\alpha$.
На каждую грань $d\Sigma_\alpha$ этого параллелограма действует сила
$d\mathbf{\mathcal{F}}_\alpha$, тогда:
$\mathring{\mathbf{n}}_\alpha \cdot P = \dfrac{d\mathbf{\mathcal{F}}}{d\mathring{\Sigma}_\alpha}$.
Запишем $\mathring{\mathbf{n}}_\alpha$, $P$, $d\mathbf{\mathcal{F}}$ в базисе
$\hat{\mathring{\mathbf{r}}_\alpha}$:
\[
  \mathring{n}_\alpha = \hat{\mathring{n}}^i_\alpha \cdot \mathring{\mathbf{r}}_i
  \quad
  \dots
  \quad
  \dots
  \quad
  d\mathbf{\mathcal{F}}_\alpha = d\hat{\mathbf{\mathcal{F}}}^i_\alpha \cdot \hat{\mathbf{r}}_i
\]
% TODO дописать

\[
  \hat{\mathring{n}}_{\alpha k} \cdot \hat{\mathring{P}}^{kj} = \dfrac{d\hat{\mathring{\mathcal{F}}}^i_\alpha}{d\mathring{\Sigma}_\alpha}.
\]

Так как $d\mathring{V}$ -- куб, то $\mathring{n}_{\alpha k } = \delta_{\alpha k}$,
поэтому:
\[
  \hat{\mathring{P}}^{\alpha k} = \dfrac{d\hat{\mathring{\mathcal{F}}}^i_\alpha}{d\mathring{\Sigma}_\alpha}.
\]

Аналогично тому, как мы рассматривали тензор Коши, рассмотрим сначала диагональные
элементы:
\[
  \hat{\mathring{P}}^{\alpha\alpha} = \dfrac{d\hat{\mathring{F}}^{\alpha}_\alpha}{d\mathring{\Sigma}_\alpha}
\]
-- $d\hat{\mathring{\mathcal{F}}}^\alpha_\alpha$ -- взяли прообраз вектора силы
$d\mathbf{\mathcal{F}}_\alpha$ в отсчетной конфигурации, и его спроецировали на 
нормаль к прообразу площадки, на которую он действует, поэтому диагональные
компоненты называются нормальными напряжениями.

Аналогично получаем, что 
\[
  \begin{cases}
    \hat{\mathring{P}}^{\alpha\beta} = \dfrac{d\hat{\mathring{F}}^\beta_\alpha}{d\mathring{\Sigma}_\alpha}, \\
    \hat{\mathring{P}}^{\alpha\gamma} = \dfrac{d\hat{\mathring{F}}^\gamma_\alpha}{d\mathring{\Sigma}_\alpha}.
  \end{cases}
\]
-- касательные напряжения.


\subsection{Уравнения движения}
Рассмотрим снова уравнение изменения количества движения в интегральной форме:
\[
  \dfrac{d}{dt} \underbrace{\int\limits_V \rho \mathbf{v} dV}_{\equiv I} = 
  \int\limits_V \rho \mathbf{f} dV + \int\limits_\Sigma \underbrace{\mathbf{t}_n d\Sigma}_{= \int\limits_{\Sigma} d\mathbf{\mathcal{F}}_\Sigma}.
\]
Согласно правилу дифференцирования интеграла по подвижному объёму:
\[
  \int\limits_V \rho \left( \dfrac{d\mathbf{v}}{dt} - \mathbf{f} \right) dV
  - \int\limits_\Sigma \mathbf{n} \cdot T d\Sigma = 0,
\]
по формуле Гаусса-Остроградского:
\[
  \int\limits_V \rho \left( \dfrac{d\mathbf{v}}{dt} - \mathbf{f} \right) dV
  - \int\limits_V \nabla \cdot T \, dV = 0,
  \Rightarrow
  \int\limits_V \left(
    \rho \left( \dfrac{d\mathbf{v}}{dt} - \mathbf{f} \right)
  - \nabla \cdot T \right) \, dV = 0,
\]

Тогда, благодаря произвольности объёма интегрирования,
\begin{equation}\label{lec_8:eq_dv_euler}
  \rho \dfrac{d\mathbf{v}}{dt}
  = \nabla \cdot T + \rho \cdot \mathbf{f}
\end{equation}
-- уравнение движения в Эйлеровом описании (в полных дифференциалах).


\begin{remark}

  \begin{enumerate}
    \item Если вдруг тензор напряжений нулевой, или, например, $T \equiv a E, a = const$,
      т.е. $\nabla \cdot T \equiv 0$, тогда:
      \[
        \rho \dfrac{d\mathbf{v}}{dt} + \rho \mathbf{f}.
      \]

    \item Если вдруг $\mathbf{v} \equiv 0$ (квазистатические задачи), то
      \[
        \nabla \cdot T = \rho \mathbf{f} = 0
      \]
      --  уравнение равновесия.
  \end{enumerate}
\end{remark}

Т.к. $\dfrac{d\mathbf{v}}{dt} = \dfrac{\partial \mathbf{v}}{\partial t} (x^i, t) + \mathbf{v} \cdot \nabla \otimes \mathbf{v}$ -- полная производная по времени в эйлеровом
описании, то \eqref{lec_8:eq_dv_euler} принимает вид:
\[
  \rho \dfrac{\partial \mathbf{v}}{\partial t} + \rho \mathbf{v} \cdot \nabla \otimes \mathbf{v}
  + \mathbf{v} \cdot
  \underbrace{\left( \dfrac{\partial \rho}{\partial t} + \nabla \cdot \rho\mathbf{v} \right)}_{=0 \text{ уравнение неразрывности}}
  = \nabla \cdot T + \rho \mathbf{f}.
\]
\[
  \dfrac{\partial \rho \mathbf{v}}{\partial t} + \nabla \cdot \left( \rho \mathbf{v}\otimes \mathbf{v} \right) = \nabla \cdot T + \rho \mathbf{f}
\]
-- дивергентная форма уравнения движения в Эйлеровом описании (такая форма,
что перед всеми производными коэффициент единица). Эту форму чаще используют в 
газовой динамике.

\paragraph{Уравнение движения в Лагранжевом (материальном) описании}

Переход от Эйлерова описания к Лагранжеву описанию (и наоборот) осуществляется
через интегральную форму.
\[
  \int\limits_V \rho \mathbf{a} \, dV
  = \int\limits_V \mathring{\rho} \left( \dfrac{\rho}{\mathring{\rho}} \right) \mathbf{a} \, dV
  = \int\limits_V \mathring{\rho} \sqrt{\dfrac{g}{\mathring{g}}} \mathbf{a} \, dV
  = \int\limits_{\mathring{V}} \mathring{\rho} \mathbf{a} \, d\mathring{V}.
\]
\[
  \int\limits_{\mathring{V}} \mathring{\rho} \left( \dfrac{d\mathbf{v}}{dt} -\mathbf{f} \right) \, d\mathring{V} = \int\limits_{\mathring{\Sigma}} \mathbf{n} \cdot P \, d\mathring{\Sigma}.
\]

% TODO дописать одно уравнение тут

\[
  \int\limits_{\mathring{V}} \mathring{\rho} \left( \dfrac{d\mathbf{v}}{dt} - \mathbf{f} - \mathring{\nabla} \cdot P \right) \, d\mathring{V} = 0
\]
в силу произвольности $\mathring{V}$:
\[
  \mathring{\rho} \dfrac{d\mathbf{v}}{dt} = \mathring{\rho} \mathbf{f} + \mathring{\rho} \mathring{\nabla} \cdot P
\]
-- уравнение движения в Лагранжевом описании.


\subsection{Закон изменения момента количества движения}
Для импульса мы получили такую аналогию с системами материальных точек:
\[
  \dfrac{d}{dt} \sum_i m_i \mathbf{v}_i = \sum_i \mathbf{f}_i
  \rightarrow
  \dfrac{d}{dt} \int\limits_V \rho d\mathbf{v} = \int_V \mathbf{f}\, dV + \int_V \mathbf{t}_n \, d\Sigma.
\]

Для системы точек закон изменения момента количества движения формулируется так:
$\mathbf{x}_i \times \mathbf{f}_i$ -- момент импульса (момент количества движения).
\[
  \dfrac{d}{dt} \sum_{j=1} \mathbf{x}_i \times m_i \mathbf{v}_i = \sum_i \mathbf{x}_i \times \mathbf{f}_i
\]
-- уравнение изменения момента количества движения систем точек.

\begin{definition}
  $\mathbf{k}' \equiv \int\limits_V \mathbf{x} \times \mathbf{v} \, dm$ -- момент
  количества движения сплошной среды.
  Преобразуя интеграл, можно получить также:
  $\mathbf{k}' \equiv \int\limits_V \rho \mathbf{x} \times \mathbf{v} \, dV$.
\end{definition}

\begin{definition}
  $\mathbf{\mu}_m' \equiv \int\limits_V \mathbf{x} \times d\mathbf{\mathcal{F}}
  = \int\limits_V \rho \mathbf{x} \times \mathbf{f} \, dV$
  -- вектор момента массовых сил.

  $\mathbf{\mu}_\Sigma' \equiv \int\limits_\Sigma \mathbf{x} \times d\mathbf{\mathcal{F}}
  = \int_\Sigma \mathbf{x} \times \mathbf{t}_n \, d\Sigma$ 
  -- вектор момента поверхностных сил.

  $\mathbf{\mu}' \equiv \mathbf{\mu}'_m + \mathbf{\mu}'_\Sigma$
  -- суммарный вектор моментов.
\end{definition}

\paragraph{Аксиома 6} (Закон изменения момента количества движения).
Для всякой сплошной среды существуют две аддитивные векторные функции:
  $\mathbf{k}''$ -- вектор собственного момента количества движения,
  $\mathbf{\mu}''$ -- вектор собственных моментов,
которые удовлетворяют следующему уравнению:
\[
  \dfrac{d\mathbf{k}}{dt} = \mathbf{\mu},
\]
где $\mathbf{k} = \mathbf{k}' + \mathbf{k}''$ -- полный вектор момента количества движения;
$\mathbf{\mu} = \mathbf{\mu}' + \mathbf{\mu}''$ -- полный вектор моментов.

\[
  \dfrac{d\mathbf{k}'}{dt} = \mathbf{\mu}'
\]


\subsection{Интегральная формулировка закона изменения момента количества движения СС}
% TODO кусок лекции



\begin{definition}
  $\mathbf{k}_m \equiv \dfrac{d\mathbf{k}''}{dm}$ -- плотность собственного момента количества движения.

  $\mathbf{h}_m \equiv \dfrac{d \mathbf{\mu}''}{dm}$ -- плотность собственного массового момента.
  
  $\mathbf{h}_\Sigma \equiv \dfrac{d \mathbf{\mu}''}{d\Sigma}$ -- плотность собственного поверхностного момента.
\end{definition}

Тогда $\mathbf{k} = \int\limits_V d\mathbf{k}'' = \int\limits_V \rho \, d\mathbf{k}_m$,
и т.д., и т.п.
Используем введённые величины для преобразования закона сохранения момента количества движения:
\[
  \dfrac{d}{dt}\int\limits_V \underbrace{\left( \mathbf{x}\times\rho\mathbf{v}+\rho\mathbf{k}_m \right)}_{\mathbf{k}' + \mathbf{k}''} \, dV
  = \int\limits_V \left( \mathbf{x}\times\rho\mathbf{f}+\rho\mathbf{h}_m \right) \, dV
  + \int\limits_\Sigma \left( \mathbf{x}\times\mathbf{t}_n + \mathbf{h}_\Sigma \right) \, d\Sigma
\]
-- интегральная форма закона сохранения момента количества движения сплошных сред.


\subsection{Дифференциальная формулировка закона изменения момента количества движения СС}  

\paragraph{Обобщённая теорема Коши}
Если
\[
  \dfrac{d}{dt} \int\limits_V \rho A \, dV 
  = \int\limits_V \rho C \, dV + \int\limits_\Sigma B_n \, d\Sigma
\]
то применяя этот закон к элементарному объёму $dV$ -- тетраэдру и стягивая этот
тетраэдр $dV \to 0$, то интегралы по $V$ -- $O(h^3)$, а интеграл по $\Sigma$ -- $O(h^2)$,
следовательно, 
\[
  \sum_{\alpha=1}^4 B_{n\alpha} = 0.
\]

Применяя полученную теорему к ЗСМКД, получаем, что:
\[
  \mathbf{h}_\Sigma + \mathbf{x} \times \mathbf{t}_n = \mathbf{n} \cdot \tilde M, \quad
  \mathbf{t}_n = \mathbf{n} \cdot T.
\]

Таким образом, согласно обобщенной теореме Коши, $\exists$ тензор $\tilde M$ -- тензор момента напряжения.
