\subsection{Дифференциальная форма закона СЭ}

\[
  \dfrac{\partial \rho \varepsilon}{\partial t}  + \nabla \cdot \left( \rho \mathbf{v} \varepsilon - T \mathbf{v} + \mathbf{q} \right) = \rho \mathbf{f} \cdot \mathbf{v} + \rho q,
\]
где $\varepsilon = e + \dfrac{v^2}{2}$ -- плотность полной энергии; $e = \dfrac{dU}{dm}$, 
$K = \int_V \dfrac{\rho \mathbf{v}^2}{2} \, dV = \int_V \dfrac{V^2}{2} \, dm$.

$\mathbf{a} = \rho \mathbf{v} \varepsilon - T \mathbf{v} + \mathbf{q} \equiv a^i \mathbf{r}_i$.
$a^i = \rho v^i \varepsilon - \tensor{T}{^i_j} v^j + q^i$.

\[
  \rho \dfrac{d\varepsilon}{dt} = \nabla \cdot (T \mathbf{v} - \mathbf{q}) + \rho \mathbf{f} \cdot \mathbf{v} + \rho q_n
\]
-- уравнение энергии в полных дифференциалах.

\begin{theorem}[<<живых сил>>]
  <<Живой силой>> называется кинетическая энергия.
  Уравнение движения сплошной среды в полных дифференциалах:
  \[
    \rho \dfrac{d\mathbf{v}}{dt} = \nabla \cdot T + \rho \mathbf{f},
  \]
  домножим его скалярно на $\mathbf{v}$:
  \[
    \rho \dfrac{d}{dt} \left( \dfrac{v^2}{2} \right) = \nabla \cdot T \cdot \mathbf{v} + \rho \mathbf{f} \cdot \mathbf{v}.
  \]

  % TODO много дописать
\end{theorem}

\begin{remark}
  Введём обозначения:
  \begin{enumerate}
    \item $K = \int_V \rho \dfrac{v^2}{2} \, dV$ -- кинетическая энергия сплошной среды;
    \item $W_m = \int_V \rho \mathbf{f} \cdot \mathbf{v} \, dV$ -- мощность внешних массовых сил;
    \item $W_{(i)} = - \int_V T \cddot \nabla \otimes \mathbf{v}^T \, dV$ -- мощность внутренних поверхностных сил;
    \item $\omega_{(i)} \equiv - T \cddot \nabla \otimes \mathbf{v}^T$ -- удельная мощность внутренних поверхностных сил, причём если $T$ -- симметричный, то
      \[
        \omega_{(i)} = \dfrac{1}{2} \left(  T \cddot \nabla \otimes \mathbf{v}^T + T \cddot \nabla \mathbf{v}^T \right) = T \cddot D,
      \]
      где тензор $D$ называется \emph{тензором скоростей деформации}.
  \end{enumerate}
  С учётом введённых выше обозначений, закон СЭ принимает вид:
  \[
    \dfrac{dK}{dt} = W_\Sigma + W_{(i)} + W_m
  \]
  -- теорема <<живых сил>> (теорема о кинетической энергии).
\end{remark}

В случае абсолютно твёрдого тела, мощность внутренних сил равна нулю: $W_{(i)} \equiv 0$.



\subsection{Уравнение притока тепла}

Вычтем из уравнения для полной энергии уравнение локальной теоремы о кинетической энергии:
\[
  \rho \dfrac{de}{dt} = - \nabla \mathbf{q} + \rho q_n + T \cddot \nabla \mathbf{v}^T
\]
-- \emph{локальное уравнение притока тепла или уравнение изменения плотности внутренней энергии}.
Проинтегрируем его:
\[
  \int_V \rho \dfrac{de}{dt} \, dV = - \int_V \nabla \cdot \mathbf{q} \, dV + \int_V \rho q_m \, dV
  + \int_V T \cddot \nabla \otimes \mathbf{v}^T \, dV
\]
С учётом наших обозначений, получаем интегральную формулировку уравнения притока тепла:
\[
  \dfrac{dU}{dt} = Q_\Sigma + Q_m - W_{(i)}.
\]
иногда обозначают $Q = Q_\Sigma + Q_m$.

\begin{align*}
  dU = Q dt - W_{(i)} dt; \\
  dU = \delta Q - \delta A_{(i)}; \\
  \delta Q = dU + \delta A_{(i)}.
\end{align*}


\subsection{Закон сохранения энергии в начальной конфигурации}

Переход от материального (лагранжевого) описания к пространственному (эйлерову) описанию
осуществляется с помощью интегральной формулировки.

Напомним интегральную формулировку закона сохранения энергии:
\[
  \dfrac{d}{dt} \int_V \rho \left( e + \dfrac{v^2}{2} \right) \, dV
  = - \int_\Sigma \mathbf{n} \cdot \mathbf{q} \, d\Sigma
  + \int_V \rho q_m \, dV
  + \int_V \rho \mathbf{f} \cdot \mathbf{v} \, dV
  + \int_\Sigma \mathbf{n} \cdot \left( T \cdot \mathbf{v} \right) \, d\Sigma.
\]

Так как:
\begin{align*}
  &\int_V \rho \left( e + \dfrac{v^2}{2} \right) \, dV
  = \int_{\mathring{V}} \mathring{\rho} \left( e + \dfrac{v^2}{2} \right) \, d\mathring{V}; \\
  &\int_\Sigma \mathbf{n} \cdot T \cdot \mathbf{v} \, d\Sigma
  = \int_\Sigma \mathbf{v}^T \left( \mathbf{n} \cdot T \right) \, d\Sigma
  = \int_\Sigma \mathbf{v} \cdot d\mathcal{\mathbf{F}}
  = \int_\Sigma \mathbf{v} \cdot \mathring{\mathbf{t}}_n \, d\mathring{\Sigma}
  = \int_{\mathring{\Sigma}} \mathbf{v} \cdot \mathring{\mathbf{n}} \cdot P \, d\mathring{\Sigma} \\
  &\int_\Sigma \mathbf{n} \cdot \mathbf{q} \, d\Sigma
  = \left|\, \begin{aligned} \mathring{q}_n = \dfrac{dQ_\Sigma}{d\mathring{\Sigma}} \\ \mathring{q}_n = - \mathring{\mathbf{n}} \cdot \mathring{\mathbf{q}} \end{aligned} \,\right| 
  = \int_{\mathring{\Sigma}} \mathring{\mathbf{n}} \cdot \mathring{\mathbf{q}} \, d\mathring{\Sigma}
\end{align*}

Тогда получаем представление уравнения изменения энергии в отсчётной конфигурации:
\[
  \dfrac{d}{dt} \int_{\mathring{V}} \mathring{\rho} \left( e+\dfrac{v^2}{2} \right) \, d\mathring{V}
  = \int_{\mathring{\Sigma}} \mathring{\mathbf{n}} \cdot \mathbf{q} \, d\mathring{\Sigma}
  + \int_{\mathring{V}} \rho q_m \, d\mathring{V}
  + \int_\mathring{V} \mathring{\rho} \mathbf{f} \cdot \mathbf{v} \, d\mathring{V}
  - \int_{\mathring{\Sigma}} \mathring{\mathbf{n}} \cdot P \cdot \mathbf{v} \, d\mathring{\Sigma}
\]
-- интегральный закон сохранения энергии в материальном описании.

По формуле Остроградского-Гаусса:
\[
  \int_{\mathring{V}} \left(  \rho \dfrac{d}{dt} \left( e+\dfrac{v^2}{2} \right) + \mathring{\nabla} \cdot \mathring{\mathbf{q}} - \mathring{\rho} q_m - \mathring{\rho} \mathbf{f} \cdot \mathbf{v} + \mathring{\nabla} \cdot \left( P\cdot\mathbf{v} \right)  \right) \, d\mathring{V} = 0
\]
в силу произвольности $V$:
\[
  \rho \dfrac{d}{dt} \left( e+\dfrac{v^2}{2} \right) + \mathring{\nabla} \cdot \mathring{\mathbf{q}} - \mathring{\rho} q_m - \mathring{\rho} \mathbf{f} \cdot \mathbf{v} + \mathring{\nabla} \cdot \left( P\cdot\mathbf{v} \right) = 0
\]
-- уравнение энергии (локальная формулировка) в материальном (лагранжевом) описании.

все функции здесь зависят от $(X^i, t)$ или $(\mathring{x}^i, t)$, причём
\[
  \dfrac{d}{dt} \left( e+\dfrac{v^2}{2} \right) = \dfrac{\partial }{\partial t} \left( e+\dfrac{v^2}{2} \right) |_{x^i}
\]
-- полная производная совпадает с частной.



\subsection{Нулевой закон термодинамики}

\paragraph{Аксиома 8 (Закон существования абсолютной температуры)}
Для всякой материальной точки $M$ любой сплошной среды для любого момента времени существует
скалярная положительная функция, обозначаемая $\Theta(X^i, t) > 0$, которая называется
\emph{абсолютной температурой}.

\begin{remark}
  Этот закон локальный.
\end{remark}
\begin{remark}
  Температура $\Theta$ в законе сохранения не повляется явным образом.
\end{remark}
\begin{remark}
  $\exists \Theta_{min} = 0$
\end{remark}



\subsection{Второй закон термодинамики}

\paragraph{Аксиома 9 (Второй закон термодинамики)}
Для всякой сплошной среды в любой момент времени существуют 2 скалярные аддитивные
функции:
\begin{enumerate}
  \item $H$ -- энтропия;
  \item $\bar{Q}^* \geqslant 0$ -- производство энтропии за счёт внутренних источников.
\end{enumerate}
Которые удовлетворяют следующему дифференциальному уравнению:
\[
  \dfrac{dH}{dt} = \bar{Q} + \bar{Q}^*,
\]
где $\bar{Q} = \bar{Q}_m + \bar{Q}_\Sigma$,
а $\bar{Q}^i \geqslant 0$ -- неравенство Планка.

\begin{remark}
  $\dfrac{dH}{dt} \geqslant \bar{Q}$ -- неравенство Клаузиуса.
\end{remark}


\subsection{Интегральная формулировка 2-го закона термодинамики}

В силу аддитивности $H$, можно записать $H = \int_V dH$, где $dH$ -- энтропия элементарного
объёма $dV$. Аналогично, $\bar{Q}^* = \int_V d\bar{Q}^*$, где $d\bar{Q}^*$ -- производство
энтропии за счёт внешних источников в $dV$.

\begin{definition}
  Введём величины:
  \begin{enumerate}
    \item $\eta = \dfrac{dH}{dm}$ -- плотность энтропии;
    \item $q^* = \theta \dfrac{d\bar{Q}^*}{dm} \geqslant 0$
      -- плотность внутреннего производства энтропии.
  \end{enumerate}
\end{definition}

Тогда можно получить интегральную формулировку второго закона термодинамики:
\[
  \dfrac{d}{dt} \int_V \rho \eta \, dV = \int_V \dfrac{\rho (q_m + q^*)}{\theta} \, dV
  - \int_\Sigma \dfrac{\mathbf{n} \cdot \mathbf{q}}{\theta} \, d\Sigma.
\]


\subsection{Дифференциальная формулировка 2-го закона термодинамики}

\[
  \rho \dfrac{d\eta}{dt} = - \nabla \cdot \left( \dfrac{\mathbf{q}}{\theta} \right) + \rho \dfrac{q_m + q^*}{\theta}
\]


Так как $\nabla \left( \dfrac{\mathbf{q}}{\theta} \right) = \dfrac{\nabla \mathbf{q}}{\theta} + \mathbf{q} \cdot \dfrac{1}{\theta^2} \nabla \theta$, то:
\begin{align*}
  \rho \dfrac{d\eta}{dt} =
  - \dfrac{\nabla \mathbf{q}}{\theta}
  + \dfrac{1}{\theta^2} \mathbf{q} \nabla \theta
  + \rho \dfrac{q_m + q^*}{\theta} \\
  \rho \theta \dfrac{d\eta}{dt} =
  - \nabla \mathbf{q}
  + \rho (q_m + q^*)
  + \dfrac{1}{\theta} \mathbf{q} \nabla \theta
\end{align*}

\begin{definition}
  Функция $\omega^* = \rho q^* + \dfrac{1}{\theta} \mathbf{q} \nabla \theta$ называется
  \emph{функцией диссипации (функцией рассеивания энергии)}
\end{definition}

\paragraph{Аксиома 9a} Для любой точки $M$ в любой сплошной среде в любой момент времени всегда выполнается неравенство Фурье: $\mathbf{q} \cdot \nabla \theta \leqslant 0$.

В терминах функции диссипации $\rho q^* = \omega^* - \dfrac{1}{\theta} \mathbf{q} \nabla \theta \geqslant 0$. Производство энтропии за счёт внутренних источников $q^*$ состоит из двух частей
(две причины внутреннего производства энтропии):
\begin{enumerate}
  \item За счёт функции диссипации энергии, которая определяется тензором деформации, температурой,
    но не зависит от $\nabla \alpha$: $\omega^* = \omega^* ( F, \theta )$ (будет доказано 
    позже). Эта функция определяет внутреннее производство энтропии за счёт внутреннего трения.
    \begin{itemize}
      \item Примером этого может послужить резкое сгибание твёрдого тела -- линия сгиба заметно
        нагревается;
      \item Вторым примером может послужить резкое движение бруска по поверхности -- поверхность 
        соприкосновения нагревается;
      \item Сварка трением;
      \item Консольно закреплённая балка, при отклонении от положения равновесия и отпускании,
        амплитуда колебаний будет уменьшаться как раз за счёт диссипации.
    \end{itemize}
    
  \item Если тело не движеться, то градиент деформации $F = E$, а $w^* \equiv 0$; но есть нагрев
    тела, причём предположим, что $\theta = \theta (X^i, t)$ и $\nabla \theta \neq 0$. Тогда
    появляется внутреннее производство энтропии.
\end{enumerate}

Если $\theta = \theta(t)$, т.е. $\nabla \theta \equiv 0$, тогда 
\[
  \rho q^* = \omega^* \geqslant 0
\]
-- неравенство диссипации.

\subsection{Тепловые машины и КПД}

\begin{definition}
  \emph{Тепловые машины} -- устройства (совокупность СС), которые преобразуют тепло в механическую
  работу (это различные двигатели), или наоборот (нагревательные или холодильные устройства).
  Или машины, которые включают процессы в обе стороны (это $\dots$)
\end{definition}

Тепловые машины можно разделить на три типа:
\begin{enumerate}
  \item Область $V$ (рабочее тело) состоит из одних и тех же материальных частиц.
    Примером таких машин служит поршень, газ под которым состоит из одних и тех же мат частиц.
  \item Рабочее тело $V$ не меняется во времени, но в этой области находится различные
    материальные точки. В качестве примера камера с горючим, в которое поступает горючее,
    уходят продукты горения.
  \item $V$ -- переменная, материальные точки разные. Двигатели, в которых под поршень поступает
    горючее, уходят продукты горения.
\end{enumerate}

Для простоты рассмотрим только тепловые машины первого типа.
Запишем 1-ый закон термодинамики (уравнение притока тепла):
\[
  \dfrac{dU}{dt} = Q - W_{(i)} \Leftrightarrow 
  U(t) = U(t_1) + \int\limits_{t_1}^t Q(\tau) \, d\tau - \int\limits_{t_1}^t W_{(i)}(\tau)\, d\tau
\]
Обозначим $\Delta U = U(t) - U(t_1)$ -- изменение внутренней энергии,
$C = \int\limits_{t_1}^t Q(\tau) \, d\tau$ -- суммарное тепло за время, 
$A_{(i)} = \int\limits_{t_1}^t W_{(i)}(\tau) \, d\tau$ -- работа внутренних сил.

Тогда получим: $\Delta U + A_{(i)} = C$.

Проинтегрируем аналогично второй закон термодинамики:
\[
  \Delta U = H(t) - H(t_1), \Delta \geqslant \bar{C},
\]
где $\bar{C} = \int\limits_{t_1}^t \bar{Q} \, d\tau$ -- суммарное производство энтропии внешних источников за время.
