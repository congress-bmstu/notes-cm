\section{Полная система законов сохранения}
Введём классификацию форм записи законов сохранения
\begin{table}[]
\centering
\label{tab:my-table}
\begin{tabular}{|ccccc|}
\hline
\multicolumn{5}{|c|}{Формы записи законов сохранения} \\ \hline
\multicolumn{3}{|c|}{\begin{tabular}[c]{@{}c@{}}В эйлеровом \\ (пространственном) описании\end{tabular}} &
  \multicolumn{2}{c|}{\begin{tabular}[c]{@{}c@{}}В лагранжевом \\ (материальном) описании\end{tabular}} \\ \hline
\multicolumn{1}{|c|}{В полных $ \mathbf{d} $} &
  \multicolumn{1}{c|}{В дивергентной форме} &
  \multicolumn{1}{c|}{$ \int $ форма} &
  \multicolumn{1}{c|}{В полных $ \mathbf{d} $} &
  $ \int $ форма \\ \hline
\end{tabular}
\caption{}
\end{table}

\begin{remark} 
  Рассматриваем только неполярные среды, то есть  
  \[
    T = T^{\mathsf T}.
  \]
  При этом исключаем у рассматриваемого закона иззменения моментов количества
  движения, то есть $ T = T^{\mathsf{T}} $ --- это оно и есть.
\end{remark}

\begin{remark}
  Все законы сохранения записываются в единой универсальной для всех всех форме.
\end{remark}

%TODO
\paragraph{1. Эйлерово описание, форма в полных дифференциалах.}
\[
    \rho \frac{d\bar A_\alpha}{dt} = \nabla \cdot B_\alpha + \rho C_\alpha, \quad
    (\alpha = 1, \ldots, 6 \text{ --- номер группы уравнений}).
\]
Здесь $ \bar A_\alpha $, $ B_\alpha $, $ C_\alpha $ --- обобщённые координатные
столбцы (векторами лучше не называть). Выпишем их: 
\begin{gather*}
  \bar{A}_\alpha = (1/\rho, \mathbf{v}, \varepsilon, \eta, \mathbf{u},
  F^{\mathsf T})^{\mathsf T}, \quad B_\alpha = (\mathbf{v}, T, T\cdot
  \mathbf{v}-\mathbf{q}, -\mathbf{q}/\theta, 0, \rhoF^{\mathsf T}\otimes
  \mathbf{v} )^{\mathsf T}, \\ C_\alpha = (0, \mathbf{f}, \mathbf{f}\cdot
  \mathbf{v} + q_m, (q_m + q^\ast)/\theta, \mathbf{v}, 0)^{\mathsf T}.
\end{gather*}

При $ \alpha = 1 $ --- уравнение неразрывности. 
\[
  \rho \frac{d(1/\rho)}{dt} = \nabla \cdot \mathbf{v}.
\]
Проверим. 
\begin{align*}
  - \frac{1}{\rho} \frac{d\rho}{dt} &= \nabla \mathbf{v},\\
  \frac{d\rho}{dt} + \rho\nabla \cdot \mathbf{v} &= 0 &\Rightarrow \frac{\partial
\rho}{\partial t} + \mathbf{v}\nabla\rho + \rho\nabla\mathbf{v} &= 0,\\
                                                                && \frac{\partial
                                                                \rho}{\partial
                                                                t}
                                                                +\nabla(\rho\mathbf{v})
                                                                &= 0.
\end{align*}



При $ \alpha = 2 $ --- уравнение движения. %TODO

При $ \alpha =3 $ --- уравнение энергии ($ \varepsilon = e + v^2/2$) 
\begin{equation}\label{eq:9}
  \rho = \frac{d \varepsilon}{dt} = \nabla \cdot (T \cdot \mathbf{v} -
  \mathbf{q}) + \rho \mathbf{f} \cdot \mathbf{v} + \rho q_m.
\end{equation}

При $ \alpha = 4 $ --- уравнение баланса энтропии. 
\begin{equation}\label{eq:10}
  \rho \frac{d\eta}{d t} = -\nabla (\mathbf{q}/\theta) + \rho \frac{q_m +
  q^\ast}{\theta}.
\end{equation}

При $ \alpha = 5 $ --- кинематическое соотношение (раньше не встречалось!) 
\[
  \rho \frac{d\mathbf{u}}{dt} = \rho \mathbf{v}.
\]
Действительно $ \frac{d\mathbf{u}}{dt} = \mathbf{v} $ --- следствие из
определения $ \mathbf{v} $ и $ \mathbf{u} $: 
\begin{align*}
  \mathbf{v} &= \frac{d \mathbf{x}}{dt} \big|_{x^i} = \frac{\partial
  \mathbf{x}}{\partial t} + \mathbf{v} \cdot \nabla \otimes \mathbf{x},\\
  %TODO %11
\end{align*}
\eqref{eq:10} $ \to $ \eqref{eq:9}:
\[
  \mathbf{v} = \frac{d}{dt}(\mathbf{u} + \mathring{\mathbf{x}}) =
  \frac{d\mathbf{u}}{dt}.
\]

При $ \alpha = 6 $ --- динамическое уравнение совместности деформации 
\[
  \rho \frac{dF^{\mathsf T}}{dt} = \nabla\cdot (\rho F^{\mathsf T}\otimes
  \mathbf{v}).
\]
Нигде не используется --- самая современная! Мало кто владеет полностью
математическим аппаратом, а мы (теперь) владеем!

\paragraph{2. Дивергентная форма законов сохранения в эйлеровом описании.}
\[\boxed{
  \frac{\partial \rho A_\alpha}{\partial t} + \nabla \cdot (\rho \mathbf{v}
\otimes A_\alpha - B_\alpha) = \rho C_\alpha, \quad (\alpha = 1, \ldots, 6).
}
\]
Заметим, что, как и выше, неизвестные константы могут быть и скалярами, и
векторами и т.\,п. 

Почему дивергентная? Потому что оператор дивергенции и потому что коэффициент
при ней не единичен (?).  
\[
  A_\alpha = (1, \mathbf{v}, \varepsilon, \eta, \mathbf{u}, F^{\mathsf T})
\]
Остальные коэффициенты те же, отличие только между $ A_1 $ и $ \bar{A}_1 $.

При $ \alpha = 1 $ имеем  
\[
  \frac{\partial \rho}{\partial t} + \nabla \cdot \rho \cdot \mathbf{v},
\]
где, например, $ A_1 \to \rho \mathbf{v} \otimes = \rho \mathbf{v} $ (такое соглашение).
Аналогично $ A_2 \to \rho \mathbf{v} \otimes \mathbf{v} $, \ldots

Остальное выписывать не будем.

Дивергентная форма хороша в численных методах, потому что можно перейти в
интегральную форму. А всё в численных методах решается интегральной формой,
никто по старинке не пользуется методом конечно-разностных сумм, не строит
прямоугольную сетку.

\paragraph{3. Интегральная форма в эйлеровом описании.}  
\[
  \frac{d}{dt} \int\limits_{V}^{} \rho A_\alpha\,d V =
  \int\limits_{\Sigma}^{}\mathbf{n} \cdot B_\alpha\,d\Sigma +
  \int\limits_{V}^{}\rho C_\alpha\,dV.
\]
(Второе слагаемое --- поток). 

Есть мнение, что интегральное исчисление более фундаментально, чем
дифференциальное. Действительно --- в том смысле, что имеет больше физических
приложений.

Например, при $ \alpha = 1 $ 
\[
    \frac{d}{dt}\int\limits_{V}^{}\rho\,dV = 0 \Rightarrow \frac{dm}{dt} = 0.
\]
В данном случае первый интеграл --- это масса.

\paragraph{4. ...}
%TODO:

При $ \alpha = 1 $  
\[
  \mathring{\rho} \frac{d}{dt} \left( \frac{\rho}{\mathring{\rho}}\det F \right)
  = 0 
\]
Но в $ \mathring{\mathcal K} $ $ \rho = \mathring{\rho} $, $ F = E $ %TODO
 
\[
    \boxed{
    \frac{\mathring{\rho}}{\rho} = \det F.}
\]


При $ \alpha = 6 $  
\[
  \mathring{\rho} \frac{dF^{\mathsf T}}{dt} = \mathring{\nabla}\cdot (\rho E
  \otimes \mathbf{v}).
\]
Так как $ \mathring{\rho} = \mathrm{const} $, то  
\begin{align*}
  \frac{dF^{\mathsf T}}{dt} &= \mathring{\nabla} \cdot (E \otimes \mathbf{v}) =
  \mathring{\nabla} \cdot \mathbf{v},\\
\end{align*}

%TODO

\paragraph{5.}
%TODO

 
\[
  \frac{d}{dt} \int\limits_{\mathring{V}}^{}\mathring{\rho}
  \mathring{A}_\alpha\,dV = \int\limits_{\mathring\Sigma}^{}\mathring n\cdot
  \mathring{B}_\alpha\,d\mathring{\Sigma} +
  \int\limits_{\mathring{V}}^{}\mathring{\rho} C\,d\mathring V.
\]
Фундаментальный закон мира: любое изменение разделяется на изменение от
внутренних источников и от внешних. Так и в квантовой физике, и в экономике, и
вообще везде.

\dimus{Современная экономика, я уж себе позволю, совсем ещё не развита. На уровне законов Ньютона
--- на 300 лет отстаёт от остальных наук. В МСС присутствуют все законы
экономики, если хотите заниматься серьёзной экономикой, занимайтесь МСС.}

При $ \alpha = 1 $  
\[
  \frac{d}{dt} \int\limits_{\mathring{V}}^{}\mathring{\rho}\,d\mathring{V},
\]
%TODO: 


Получили 18 неизвестных и 29 уравнений --- незамкнутая система уравнений в
частных производных. Плохо.

Незамкнутость системы уравнений --- закономерный результат. Потому что мы
сформулировали только общие уравнения (для твёрдых сред, жидких, газообразных и т.\,п.).
Нигде не проявилась специфика. Да и твёрдые среды бывают разные, жидкие тоже.
