
% TODO пропущено поллекции

\subsection{Закон сохранения энергии (первый закон термодинамики)}

\begin{definition}
  Введём следующие величины:
  \begin{enumerate}
    \item Кинетической энергией называется величина:
      \[
        K = \int\limits_V \dfrac{v^2}{2} \, dm = \int\limits_V \dfrac{\rho v^2}{2} \, dV;
      \]
    \item Мощностями массовых и поверхностных сил называются величины:
      \[
        W_m = \int\limits_V \mathbf{f} \cdot \mathbf{v} \, dm
        = \int\limits_V \rho \mathbf{f} \cdot \mathbf{v} \, dm; \quad
        W_\Sigma = \int\limits_\Sigma \mathbf{t}_n \cdot \mathbf{v} \, d\Sigma
      \]
  \end{enumerate}
\end{definition}

\paragraph{Аксиома 7. Закон сохранения энергии (первый закон термодинамики)}[формулировка К.Трусдела]
Всякая сплошная среда в актуальной конфигурации $K$ в любой момент времени $t \geqslant 0$
обладает двумя скалярными аддитивными функциями:
$U(K, t)$ -- внутренняя энергия сплошной среды, $Q(K, t)$ -- скорость нагрева сплошной среды;
такими, что
\[
  \dfrac{dE}{dt} = W + Q,
\]
где $E = U + K$ -- полная энергия;
$W = W_m + W_\Sigma$ -- мощность внешних сил.
