\section{Основополагающие аксиомы МСС}
% Лекция 3 -- 2024-02-21

\subsection{Определение сплошной среды}

\begin{figure}[H]
	\centering
	\begin{minipage}{0.3\linewidth}
		\centering
		\includesvg[scale=0.8]{lec03_mss_in_a_nutshell}
	\end{minipage} 
	\hspace{25mm}
	\begin{minipage}{0.3\linewidth}
		\centering
		\includesvg[scale=1]{lec03_continuum}
		\caption{В любом бесконечно малом объеме всегда много вещества}
	\end{minipage}	
\end{figure}

%\begin{figure}[H]
%	\centering
%	\includesvg[scale=0.8]{mss_in_a_nutshell}
%\end{figure}

МСС изучает <<не очень маленькие>> и <<не очень большие>> объекты --- тела, которые состоят
из материальных точек.

%\begin{figure}[h!]
%  \centering
%  \includesvg[scale=1]{continuum}
%  \caption{В любом бесконечно малом объеме всегда много вещества}
%\end{figure}

\begin{figure}[H]
	\centering
	\includesvg[scale=1]{lec03_twoplanes}
\end{figure}

\begin{definition}
  \emph{Сплошной средой} называется тело $B$, для которого введено взаимно однозначное соотвествие
  или отображение в некоторое метрическое простанство $\chi$.
\end{definition}

\begin{definition}
  Множество всех тел $B$ называется \emph{вселенной}.
\end{definition}

%\begin{figure}[H]
%	\centering
%	\includesvg[scale=1]{twoplanes}
%\end{figure}

\paragraph{Аксиома 1 (аксиома сплошности).} Образ $W(B)$ всякой сплошной среды образует континуальное
множество в $\chi$. Открытое континуальное множество суть область $V$ в $\chi$.

\begin{wrapfigure}{r}{0.5\textwidth}
	\centering
	\includesvg[scale=1]{lec03_continuumaxiom}
\end{wrapfigure}

\[
  \forall M \in W(B)\colon \exists \mathring{U}_\delta (M) \subset W(B),
\]
где $\mathring{U}_\delta$ --- проколотая окрестность.
\vspace{45pt}

\paragraph{Аксиома 2 (евклидовость пространства).} 
\begin{wrapfigure}{l}{0.5\textwidth}
	\centering
	\includesvg[scale=1]{lec03_euclidspace}
\end{wrapfigure}
В качестве метрического пространства $\chi$,
в котором рассматривается все тела $B$ вселенной можно выбрать трёхмерное 
точечно-евклидово пространство $\mathcal{E}^a_3 = \chi$. В $\mathcal{E}_3^a$
существует много различных <<конструкций>>: точки (материальные точки
$M \in W(B)$), векторы ($M_1, M_2 \in \mathcal{E}_3$), единая декартова система координат
($O\bar{\mathbf{e}}_i\colon \forall M \in W(B) \ \exists! \mathbf{OM} = \mathbf
x = x^i \bar{\mathbf{e}}_i$).

Случай, когда нет единой декартовой СК, встретился в примере \ref{cilinder_surface}.

\paragraph{Аксиома 3 (существование абсолютного времени).} 
Для всякого тела $V$ и для всякого
параметра $0 \leqslant t \in \mathbb{R}$ определено положение тела $V(t)
\subset \mathcal{E}_3^a$ в момент $ t $.

Есть самолёт, зависящий от какого-то параметра и при каких-то разных значениях этого параметра 
образы этого самолёта будут разными. И в разные моменты времени одна и та же материальная точка
будет иметь разные радиусы-векторы.

\begin{wrapfigure}{r}{0.5\textwidth}
	\centering
	\includesvg[scale=1.1]{lec03_planes}
\end{wrapfigure}

Абсолютизм времени состоит в том, что при движении с большими скоростями возникают релятивистские 
эффекты, но в классической механике сплошных сред $|\mathbf{v}| \ll c$, чтобы время текло одинаково
для всех тел.


\begin{definition}
  \emph{Движением} тел и материальных точек называется изменение радиусов-векторов материальных точек в
  единой декартовой системе координат $O \bar{\mathbf{e}}_i$.
\end{definition}

%\begin{figure}[H]
%	\centering
%	\includesvg[scale=1.1]{planes}
%\end{figure}


\subsection{Кинематика сплошных сред}
\begin{definition}
  Кинематика сплошных сред -- это раздел, в котором изучается изучается движение тел без анализа
  причин, которые вызывают это движение.
\end{definition}

\paragraph{Лагранжевы и эйлеровы координаты.}
\begin{figure}[H]
	\centering
	\includesvg[scale=1.1]{lec03_lagrang_and_eulier}
\end{figure}

Рассмотрим произвольную сплошную среду в некоторый момент времени $t_1 \geqslant 0$ и в некоторый
другой момент времени $t_2 \geqslant 0$.

Эта сплошная среда $B$ состоит из одних и тех же материальных точек $M$ во всех рассматриваемых 
моментах времени\footnote{Имеется в виду, что мы рассматриваем такой класс тел, то есть существуют
и другие.}.
  
Благодаря евклидовости пространства, всегда существует единая декартова система координат. В
ней мы можем следить за радиус-вектором какой-либо материальной точки.

Рассмотрим положение тела в $O\bar{\mathbf{e}}_i$ в некоторый фиксированный момент времени, 
в котором удобно выбрать $t=0$.

\begin{figure}[H]
	\centering
	\includesvg[scale=1.1]{lec03_lagrang_and_eulier2}
\end{figure}

Положение тела $\mathring{V}$ в $O\bar{\mathbf{e}}_i$ при $t=0$ называется
\emph{отсчётной}
(\emph{лагранжевой}) конфигурацией $\mathring{\mathcal{K}}$.

Положение того же самого тела $V$ в момент времени $t > 0$ называется актуальной (эйлеровой)
конфигурацией $\mathcal{K}$.

Укажем далее способ <<паспортизации>> всех материальных точек. В момент времени $t = 0 :
\mathbf{x}(M) = \mathring{\mathbf{x}} = \mathring{x}^{i} \bar{\mathbf{e}}_i$, т.е. $\mathring{x}^{i}$
-- декартовы коодинаты материальной
точки $M$ при $t=0$. Введём криволинейные координаты матириальной точки при $t=0$:
$X^i = X^i(\mathring{x}^j) \Leftrightarrow \mathring{V} \to \mathring{V}_x$. Из аксиомы 3 следует, что для той же 
материальной точки $M$ сплошной среды $B$ в моменты времени $t > 0$ соотвествует $\mathbf{x}(t) = x^i \bar{\mathbf{e}}_i$.

Введём такие криволинейные координаты $X^i$ материальных точек, которые введены в
$\mathring{\mathcal{K}}$, а дальше \textbf{движутся вместе с материальными точками}. В частном
случае эти криволинейные координаты могут совпадать с обычными координатами $X^i = \mathring{x}^i$.

\begin{figure}[H]
	\centering
	\includesvg[scale=0.8]{lec03_balka}
\end{figure}


Координаты движутся вместе с материальными точками означает, что для любого момента времени $t$ 
все материальные точки $M$ тела имеют одни и те же координаты $X^i$. Такие координаты $X^i$ называют
\emph{лагранжевыми координатами} материальной точки $M$.

Введённые лагранжевы координаты $X^i$ -- это способ <<паспортизации>> материальных точек (то есть
способ их различения).

Полагая, что $X^i(x^i)$ -- это гладкие функции, определённые в области $\mathring{V}$ и невырождены:
\[
  \forall \mathring{x} \in \mathring V : \det \left( \frac{\partial X^i}{\partial x^j}  \right) \neq 0.
\]

Тогда из теоремы об обратной функции из курса ТФНП существует обратная функция $\mathring{x}^i
= \mathring{x}^i (X^j), \forall X^j \in V_x$. Причём
\[
  \det \left( \frac{\partial \mathring{x}^i}{\partial X^{j}}  \right)
  = \frac{1}{\det \left( \frac{\partial X^i}{\partial \mathring{x}^i}  \right) } \neq 0.
\]
Тогда $\mathring{\mathbf{x}} = \mathring{x} \bar{\mathbf{e}}_i = \mathring{x}^i (X^j) \bar{\mathbf{e}}_i$.
($\bar{\mathbf{e}}_i$ -- не зависит от системы координат). Тогда получается, что
$\mathring{\mathbf{x}} = \mathring{\mathbf{x}} ( X^i) = \mathring{x}^i \bar{\mathbf{e}}_i$.

\begin{wrapfigure}{r}{0.5\textwidth}
	\centering
	\includesvg[scale=0.8]{lec03_hedg}
\end{wrapfigure}

Рассмотрим теперь момент времени $t = 0$: $\mathbf{x} = \mathbf{x}(X^i, t)$ -- из введения лагранжевых 
координат и способа паспортизации материальных точек. Если зафиксировать $t$ (но берём разные
криволинейные координаты), то получим множество радиус-векторов:

%\begin{figure}[H]
%	\centering
%	\includesvg[scale=0.7]{hedg}
%\end{figure}

Если зафиксируем криволинейные координаты, а $t$ будем менять, то получим параметрическое
представление кривой $\mathbf{x} = \mathbf{x} ( X^i, t )$ -- траектория материальной точки (с координатами
$X^i$). Причём траектории не пересекаются в один и тот же момент времени. Уравнение $\mathbf{x} = \mathbf{x}(X^i, t)$ называется законом движения материальных точек.

Если бы был известен закон движения всех материальных точек, то задача кинематики была бы решена.
$x^{i}$, $\mathring{x}^i$ -- эйлеровы координаты материальной точки $M$. Из уравнений каких-то
следует, что $\mathbf{x} = \mathbf{x} (X^i, t) = x^j ( X^i, t) \bar{\mathbf{e}}_j$. $x^j = x^j(X^j, t)$
-- связь эйлеровых координат с лагранжевыми.

\subsection{Материальное и пространственное описание движения тела}

В МСС состояние тел описывается неокоторыми:
\begin{itemize}
  \item скалярными величинами: $\theta, \rho, \dots$;
  \item векторными величинами: $\mathbf{x}, \mathbf{v}, \mathbf{a}, \dots$;
  \item тензорными величинами: $E, F, T, \dots$.
\end{itemize}

Эти тензоры являются функциями от $X^i$ и $t$:
\begin{itemize}
  \item $\mathbf{x} = \mathbf{x} (X^i, t)$;
  \item $\mathbf{b} = \mathbf{b} (X^i, t) \forall \mathbf{b} \in \mathcal{E}_3$;
  \item $\xi = \xi(X^i, t) \forall $ скаляра;
  \item $A = A(X^i, t) \forall$ тензора.
\end{itemize}
Везде $t \in \mathbb{R}_{+0}$. Эти все штуки называются \emph{полями} скаляров, тензоров, векторов.

Существуют обратные $X^i = X^i ( x^j, t )$, если они гладкие и невырожденные: $\det \frac{\partial X^i}{\partial x^j} \neq 0 \forall x^i \in V \times [0, +\infty)$.

Тогда любое поле: $\xi = \xi(X^i, t) = \xi(X^i(x^j, t), t) = \tilde{\xi} (x^j, t)$ (аналогично и с
векторами и с тензорами). В силу взаимной однозначности функции соответствия между эйлеровыми и 
декартовыми координатами, всегда можно использовать запись полей, которые описывают состояние
тела, либо через $(X^i, t)$ (если используется такое описание, то говорят, что используется
материальное (лагранжево) описание), либо через $(x^i, t)$ (если используется это описание, то 
говорят, что применяется пространственное (эйлерово) описание).

Рассмотрим некоторые примеры, когда удобней использовать какое-либо описание.

\begin{example} 
  Для твёрдых тел чаще используется лагранжево описание движения тела.
  
  \begin{figure}[H]
  	\centering
  	\includesvg[scale=0.8]{lec03_drummer}
  \end{figure}
  
  При использовании лагранжевого описания, мы точно знаем область, в которой нам нужны тензоры
  всякие, потому что область совпадает с изначальной.

  При использовании эйлерового описания, область неизвестна и основная задача будет в её
  нахождении.
\end{example}

\begin{example}
  Газовая динамика.
  
  \begin{figure}[H]
  	\centering
  	\includesvg[scale=0.8]{lec03_engine}
  \end{figure}
  
  Область, в которой ищутся характеристики движения, постоянна, потому что нас не интересуют
  области, в которых жидкость была до данного момента, и в которые она попадёт после интересующего 
  момента, поэтому рассматриваем

  Нам не нужно находить радиус-вектора точек (хотя мы бы могли), нам нужно находить скалярные поля
  в интересующей области -- плотность, давление и т.д. Уравнения газовой динамики позволяют разделить
  задачи нахождения радиус-векторов и этих полей (в твёрдых телах такого не мы сделать не можем).
\end{example}

\subsection{Локальные базисы в $\mathring{\mathcal{K}}$ и $\mathcal{K}$}

Рассмотрим уравнение движения материальных точек $\mathbf{x} = \mathbf{x} (X^i, t)$ -- оно всегда
сущестует, пусть даже мы его не находим. Предполагаем, что эти функции гладкие, продифференцируем
их по $X^j$:
\[
  \mathbf{r}_i = \frac{\partial \mathbf{x}}{\partial X^i}; \quad
  \mathring{\mathbf{r}}_i = \frac{\partial \mathring{\mathbf{x}}}{\partial X^i},
\]
локальные базисы в $\mathcal{K}$ и $\mathring{\mathcal{K}}$ (то, что это базис, следует из
невырожденности).

\begin{wrapfigure}{r}{0.5\textwidth}
	\centering
	\includesvg[scale=1.1]{lec03_localbasis}
\end{wrapfigure}

$\mathbf{r}_i$ и $\mathring{\mathbf{r}}_i$ направлены по касательным к Лагранжевым координатам $X^i$. 
Локальные базисы $\mathbf{r}_i$ движутся вместе с материальными точками. 
\[
  \mathbf{r}_i
  = \frac{\partial \mathbf{x}}{\partial X^i}
  = \frac{\partial x^j \bar{\mathbf{e}}_j}{\partial X^i}  
  = \tensor{Q}{^j_i} \bar{\mathbf{e}}_j,
\]
где $\tensor{Q}{^j_i}$ -- якобиевая матрица.

\[
  \mathring{\mathbf{r}}_i = \frac{\partial \mathring{\mathbf{x}}}{\partial X^i}
  = \frac{\partial x^{0j}}{\partial X^i} \bar{\mathbf{e}}_j
  = \tensor{\mathring{Q}}{^j_i} \bar{\mathbf{e}}_j
\]

Так как предполагаются невырожденными, то существуют обратные матрицы:
\[
  \tensor{P}{^j_i} \tensor{Q}{^i_k} = \delta^j_{\, k}; \quad
  \tensor{\mathring{P}}{^j_i} \tensor{\mathring{Q}}{^i_k} = \delta^j_{\, k}
\]

$P$ -- обратная якобиевая матрица.

\[
  g_{ij} = \mathbf{r}_i \cdot \mathbf{r}_j; 
  \quad
  \mathring{g}_{ij} = \mathring{\mathbf{r}}_i \cdot \mathring{\mathbf{r}}_j;
\]
-- метрические матрицы отсчетного и произвольного состояния (причём $\det g_{ij} \neq 0$, что
доказывается с помощью подстановки вместо $\mathbf{r}_i = \tensor{Q}{^j_i} \bar{\mathbf{e}}_j$, попутно
получаем еще одну формулу для метрической матрицы: 
$g_{ij} = \tensor{Q}{^k_i} \tensor{Q}{^l_j} \delta_{kl}$, $\det g_{ij} = (\det Q )^2$).

Обозначим также $\mathring{g} = \det \mathring{g}_{ij} \neq 0$ и $g = \det g_{ij} \neq 0$.
Тогда введём $\mathbf{r}^i  = g^{ij} \mathbf{r}_j, \mathring{\mathbf{r}}^i = \mathring{g}^{ij}
\mathring{\mathbf{r}}_j$ --
векторы взаимных базисов в $\mathcal{K}$ и $\mathring{\mathcal{K}}$.

Их свойства:
\[
  \mathbf{r}_i \cdot \mathbf{r}^j = \delta^j_i, \quad \mathring{\mathbf{r}}_i \cdot \mathring{\mathbf{r}}^j = \delta^j_i
\]
\begin{proof}
  \[
    \mathbf{r}_i \cdot (g^{jk} \mathbf{r}_k)
    = g^{jk} \mathbf{r}_i \cdot \mathbf{r}_k
    = g^{jk} \cdot g_{ik} = \delta_i^j.
  \]
\end{proof}

Следствие: $\mathbf{r}_i \times \mathbf{r}_j = \sqrt{g} \varepsilon_{ijk} \mathbf{r}^k$.

\begin{proof}
  $\mathbf{a} \times \mathbf{b} = \sqrt{g} \varepsilon_{ijk} a^i b^j \mathbf{r}^k$.
  Из $\mathbf{a} = a^i \mathbf{r}_i$ и $\mathbf{b} = b^j \mathbf{r}_i$, поэтому 
  \[
    \mathbf{r}_i \times \mathbf{r}_j
    % TODO дописать доказательство
  \]
\end{proof}

Аналогично всё с ноликовыми векторами.

Следствие еще одно:
\[
  \mathbf{r}_1 \cdot (\mathbf{r}_2 \times \mathbf{r}_3)
  = \mathbf{r}_1 \cdot \sqrt{g} \mathbf{r}^1
  = \sqrt{g} \cdot 1 = \sqrt{g}.
\]

Таким образом, $\sqrt{g} = \mathbf{r}_1 \cdot (\mathbf{r}_2 \times \mathbf{r}_3)$.


\subsection{Ковариантная производная в $\mathcal{K}$ и $\mathring{\mathcal{K}}$}

Введем набла оператор в $\mathcal{K}$ -- символический дифференциальный оператор:
\[
  \nabla \equiv \mathbf{r}^i \frac{\partial }{\partial X^i} 
  = \mathbf{r}^1 \frac{\partial }{\partial X^1} + \mathbf{r}^2 \frac{\partial }{\partial X^2} 
  + \mathbf{r}^3 \frac{\partial }{\partial X^3};
\]

Аналогично с наблей в $\mathring{\mathcal{K}}$.

Применение его:
\begin{enumerate}
  \item К скаляру: градиент скаляра -- вектор
    \[
      \nabla \varphi = \mathbf{r}^i \frac{\partial \varphi}{\partial X^i}.
    \]
    Из свойств отметим, что он инвариантен, то есть 
    \[
      \nabla \varphi = \bar{\mathbf{e}}^i \frac{\partial \varphi}{\partial x^i} 
      = \frac{\partial \varphi}{\partial x^1} \bar{\mathbf{e}}_1 + \frac{\partial \varphi}{\partial x^2} \bar{\mathbf{e}}_2 + \frac{\partial \varphi}{\partial x^3} \bar{\mathbf{e}}_3.
    \]

  \item К вектору:
    \begin{itemize}
      \item тензорно: градиент вектора -- тензор второго ранга:
        \[
          \nabla \otimes \mathbf{a}
          = \mathbf{r}^i \frac{\partial }{\partial X^i} \otimes \mathbf{a}
          = \mathbf{r}^i \otimes \frac{\partial \mathbf{a}}{\partial X^i} 
        \]
        Рассмотрим
        \[
          \frac{\partial \mathbf{a}}{\partial X^i} 
          = \frac{\partial a^j \mathbf{r}_j}{\partial X^i} 
          = \frac{\partial a^j}{\partial X^i} \mathbf{r}_j
            + a^j \frac{\partial \mathbf{r}_j}{\partial X^i} 
            = \left( \frac{\partial a^k}{\partial X^i} + a^j \Gamma^k_{ji} \right) \mathbf{r}_k,
        \]
        где введено обозначение:
        $ \frac{\partial \mathbf{r}_j}{\partial X^i} 
          = \Gamma^k_{ji} \mathbf{r}_k $
        -- символы Кристоффеля.

        Обозначим: $\nabla_i a^k \equiv \frac{\partial a^k}{\partial X^i} + a^j \Gamma^k_{ji}$
        -- ковариантная производная от контравариантных компонент вектора.
        Тогда $ \frac{\partial \mathbf{a}}{\partial X^i} = (\nabla_i a^k) \mathbf{r}_k$.

        Если $\mathbf{r}_i \equiv \bar{\mathbf{e}}_i$, то $\Gamma^k_{ji} \equiv 0$. То есть
        ковариантная производная совпадёт с частной производной.

        Подставим полученное в начало:
        \[
          \nabla \otimes \mathbf{a} = \mathbf{r}^i \otimes (\nabla_i a^k) \mathbf{r}_k 
          = (\nabla_i a^k) \mathbf{r}^i \otimes \mathbf{r}_k
        \]
        Компоненты этого тензора в смешанном диадном локальном базисе.

        \textbf{Теорема Риччи}: $\nabla_i g_{jk} \equiv 0$, из нее следует, что можно опускать и 
        поднимать индексы под знаком контравариантной производной.

        \[
          \nabla^i a_k \equiv g^{ij} \nabla_j a_k
        \]
        -- контравариантая производная от ковариантных компонент вектора.

\[
          \nabla \otimes \mathbf{a}
          = (\nabla_i a^k) \mathbf{r}^i \otimes \mathbf{r}_k
          = (\nabla^i a^k) \mathbf{r}_i \otimes \mathbf{r}_k
        \]

        Отметим, что $ \frac{\partial a_k}{\partial X^i} $ -- не являются компонентами какого-то
        тензора, но $\nabla_i a_k$, $\nabla^i a^k$ -- являются компонентами тензора 2-го ранга.

    \end{itemize}
\end{enumerate}


