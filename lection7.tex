\begin{remark}
  \begin{enumerate}

    \item В этой аксиоме введено понятие силы $\mathcal{\mathbf{F}}$, которое не может 
      быть определено через какие-либо введенные ранее понятия.

    \item <<О первом законе Ньютона>>. Инерциальные системы отсчета. В 
      аксиоме 2 о евклидовости пространства $\mathbb{E}_3^a$ предполагается, что
      всегда можно ввести единую (единую для всех тел и для любого момента времени)
      декартовую ортонормированную систему координат. Получается, что
      первая часть закона Ньютона содержится в аксиоме евклидовости.
      Вторая часть первого закона Ньютона содержиться в аксиоме 5.

      Фактически, второй закон Ньютона -- частный случай \eqref{axiom-5}.
      Абсолютно твёрдое тело -- такая сплошная среда, для которой рассмотренный
      закон для любой точки в любой момент времени ???
      Если рассмотреть абсолютно твёрдое тело, то 
      \[
        \mathbf{v} = \mathbf{v}(t) \forall X^j \in V_X
      \]
      -- одинаковая скорость любой точки тела (не рассматривается вращение).
      Тогда $\mathbf{I} = M \cdot \mathbf{v}$, а, следовательно, \eqref{axiom-5}
      принимает вид:
      \[
        \dfrac{d\mathbf{I}}{dt} = M \dfrac{d\mathbf{v}}{dt} = \mathbf{\mathcal{F}}.
      \]
      Тогда обозначая ускорение $\mathbf{a} = \dfrac{d\mathbf{v}}{dt}$, 
      получаем классический второй закон Ньютона
      $M\mathbf{a} = \mathbf{\mathcal{F}}$.

    \item <<О неинерциальных системах отсчёта>>. Существуют разные системы 
      отсчёта, в том числе подвижные. В таких системах отсчёта законы меняются.
      Отсюда следует, что уравнение \eqref{axiom-5} -- необъективно, то есть зависит
      от системы отсчета (закон сохранения массы, к примеру, не зависел от
      системы отсчёта). Причём указанная аксиома никак не говорит о том, какой
      именно базис является инерциальным. Оказывается, что данная проблема 
      не решаема. В каждой задаче необязательно учитывать все эти неинерциальности,
      например, в случае расчёта машины не обязательно учитывать вращение Земли.
      % TODO рисунок базисы чото туда сюда
      % TODO рисунок машинки
  \end{enumerate}
\end{remark}

\subsection{Силы в МСС}

Разделим силы на внешние и внутренние. Про внешние силы вроде понятно, но что
такое внутренние? Рассмотрим некоторое тело. Разделим его
мысленно на части, каждая из частей действует на другую с некоторой силой.
Эти силы и назовём внутренними.

Также силы деляться на массовые и поверхностные.

Перечислим основные силы:
\begin{itemize}
  \item Массовые силы:
    Внешние:
    \begin{itemize}
      \item гравитационные силы;
      \item силы 
      \item электромагнитные силы;
    \end{itemize}
    Внутренние:
    \begin{itemize}
      \item силы инерции;
    \end{itemize}

  \item Поверхностные силы:
    \begin{itemize}
      \item силы трения -- внешние;
      \item <<силы упругости>> -- внутренние.
    \end{itemize}
\end{itemize}

Рассмотрим отдельно массовые и поверхностные силы.

\paragraph{Массовые силы (внешние)}
Разобьём тело на континуальное множество областей.
Для тела в актуальной конфигурации и точки $M$ с некоторой окрестностью $dV$ массы
$dm$. Согласно аксиоме 5, существует сила взаимодействия $dV$ с \emph{внешностью}, 
по отношению ко всему телу, обозначим эту силу $\mathbf{dF}_m$ -- внешняя массовая сила.

Введём новое понятие $\mathbf{f} = \dfrac{\mathbf{dF}_m}{dm}, dm > 0$ -- 
\emph{вектор плотности массовых сил}.

Для внешних массовых сил суммарный вектор внешних массовых сил:
\[
  \mathbf{\mathcal{F}}_m = \int\limits_V d\mathbf{\mathcal{F}}_m
  = \int\limits_V \rho \mathbf{f} \, dV.
\]

\paragraph{Поверхностные силы (внешние)}
Поверхность тела в актуальной конфигурации обозначим $\Sigma$. Для точки $M \in \Sigma$
выберем некоторую площадку $d\Sigma$. На часть объёма $d\tilde{V} = dV \bigcap V$
действует некоторая сила $d\mathbf{F}_\Sigma$ -- \emph{внешняя поверхностная сила}.
Для неё введём вектор $\mathbf{s} = \dfrac{d\mathbf{F}_\Sigma}{d\Sigma}$
-- \emph{вектор плотности поверхностных сил}.

Тогда в силу аддитивности сил можно сказать, что вектор суммарных внешних
поверхностных сил, действующих на поверхность тела $V$: (так как
$\Sigma = \int_\Sigma d\Sigma$)
\[
  \mathbf{\mathcal{F}} = \int\limits_\Sigma d\mathbf{\mathcal{F}}_\Sigma 
  = \int\limits_\Sigma \mathbf{s}' d\Sigma.
\]

Суммарная внешняя сила, действующая на тело $V$ со стороны окружающих его тел:
\[
  \mathbf{\mathcal{F}} = \mathbf{\mathcal{F}}_m + \mathbf{\mathcal{F}}_\Sigma.
\]


\subsection{Интегральная формулировка закона изменения количества движения}
Подставим полученное в \eqref{axiom-5}:
\begin{equation}\label{integral-form-I}
  \dfrac{d}{dt} \int\limits_V \rho \mathbf{v} dV =
  \int\limits_V \rho \mathbf{f} \, dV + \int\limits_\Sigma \mathbf{s} \, d\Sigma.
\end{equation}

Красота заключается в том, что законы сохранения имеют одинаковый вид, для
сравнения:
\[
  \dfrac{d}{dt} \int\limits_V \rho \, dV = 0.
\]

Используем правило дифференцирования интеграла по подвижному объёму. Для этого 
приведём без доказательства следующую формулу:

Упражение: доказать:
\[
  \dfrac{d}{dt} \int\limits_{V(t)} \rho \mathbf{a} \, dV =
  \int\limits_{V(t)} \dfrac{d\mathbf{a}}{dt} \, dV
\]

Тогда:
\begin{align*}
  &\int\limits_V \rho \dfrac{d\mathbf{v}}{dt} \, dV = \int\limits_V \rho \mathbf{f} \, dV
  + \int\limits_\Sigma \mathbf{s} \, d\Sigma, \\
  &\int\limits_V \rho \left( \mathbf{f} - \dfrac{d\mathbf{v}}{dt} \right) \, dV
  + \int\limits_\Sigma \mathbf{s} \, d\Sigma = 0.
\end{align*}
-- \emph{уравнение равновесия всех сил, действующих на тело}, где 
$\mathbf{\mathcal{F}}^{(i)}_{m} = - \int\limits_V \rho \dfrac{d\mathbf{v}}{dt} \, dV$
-- внутренняя массовая сила.



\subsection{Внутренние поверхностные силы}

Рассмотрим сплошную среду $V$. Разделим её на две части $V_1, V_2$. Выберем
произвольную точку $M \in \Sigma_0 \bigcap V$. Введем области $V_1$ и $V_2$
и вектор нормали $\mathbf{n}$, внешний по отношению к $V_2$. $\mathbf{n}$
отнесём в точку $M$ с площадкой $d\Sigma$.
% TODO рисунок
Тогда область $V_1$ действует на $V_2$ с силой $d\mathbf{\mathcal{F}}_2$, и
наоборот $V_2$ на $V_1$ с силой $d\mathbf{\mathcal{F}}_1$. Поскольку появление
этих сил связано с разбиением тел некоторой поверхностью, то эти силы будем
считать поверхностными, а не массовыми. Тогда можно ввести плостности этих
поверхностных сил $\mathbf{s}$: $\mathbf{s}_1 = \dfrac{d\mathbf{\mathcal{F}}_1}{d\Sigma}, \mathbf{s}_2 = \dfrac{d\mathbf{\mathcal{F}_2}}{d\Sigma}$.

Рассмотрим $\mathbf{t}_n = \dfrac{d\mathbf{\mathcal{F}}_1}{d\Sigma}$ и
$\mathbf{t}_{-n} = \dfrac{d\mathbf{\mathcal{F}}}{d\Sigma}$.
Их можно трактовать так: через данную точку можно провести некоторую поверхность
разделения тела, и вдоль элементарной площадки вокруг точки $M$ будут появляться
силы, а значит и $\mathbf{t}_n$ и $\mathbf{t}_{-n}$ -- \emph{векторы напряжений}.


\subsection{Первая теорема Коши}

\begin{theorem}[Коши, 1]
  Если $\mathbf{t}_n$ определена для поверхности $\Sigma$, не являющейся 
  поверхностью разрыва, то:
  \[
    \mathbf{t}_n = - \mathbf{t}_{-n}.
  \]

  Даже если поверхность $\Sigma$ сама гладкая, то это не значит, что решение
  будет само непрерывным (поверхности со скачками называем поверхностями
  разрыва).
\end{theorem}
\begin{proof}
  Рассмотрим область $V$ и разобьём её на две части $V_1$ и $V_2$ с помощью 
  поверхности $\Sigma$. И применим к $V_1$ и $V_2$ закон об изменении 
  количества движения в интегральной форме \eqref{integral-form-I}:
  \[
    \begin{cases}
      \int\limits_{V_1} {\rho \left(\mathbf{f} - \dfrac{d\mathbf{v}}{dt}\right) \, dV}
      + \int\limits_{\Sigma_1} \mathbf{s} \, d\Sigma
      + \int\limits_{\Sigma_0} \mathbf{t}_n \, d\Sigma = 0, \\

      \int\limits_{V_2} \rho \left(\mathbf{f} - \dfrac{d\mathbf{v}}{dt}\right) \, dV
      + \int\limits_{\Sigma_2} \mathbf{s} \, d\Sigma
      + \int\limits_{\Sigma_0} \mathbf{t}_{-n} \, d\Sigma = 0, \\

      \int\limits_{V_1 \bigcup V_2} \rho \left(\mathbf{f} - \dfrac{d\mathbf{v}}{dt}\right) \, dV
      + \int\limits_{\Sigma_1 \bigcup \Sigma_2} \mathbf{s} \, d\Sigma = 0
    \end{cases}
    \Rightarrow
    \int\limits_{\Sigma_0} (\mathbf{t}_n + \mathbf{t}_{-n}) d\Sigma &= 0.
  \]

  Поскольку $\Sigma_0$ -- произвольная, то её можно заменить на $d\Sigma_0$, 
  или по-другому, можно сказать, что в силу произвольности, 
  равенство нулю интеграла означает равенство нулю подинтегральной функции:
  \[
    \mathbf{t}_n + \mathbf{t}_{-n} = 0.
  \]

  Неявно использовали условие о гладкости $\mathbf{t}_n$.
\end{proof}


\subsection{Вторая теорема Коши}

Рассмотрим элементарный объём $dV$ в виде тетраэдра, который построен на
элементарных векторах $d\mathbf{x}_\alpha = \mathbf{r}_\alpha \cdot dX^\alpha$
(суммирования нет). У тетраэдра 4 грани, 3 из них назовём \emph{координатными}
-- $d\Sigma_\alpha, \alpha=1, 2, 3$, а оставшуюся \emph{наклонной} $d\Sigma_0$.
Определим вектора нормалей на этих гранях
$\mathbf{n}_\alpha = - \dfrac{\mathbf{r}^\alpha}{|\mathbf{r}^\alpha|}$ 
(т.к. $\mathbf{r}_\alpha \cdot \mathbf{r}^\beta = \delta^\beta_\alpha
\Rightarrow \mathbf{r}_\alpha \times \mathbf{r}_\beta = \sqrt{g} r^\gamma$).
А к $d\Sigma_0$ -- $\mathbf{n}$.

Рассмотрим следующее соотношение для 
любой поверхности $\Sigma$, ограничивающей замкнутую область:
\[
  \int\limits_\Sigma \mathbf{n} d\Sigma
  = \int\limits_\Sigma \mathbf{n} \cdot E \, d\Sigma
  = \int\limits_V \nabla \cdot E \, dV
  = 0,
\]
(использована формула Гаусса-Остроградского, а значит предполагается гладкость 
поверхности $\Sigma$).

Применим эту формулу к тетраэдру:
\[
  \int\limits_V \mathbf{n} \, d\Sigma
  = \sum_{\alpha=1}^3 \mathbf{n}_\alpha d\Sigma_\alpha + \mathbf{n} d\Sigma_0
  = 0.
\]

Подставим выражения для нормалей:
\begin{align*}
  \mathbf{n} d\Sigma_0
  &= \sum_{\alpha=1}^3 \dfrac{\mathbf{r}^\alpha}{|\mathbf{r}^\alpha|} d\Sigma_\alpha \\
  \mathbf{n} \cdot \mathbf{r}_\beta
  &= \sum_{\alpha=1}^3 \dfrac{\mathbf{r}^\alpha\cdot\mathbf{r}_\beta}{|\mathbf{r}^\alpha|} d\Sigma_\alpha
  = \dfrac{d\Sigma_\beta}{|\mathbf{r}^\beta|} \\
  d\Sigma_\beta = |\mathbf{r}^\beta| \mathbf{n} \cdot \mathbf{r} d\Sigma_0.
\end{align*}

Применим к тетраэдру закон об изменении количества движения в форме
\eqref{integral-form-I}:
\[
  \int\limits_\Sigma \mathbf{t}_n \, d\Sigma
  = \int\limits_V \rho \left( \dfrac{d\mathbf{v}}{dt} - \mathbf{f} \right) \, dV
\]
Введём обозначение: на каждую площадку $d\Sigma_\alpha$ и $d\Sigma_0$
действуют силы $d\mathbf{\mathcal{F}}_\alpha$ и $d\mathbf{\mathcal{F}}_0$.
Тогда на наклонной площадке с нормалью $\mathbf{n}$ можно ввести 
$\mathbf{t}_n = \dfrac{d\mathbf{\mathcal{F}}_0}{d\Sigma_0}$. Для остальных
площадок $\mathbf{t}_\alpha = \dfrac{d\mathbf{\mathcal{F}}_\alpha}{d\Sigma_\alpha}$.

\begin{align*}
  \mathbf{t}_n d\Sigma_0 + \sum_{\alpha=1}^3 \mathbf{t}_\alpha d\Sigma_\alpha
  &= \rho \left( \dfrac{d\mathbf{v}}{dt} - \mathbf{f} \right) \, dV \\
  \mathbf{t}_n d\Sigma_0
  + \sum_{\alpha=1}^3 \mathbf{t}_\alpha (\mathbf{n} \cdot \mathbf{r}_\alpha) |\mathbf{r}^\alpha| d\Sigma_0 &= \rho \left( \dfrac{d\mathbf{v}}{dt} - \mathbf{f} \right) \, dV \\
  \mathbf{t}_n - \sum_{\alpha=1}^3 \mathbf{t}_\alpha (\mathbf{n} \cdot \mathbf{r}_\alpha) |\mathbf{r}^\alpha| &= \rho \left( \dfrac{d\mathbf{v}}{dt} - \mathbf{f} \right) \dfrac{dV}{d\Sigma_0}
\end{align*}
Совершим предельный переход, стягивая $dV$ в точку, тогда
$h=\max_\alpha \left\{ h_\alpha \right\} , d\Sigma_0 = O(h^2);
dV = O(h^3) \Rightarrow \dfrac{dV}{d\Sigma_0} = O(h)$. Тогда получается, что
правая часть -- бесконечно малая по сравнению с левой:
\[
  \mathbf{t}_n = \sum_{\alpha=1}^3 (\mathbf{n} \cdot \mathbf{r}_\alpha) |\mathbf{r}^\alpha| \mathbf{t}_\alpha.
\]
так как $|\mathbf{r}^\alpha| = \sqrt{g^{\alpha\alpha}}$, то введём обозначение:
$\mathbf{t}^\alpha \equiv \mathbf{t}_\alpha \sqrt{g^{\alpha\alpha}}$. Тогда
\begin{align*}
  \mathbf{t}_n
  &= \sum_{\alpha=1}^3 (\mathbf{n} \cdot \mathbf{r}_\alpha) \mathbf{t}^\alpha, \\
  \mathbf{t}_n
  &= \mathbf{n} \sum_{\alpha=1}^3 \mathbf{r}_\alpha \otimes \mathbf{t}^\alpha.
\end{align*}

Тензор $T = \sum_{\alpha=1}^3 \mathbf{r}_\alpha \otimes \mathbf{t}^\alpha
= \mathbf{r}_i \otimes \mathbf{t}^i$ называется \emph{тензором истинных
деформаций Коши}.

Тогда $\mathbf{t}_n = \mathbf{n} \cdot T$ -- \emph{формула Коши}.

\begin{remark}
  Ранее мы ввели тензор градиента деформации:
  $F = \mathbf{r}_i \otimes \mathring{\mathbf{r}}^i$: $\mathbf{r}_i = F \cdot \mathring{\mathbf{r}}_i$.
  Теперь мы получили тензор истинных деформаций Коши:
  $T = \mathbf{r}_i \otimes \mathbf{t}^i$: $\mathbf{t}_n = T^T \cdot \mathbf{n}$.
\end{remark}

\begin{remark}
  Для тела $V$ и произвольной точки $M$ и проходящей через эту точку площадки
  $d\Sigma$ (для каждой площадки будет свой вектор напряжений $\mathbf{t}_n$),
  но существует такой тензор, что можно узнать любое такое напряжение 
  с помощью действия этого тензора на $\mathbf{n}$.
\end{remark}

\begin{remark}
  Тензор напряжений $T$ введён в актуальной конфигурации. Для него имеет место
  формула:
  \[
    d\mathbf{\mathcal{F}} = \mathbf{t}_n d\Sigma = \mathbf{n} \cdot T d\Sigma.
  \]
\end{remark}

\subsection{Тензор напряжений Теолы-Кирхгофа}

Рассмотрим $\mathring{\mathbf{t}} = \dfrac{d\mathbf{\mathcal{F}}}{d\Sigma_0}$.
\[
  d\mathbf{\mathcal{F}} = \mathring{\mathbf{t}}_n d\Sigma_0
  = \mathbf{n} \cdot T d\Sigma = T^T \cdot \mathbf{n} d\Sigma,
\]
но $\mathbf{n} d\Sigma
= \sqrt{g/\mathring{g}} F^{-1T} \mathring{\mathbf{n}} d\mathring{\Sigma}$, 
тогда
\[
  d\mathbf{\mathcal{F}}
  = \sqrt{\dfrac{g}{\mathring{g}}} T^T \cdot F^{-1T} \cdot \mathbf{n} d\Sigma
\]

\[
  \mathbf{t}_n
  = \sqrt{\dfrac{g}{\mathring{g}}} \left( F^{-1} \cdot T \right)^T \mathring{\mathbf{n}}
\]

\[
  \mathring{\mathbf{t}}_n = \mathbf{\mathring{n}} \cdot \sqrt{\dfrac{g}{\mathring{g}}} F^{-1} T
\]

Введем тензор Теолы-Кирхгофа: $P=\sqrt{\dfrac{g}{\mathring{g}}} F^{-1} T$:
\[
  \begin{cases}
    \mathring{\mathbf{t}}_n = \mathring{\mathbf{n}} \cdot P, \\
    \mathbf{t}_n = \mathbf{n} \cdot T.
  \end{cases}
\]

\[
  \begin{cases}
    d\mathbf{F} = \mathbf{t}_n d\Sigma = \mathbf{n} \cdot T d\Sigma, \\
    d\mathbf{F} = \mathring{t}_n d\mathring{\Sigma} = \mathring{\mathbf{n}} \cdot P d\mathring{\Sigma}
  \end{cases}
  
\]
