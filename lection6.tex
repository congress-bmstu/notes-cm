\subsection{Геометрическая картина преобразования малой окрестности произвольной материальной
точки $M$}

% TODO рисунок

Согласно теореме, полярное разложение градиента деформации: 
\[
  F = O \cdot U = V \cdot O
\]

Тогда
\[
  d\mathbf{x} = O \cdot U \cdot d\mathring{\mathbf{x}} = V \cdot O \cdot d \mathring{\mathbf{x}}.
\]

Введём обозначения:
\[
  \begin{cases}
    d\mathbf{x}' = U \cdot d\mathring{\mathbf{x}}, \\
    d\mathbf{x} = O \cdot d\mathbf{x}',
  \end{cases}
\]
\[
  \begin{cases}
    d\mathbf{x}' = O \cdot d\mathring{\mathbf{x}}, \\
    d\mathbf{x} = V \cdot d\mathbf{x}',
  \end{cases}
  
\]

таким образом, с помощью полярного разложения преобразование малой окрестности
$dV$ точки $M$ при переходе из $\mathring{K}$ в $K$ можно представить как суперпозицию
двух преобразований. Причём это можно сделать двумя способами:
\begin{itemize}
  \item преобразование с помощью $U$ или $V$ (изменение углов и преобразование длин);
  \item преобразование поворота как жесткого целого.
\end{itemize}

\begin{example}
  Геометрическая картина преобразования малой окрестности произвольной материальной
точки $M$:
  \begin{itemize}
    \item Первый способ:
      Если $d\mathring{V}$ -- куб. На рисунке представлено действие тензора $U$ (или $V$),
      а далее действие тензора $O$ -- просто поворот.
      % TODO рисунок три кубика
      Меняются углы и длины ребёр -- так действует $U$. После этого действует тензор $O$, который 
      сохраняет углы и длины.

    \item Второй способ:
      сначала выполнен поворот, потом применён тензор $V$.
  \end{itemize}
\end{example}



\subsection{Скоростные характеристики движения сплошной среды}

Введём понятие \emph{скорости} материальной точки $M$.

Рассмотрим закон движения материальной точки сплошной среды:
\begin{equation}\label{lec_6:eq_zakon_dv}
  \mathbf{x} = \mathbf{x} (X^i, t), X^i \in V_{x}, t \in [0, t_{max})
\end{equation}
По предплоложению, закон движения гладкий в каждой точке, а также невырожденный.

Дифференцируя \eqref{lec_6:eq_zakon_dv}:
\[
  \mathbf{v}(X^i, t) \equiv \left. \dfrac{\partial \mathbf{x}}{\partial t} (X^i, t) \right|_{X^i}
\]

Разложим закон движения по декартовому базису:
$\mathbf{x} = x^i \bar{\mathbf{e}}_i = x^i(X^j, t) \bar{\mathbf{e}}_i$.
Тогда: $\mathbf{v} = \dfrac{\partial }{\partial t} \left( x^i \bar{\mathbf{e}}_i \right) 
  = \dfrac{\partial x^i}{\partial t} \bar{\mathbf{e}}_i = \bar{v}^i \bar{\mathbf{e}}_i$.
Декартовы компоненты вектора скорости:
$\bar{v}^i (X^j, t) = \dfrac{\partial x^i}{\partial t} (X^j, t)$.

Рассмотрим произвольное гладкое векторное поле, переменное по $X^i$ и $t$:
$\mathbf{a} = \mathbf{a}(X^i, t) = \mathbf{a}(X^j(x^k, t), t) = \tilde{\mathbf{a}} (x^k, t)$.
Тогда рассмотрим производную:
$ \dfrac{d\mathbf{a}}{dt} \equiv \left.\dfrac{\partial \mathbf{a}}{\partial t} (X^j, t) \right|_{X^i}$ -- полная производная от векторного поля по времени $t$ (aka \emph{субстанциональная}).

Рассмотрим теперь представление этой полной производной:
\[
  \dfrac{d\mathbf{a}}{dt} = \dfrac{d}{dt} \tilde{\mathbf{a}}(x^k, t)
  = \dfrac{d}{dt} \tilde{\mathbf{a}} (x^k(X^i, t), t)
  = \dfrac{\partial \hat{\mathbf{a}}}{\partial x^k} \cdot \dfrac{\partial x^k}{\partial t}
  + \left. \dfrac{\partial \hat{\mathbf{a}}}{\partial t} \right|_{x^k}
\]

Преобразуем первое слагаемое:
\begin{multline*}
  \dfrac{\partial \tilde{\mathbf{a}}}{\partial x^k} \cdot \dfrac{\partial x^k}{\partial t} 
  = \dfrac{\partial \mathbf{a}}{\partial X^i} \cdot \dfrac{\partial X^i}{\partial x^k} \cdot \dfrac{\partial x^k}{\partial t}
  = \dfrac{\partial \mathbf{a}}{\partial X^i} \cdot \tensor{P}{^i_k} \cdot \mathbf{v}^k (x^i, t)
  = \mathbf{v}^j \cdot \delta^k_j \cdot \tensor{P}{^i_k} \cdot \dfrac{\partial \mathbf{a}}{\partial X^i} = \\
  = \mathbf{v}^j \cdot \bar{\mathbf{e}}_j \cdot \bar{\mathbf{e}}^k \cdot \tensor{P}{^i_k} \cdot \dfrac{\partial \mathbf{a}}{\partial X^i}
  = \mathbf{v} \cdot \mathbf{r}^i \otimes \dfrac{\partial \mathbf{a}}{\partial X^i} 
  = \mathbf{v} \cdot \nabla \otimes \mathbf{a}
\end{multline*}

Таким образом:
\begin{equation}\label{lec_6:full_derivative}
  \dfrac{d \mathbf{a}}{dt}
  = \mathbf{v} \cdot \nabla \otimes \mathbf{a}
  + \left. \dfrac{\partial \hat{\mathbf{a}}}{\partial t} \right|_{x^k}
\end{equation}

% \[
%   \dfrac{d}{dt} \mathbf{a} = \left. \dfrac{\partial \mathbf{a}}{\partial t} (x^i, t) \right|_{x^i}
%     + \mathbf{v} \cdot \nabla \otimes \mathbf{a}
% \]


$\mathbf{v} \cdot \nabla \otimes \mathbf{a}$ -- конвективная производная от векторного поля $\mathbf{a}$.

% TODO рисунок какие-то пауки чото палки синусоида чиво

Обобщением формулы \eqref{lec_6:full_derivative} для скаляров будет:
\[
  \dfrac{d\varphi}{dt} (x^i, t) = \left. \dfrac{\partial \varphi}{\partial t} \right|_{x^i}
  + \mathbf{v} \cdot \nabla \varphi.
\]

Для тензорного поля:
\[
  \dfrac{d\Omega}{dt} = \left. \dfrac{\partial \Omega}{\partial t} (x^i, t) \right|_{x^i}
    + \mathbf{v} \cdot \nabla \otimes \Omega.
\]

Еще введём интересный объект, рассмотрение свойств которого даже не входит в курс:

\begin{definition}
  $D = \dfrac{1}{2} \left( \nabla \otimes \mathbf{v} + \nabla \otimes \mathbf{v}^T \right) $
  называется \emph{тензором скоростных деформаций} (симметричный).
\end{definition}



\section{Законы сохранения}

\subsection{Закон сохранения массы}

\paragraph{Аксиома 4. Закон сохранения массы}
Для всякой сплошной среды $V$ в $\mathcal{K}$ в любой момент времени $t>0$
существует скалярная функция $M(V, t)$, называемая \emph{массой тела} (массой
сплошной среды), и обладающая следующими свойствами:
\begin{itemize}
  \item Положительность: $M>0, M \in \mathbb{R}_{+}$;
  \item Аддитивность: $M(V_1 \bigcup V_2, t) = M(V_1, t) + M(V_2, t)$
    для любого разбиения области $V$, в том числе и на континуальное разбиение.
  \item Инвариантность по отношению к любому движению: $M(V_1, t) = \operatorname{const}$
    $\Leftrightarrow$ $ \dfrac{dM}{dt} = 0 $, если тело состоит из одних и тех 
    же материальных точек.
\end{itemize}

Массу нельзя определить ни через какие введёные выше понятия. Впервые масса появляется
в этой аксиоме.

% TODO рисунок

\begin{corollary}
  В силу аддитивности массы массу сплошной среды $V$ можно представить следующим образом:
  $V = \int_V dV,$ так как каждая сплошная среда, в том числе $dV$ обладает
  массой, то обозначим массу элементарного объёма $dV$ как $dm$ (следует из аксиомы и из
  принципов дифференциального исчисления). Тогда в силу аддитивности:
  $M(V, t) = \int_V dm$.
\end{corollary}

\begin{definition}
  % TODO рисунок

  Введём отношение $ \dfrac{dm}{dV} \equiv \rho > 0$ и назовём его \emph{плотностью}
  вещества (согласно принятым обозначениям, $dV$ -- объём области $dV$).

  В отличие от массы, плотность определена в точке -- это предел отношения
  массы к объёму окрестности: $\rho = \lim_{|\Delta V| \to 0} \dfrac{\Delta m}{|\Delta V|} = \dfrac{dm}{dV}$.
  Таким образом, $\rho(X^i, t)$ -- скалярное поле.
\end{definition}

\begin{corollary}
  Следовательно, $dm = \rho dV$. Тогда масса тела: $M = \int_V \rho \, dV$, $\rho(x^i, t), x^i \in V(t), t\in[0, t_{max})$
\end{corollary}

Если $ \dfrac{dM}{dt} = 0 $, тогда закон сохранения массы в интегральной форме:
\[
  \dfrac{d}{dt} \int\limits_{V(t)} \rho dV = 0
\]

\begin{corollary}
  Так как масса тел сохраняется для любых моментов времени, если тела состоят 
  из одних и тех же точек, то для элементарного объёма $d\mathring{V}$ и $dV$
  масса должна сохраняться: если $d\mathring{V} \to d\mathring{m} \mathring{\rho} d\mathring{V}$
  и $dV \to dm = \rho dV$, то
  \begin{equation}\label{lec_6:eq_lagrange_1form}
    dm = d\mathring{m} \Leftrightarrow
    \mathring{\rho} d\mathring{V} = \rho dV.
  \end{equation}
  -- закон сохранения массы в дифференциальной форме, или уравнение неразрывности в 
  переменных Лагранжа в форме №1.
  .
  % TODO рисунок
\end{corollary}



\subsection{Формы №2 и №3 уравнения неразрывности}

Вспомним, что такое $d\mathring{V}$: $d\mathring{V} = d\mathbf{x}_1 \cdot \left( d\mathring{\mathbf{x}}_1 \times d\mathring{\mathbf{x}}_3 \right) $ -- смешанное произведение, если $d\mathring{\bar{x}}_\alpha = \vec{MM}_\alpha$, то $d\mathbf{x}_\alpha = \vec{MM}_\alpha$.

Так как $d\mathring{\mathbf{x}}_\alpha = \mathring{\mathbf{r}}_\alpha dX^\alpha$,
$d\mathbf{x}_\alpha = \mathbf{r}_\alpha dX^\alpha$, то $d\mathring{V} = \mathring{\mathbf{r}}_1 \cdot \left( \mathring{\mathbf{r}}_2 \times \mathring{\mathbf{r}}_3 \right) dX^1 dX^2 dX^3 $,
и, аналогично: $dV = \mathbf{r}_1 \cdot \left(\mathbf{r}_2 \times \mathbf{r}_3 \right) dX^1 dX^2 dX^3$, тогда $ \dfrac{dV}{d\mathring{V}} = \sqrt{\dfrac{g}{\mathring{g}}} $.

Вспомним \eqref{lec_6:eq_lagrange_1form}: $ \dfrac{\rho}{\mathring{\rho}} = \sqrt{\dfrac{g}{\mathring{g}}} $, тогда
\[
  \rho \sqrt{g} = \mathring{\rho} \sqrt{\mathring{g}}
\]
-- вторая форма уравнения неразрывности в переменных Лагранжа.

$\dots$

\[
  \sqrt{\dfrac{g}{\mathring{g}}}
  = \det\left(\dfrac{\dfrac{\partial x^i}{\partial X^j}}{\dfrac{\partial \mathring{x}^k}{\partial X^j}}\right)
  = \det\left({\dfrac{\partial x^i}{\partial X^j} \cdot \dfrac{\partial X^j}{\partial \mathring{x}^k}}\right)
  = \det \left( \dfrac{\partial x^i}{\partial \mathring{x}^k}  \right) 
\]

Но $F = \left( \dfrac{\partial x^i}{\partial \mathring{x}^j}  \right) \bar{\mathbf{e}_i} \otimes \bar{\mathbf{e}}^j$,
и $\det F = \det \left( \dfrac{\partial x^i}{\partial \mathring{x}^j}  \right) $.

Таким образом: $\sqrt{\dfrac{g}{\mathring{g}}} = \det F$.

\[
  \dfrac{\rho}{\mathring{\rho}} = \sqrt{\dfrac{\mathring{g}}{g}} = \dfrac{1}{\det F}
  \Rightarrow
  \mathring{\rho} = \rho \cdot \det F.
\]

То есть именно геометрия определяет изменение плотности (но не начальную плотность), несмотря
на то, что масса возникает независимо от геометрии.



\subsection{Дифференцирование интегралов по подвижному объёму}

\[
  M(t) = \int_{V(t)} \rho dV
  = \int_{V(0)} \rho(X^i, 0) \, d\mathring{V}
  = \int_{\mathring{V}} \mathring{\rho} \, d\mathring{V}
  = \mathring{M}
\]
.

Таким образом, всегда можно сказать, что $\int_V \rho \, dV = \int_\mathring{V} \mathring{\rho} \, d\mathring{V}$

Рассмотрим далее следующий интеграл: $\int_{V(t)} \mathbf{a} (x^i, t) \, dV$
% TODO рисунок ежа

Вычислим $\dfrac{d}{dt} \int_{V(t)} \mathbf{a}(x^i, t) \, dV$. Согласно правилам
замены переменных от $x^i$ к $\mathring{x}^i$: $x^i = x^i(\mathring{x}^j, t)$
при фиксированном $t$. Тогда:
\begin{multline*}
  \dfrac{d}{dt} \int_{V(t)} \mathbf{a}(x^i, t) \, dV
  = \dfrac{d}{dt} \int_{\mathring{V}} \mathbf{a}(\mathring{x}^i, t) \det \left( \dfrac{\partial x^i}{\partial \mathring{x}^j}  \right) \, d\mathring{V}
  = \dfrac{d}{dt} \int_{\mathring{V}} \mathbf{a}(\mathring{x}^i, t) \sqrt{\dfrac{g}{\mathring{g}}} \, d\mathring{V} = \\
  = \int_{\mathring{V}} \dfrac{d}{dt} \left( \mathbf{a} \sqrt{\dfrac{g}{\mathring{g}}} \right) \, d\mathring{V}
  = \int_{\mathring{V}} \dfrac{1}{\sqrt{\mathring{g}}} \left( \dfrac{d\mathbf{a}}{dt}\sqrt{g} + \mathbf{a} \dfrac{d\sqrt{g}}{dt} \right) \, d\mathring{V} = 
\end{multline*}
Имеет место следующая формула (без доказательства): $\dfrac{d}{dt}\sqrt{g} = \sqrt{g} \, \nabla \cdot \mathbf{v}$.
Тогда продолжая:
\begin{multline*}
  = \int_{\mathring{V}} \dfrac{\sqrt{g}}{\sqrt{\mathring{g}}} \left( \dfrac{d\mathbf{a}}{dt} + \mathbf{a} \cdot \nabla \cdot \mathbf{\mathring{v}} \right) \, d\mathring{V}
  = \int_{V(t)} \left( \dfrac{d \mathbf{a}}{dt} + \mathbf{a} \nabla \cdot \mathbf{v} \right) \, dV = \\
  = \int_{V(t)} \left( \left. \dfrac{\partial \mathbf{a}}{\partial t} \right|_{x^i} 
    + \mathbf{v} \cdot \nabla \otimes \mathbf{a} + \mathbf{a} \nabla \cdot \mathbf{v}\right) dV
  = \int_{V(t)} \left( \dfrac{\partial \mathbf{a}}{\partial t} + \nabla \cdot \left( \mathbf{v} \otimes \mathbf{a} \right)  \right) dV
\end{multline*}

Таким образом:
\[
  \dfrac{d}{dt} \int_{V(t)} \mathbf{a}(x^i, t) \, dV
  = \int_{V(t)} \left( \dfrac{\partial \mathbf{a}}{\partial t} + \nabla \cdot \left( \mathbf{v} \otimes \mathbf{a} \right)  \right) dV
\]
-- формула дифференцирования интеграла по переменному объёму (обобщение формулы
дифференцирования интеграла с переменным верхним пределом).



\subsection{Уравнение неразрывности в переменных Эйлера}

Применим эту формулу к частному случаю $\mathbf{a} = \varphi \bar{\mathbf{e}}_1$, где
$\varphi(x^i, t)$ -- скалярное поле.
\[
  \dfrac{d}{dt} \int_V \varphi \bar{\mathbf{e}}_1 \, dV
  = \int_V \left( \dfrac{\partial \varphi}{\partial t} \bar{\mathbf{e}}_1 + \nabla \cdot \left( \mathbf{v} \otimes \varphi \bar{\mathbf{e}}_1 \right)  \right) \, dV
  = \int_V \left( \dfrac{\partial \varphi}{\partial t}  + \nabla \cdot (\varphi \mathbf{v}) \right) dV
\]
В качестве $\varphi$ выберем плотность:
\[
  \dfrac{d}{dt} \int_V \rho dV
  = \int_V \left( \dfrac{\partial \rho}{\partial t} + \nabla (\rho \mathbf{v}) \right) \, dV
\]

Из закона сохранения массы в интегральной форме следует, что 
\begin{equation}\label{lec_6:eq_mass_integ}
  \int_V \left( \dfrac{\partial \rho}{\partial t} + \nabla \cdot (\rho \mathbf{v}) \right) \, dV = 0
\end{equation}

% TODO чтото тут

\[
  \int_{V(t)} \Omega dV = 0
\]
-- если $V(t)$ -- некоторая конкретная область, тогда это уравнение -- интегральное
(Вольтерры), если $V(t)$ -- произвольная (т.е. ищем решения, верные для целого класса
областей), тогда $\Omega = 0$. Вот таким образом можно перейти от интегрального
представления закона к дифференциальному.

Применим такую процедуру к \eqref{lec_6:eq_mass_integ}:
\[
  \dfrac{\partial \rho}{\partial t}  + \nabla \cdot (\rho \mathbf{v}) = 0, 
  \rho, \mathbf{v} // (x^i, t), x^i \in V, t \in [0, t_{max}]
\]
-- уравнение неразрывности в переменных Эйлера. 

Это одно из уравнений газовой динамики. Оно имеет дивергентную форму --
это такая форма, когда перед всеми частными производными коэффициенты единица:
в декартовом базисе это уравнение принимает вид:
\[
  \dfrac{\partial \rho}{\partial t}  + \dfrac{\partial \rho \mathbf{v}^i}{\partial x^i} = 0.
\]
также оно первого порядка, гиперболического типа.

Отметим, что все законы сохранения будут иметь похожую форму.

Преобразуем:
\begin{align*}
  \dfrac{\partial \rho}{\partial t} + \nabla \rho \mathbf{v} + \rho \nabla \cdot \mathbf{v} &= 0 \\
  \left( \dfrac{\partial \rho}{\partial t} + \mathbf{v} \nabla \rho \right) &= - \rho \nabla \mathbf{v} \\
  \dfrac{d \rho}{dt} &= - \rho \nabla \cdot \mathbf{v}
\end{align*}
последнее называют уравнением неразнывности в полных дифференциалах.

\subsection{Закон изменения количества движения (закон сохранения импульса)}

Обоснование этого закона (нестрогое). Рассмотрим точки массами $m_i$ (в сплошной среде такого
быть не может), каждая со скоростью $\mathbf{v}_i$. Какие-то точки пусть будут соединены
какими-то связями. Вектор $m_i \mathbf{v}_i$ назовём вектором импульса (количества движения)
материальной точки. Рассмотрим $\dfrac{d}{dt} \sum_{i=1}^N m_i \mathbf{v}_i = \sum_{i=1}^N \mathbf{F}_i^{(e)}$ ($\mathbf{F}_i^{(e)}$ -- внешние силы, а внутренние силы компенсируются).

% TODO рисунок

Пока нету понятия силы, и оказывается, что нельзя ввести силу из того, что мы уже имеем.

\begin{definition}
  % TODO рисунок
  Векторная величина:
  \[
    \mathbf{I} = \int_V \mathbf{v} \, dm
  \]
  называется вектором \emph{количества движения} (вектор импульса) сплошной среды.
\end{definition}


\paragraph{Аксиома 5. Закон изменения количества движения}

Любой паре $V_1, V_2$ тел сопоставляется вектор-функция $\mathcal{F}(V_1, V_2, t)$, называемая
вектором силы взаимодействия тел $V_1$ и $V_2$, обладающая следующими свойствами:
\begin{itemize}
  \item Аддитивность: $\mathcal{\mathbf{F}}(V_1 \bigcup V_1', V_2, t)
    = \mathcal{\mathbf{F}}(V_1, V_2, t) + \mathcal{\mathbf{F}}(V_1', V_2, t)$
    (аналогично для разбиения тела $V_2$: $\mathcal{\mathbf{F}} (V_1, V_2\bigcup V_2', t)
    = \mathcal{\mathbf{F}}(V_1, V_2, t) + \mathcal{\mathbf{F}}(V_1, V_2', t)$). Это
    условие выполняется для любого (в том числе континуального) разбиения.

  \item Для любого тела $V$, изменение его вектора количества движения $\mathbf{I}$
    определяется его суммарным вектором сил $\mathcal{\mathbf{F}}$ взаимодействия
    данного тела со всеми окружающими его телами:
    \begin{equation}\label{axiom-5}
      \dfrac{d\mathbf{I}}{dt} = \mathcal{\mathbf{F}},
    \end{equation}
    где $\mathbf{I} = \int_V \rho \mathbf{v} \, dV$ -- вектор количества движения (импульса)
    тела $V$, а $\mathcal{\mathbf{F}}$ -- суммарная сила взаимодействия данного
    тела со всеми остальными.
    Причем некоторые части этого вектора сил могут быть нулевым.
\end{itemize}
