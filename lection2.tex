% Лекция 2 -- 2024-02-14
\subsection{Тензоры в евклидовом провстранстве}
В евклидовом пространстве можно ввести, например, \emph{метрическую
матрицу}
\[
  g_{ij} = \mathbf{e}_i \cdot \mathbf{e}_j,
\]

Также еще вот $\forall \mathbf{a} \in \mathcal{E}_n, \exists a^i :  \mathbf{a} = a^i \mathbf{e}_i$.

Еще вот есть тензоры. Например, градиент деформации:
\[
  F = \mathbf{r}_i \otimes \mathring{\mathbf{r}}^i,
\]

\begin{figure}[H]
	\centering
	\includesvg[scale = 0.8]{lec02_vecs1}
\end{figure}
где $\mathring{\mathbf{r}}_i \cdot \mathring{\mathbf{r}}^j = \delta_i^j$,
  $\otimes$ -- тензорное произведение.
  

\begin{wrapfigure}{r}{0.5\textwidth}
	\centering
	\includesvg[scale=0.8]{lec02_RandLvecs}
\end{wrapfigure}

Если совместить базисы (перенести к началу координат перенести), то получим объект, состоящий
из векторов: $\mathring{\mathbf{r}}_1$, $\mathring{\mathbf{r}}_2$, $\mathring{\mathbf{r}}_3$, $\mathbf{r}_1$,
$\mathbf{r}_2$, $\mathbf{r}_3$. Запишем эти вектора в таком порядке:
\[
  \mathbf{r}_1 \mathring{\mathbf{r}}_1 \mathbf{r}_2 \mathring{\mathbf{r}}_2 \mathbf{r}_3 \mathring{\mathbf{r}}_3 = \mathbf{r}_i \mathbf{r}^i,
\]
это -- \emph{векторный набор}. В каждой паре первый назовём \emph{левым} вектором, а
второй -- \emph{правым}. Сейчас мы этот векторный набор собрали из двух базисов, но вот потом
разберёмся как его построить по каким-то просто наборам векторов. 

%\begin{figure}[H]
%	\centering
%	\includesvg[scale=0.8]{RandLvecs}
%\end{figure}

Оказывается, что если взять такой векторный набор, то на нём можно построить все операции, которые
нас интересуют. Тогда $F = [\mathbf{r}_i \mathbf{r}^i]$ (суммирование не подразумевается).

\begin{wrapfigure}{r}{0.5\textwidth}
	\centering
	\includesvg[scale=0.6]{lec02_blobs}
\end{wrapfigure}

Один из способов определить тензор (второго ранга):

\begin{definition}[тензора второго ранга]
  Тензор второго ранга -- это класс эквивалентости векторных наборов.
\end{definition}

\[
  \mathbf{r}_i = F \cdot \mathbf{r}_i^0.
\]
\newline

\subsection{Наиболее основные подходы к определению тензоров (2 ранга)}

\begin{definition}
  Тензор -- такой опрератор линейного преобразования векторов, который действует таким образом:
  \[
    \mathbf{b}, \mathbf{a} \in \mathcal{E}_n : \mathbf{a} = T \cdot \mathbf{b}.
  \]
\end{definition}

\begin{definition}
  Тензор -- некоторый инвариантный объект, который в некотором базисе (диадном) $\mathbf{e}_i \otimes \mathbf{e}_j$ имеет компоненты $T^{ij}$ (матрица компонент).
  При переходе в другой базис $\mathbf{e}_i \to \mathbf{e}_i' = Q^j_{\, i} \mathbf{e}_j$ компоненты тензора
  будут преобразовываться по аналогии с компонентами вектора:
  \[
    \mathbf{a} = a^i \mathbf{e}_i = {a^i}' {\mathbf{e}_i}' = a^i P^k_{\,i} \mathbf{e}_k'
    \Rightarrow
    {a^k}' = P^k_{\, i} a^i.
  \]

  \[
    {T^{ij}}' = P^i_{\, k} P^j_{\, l} T^{kl}.
  \]

\end{definition}

Получается, тензор мы вводим с помощью некоторого порождающего пространства:
$\mathbf{a} \in \mathcal{L}_n, \mathbf{e}_i, \mathbf{e}_i' \in \mathcal{L}_n$.

Впринцепе определение правильное, но вызывает некоторые вопросы:
\begin{enumerate}
  \item что такое инвариантный объект? -- ну то, что он не зависит от базиса (так же как и векторы).

  \item где разложение тензора по базису? т.е. как найти его координаты хоть в каком-то базисе? 
    $T = T^{ij} e_i \otimes e_j$ -- здесь непонятно что это за базисные диады.

  \item порождающее пространство является линейным. Должны прийти к тому, что тензор является
    линейным оператором. таким образом, оба определения являются трактованием части свойств
    
    \begin{figure}[H]
    	\centering
    	\includesvg[scale=0.8]{lec02_squares}
    \end{figure}

  \item нет самой конструкции тензора -- вот в геометрическом определении выше четко алгоритм 
    построения приведён.
\end{enumerate}

Вот у этих дядей тензоры строятся аналогично нашему построению:
\begin{itemize}
  \item Ефимов, Розендорнд
  \item Победря Б.Е.
  \item Тарлаковский Д.В.
\end{itemize}
однако, векторные наборы у нас вполне конкретной длины, а у них он произвольной длины.

\begin{definition}[геометрическое]
Пусть
  \begin{enumerate}
    \item $\mathcal{L}_n$ -- линейное (векторное) пространство -- \emph{порождающее пространство}.
      Выберем 2 системы векторов в $\mathcal{L}$: $\mathbf{a}_i, \mathbf{b}^{[i]}, i = \overline{1, n}$.
      (квадратные скобки -- просто обозначение).

    \item Построим формальный векторный набор из $\mathbf{a}_i, \mathbf{b}^{[j]}$ длины $2n$:
      Вектора из $\mathbf{a}_i$ называем левыми, а из $\mathbf{b}^{[i]}$ -- правыми: 
      $A \equiv \mathbf{a}_1 \mathbf{b}^{[1]} \mathbf{a}_2 \mathbf{b}^{[2]} \dots \mathbf{a}_n \mathbf{b}^{[n]}
      \equiv \mathbf{a}_i \mathbf{b}^{[i]}$.
      Здесь у нас n пар векторов (и в каждой паре есть левый $\mathbf{a}_i$ и правый вектор
      $\mathbf{b}^{[j]}$).

    \item Введем операции с векторными наборами:
      \begin{enumerate}
        \item сложение однотипных векторных наборов. Однотипными будем называть такие наборы, у 
          которых совпадают хотя бы либо все левые либо все правые векторы:
          \[
            A_1 \equiv \mathbf{a}_i \mathbf{b}^{[i]} \leftrightarrow A_2 = \mathbf{a}_i \mathbf{c}^{[i]}; 
            \quad
            A_1 \leftrightarrow A_3 = \mathbf{d}_i \mathbf{b}^{[i]}.
          \]
          тогда:
          \[
            A_1+A_2 = \mathbf{a}_i (\mathbf{b}^{[i]} + \mathbf{c}^{[i]}); \quad
            A_1+A_3 = (\mathbf{a}_i + \mathbf{d}_i) \mathbf{b}^{[i]}.
          \]
          Это частичная операция -- то есть такая, которая определена только на подмножестве всех.

        \item Умножение на число $s \in \mathbb{R}$.
          \[
            sA = (s \mathbf{a}_i) \mathbf{b}^{[i]} = \mathbf{a}_i (s \mathbf{b}^{[i]}).
          \]
        
        \item \emph{Эквивалентность} векторных наборов:
          Векторные наборы $A$ и $B$ называются эквивалентными, если выполняется хотя бы одно 
          из следующих условий:
          \begin{enumerate}
            \item векторные наборы $A$ и $B$ состоят из одних и тех же пар, но упорядоченных
              произвольным образом.
              
              Например,
                $A = a_1 b^{[1]} a_2 b^{[2]}, \, B = a_2 b^{[2]} a_1 b^{[1]}$ тогда $A \sim B$.

            \item Набор $B$ (А) может быть получен из другого набора с помощью согласованной операции
              умножения левых и правых векторов:
              \[
                A = \mathbf{a}_i \mathbf{b}^{[i]} \sim B = (s \mathbf{a}_i) (\dfrac{1}{s} \mathbf{b}^{[i]})
                \forall s \in \mathbb{R}, s \neq 0.
              \]

            \item если в A и B все векторы $\mathbf{a}_i$ и $\mathbf{b}^{[i]}$ совпадают, кроме тех пар,
              у которых хотя бы один вектор нулевой.
              Например,
              $A=\mathbf{a}_1 \mathbf{b}^{[1]} (\mathbf{a}_i \mathbf{0}) \sim B = \mathbf{a}_1 \mathbf{b}^{[1]} (\mathbf{0} \mathbf{b}^{[2]}) \sim \mathbf{a}_1 \mathbf{b}^{[1]} (\mathbf{c}_{2} \mathbf{0})$.
          \end{enumerate}

        \item Пусть теперь есть некоторый векторный набор $A = \mathbf{a}_i \mathbf{b}^{[i]}$ тогда введем 
          множество всех векторных наборов $B = \mathbf{c}_i \mathbf{d}^{[i]}$, эквивалентных $A$ и
          обозначим его $T = [A] = [\mathbf{a}_i \mathbf{b}^{[i]}]$. Таким образом определён
          \emph{тензор 2-го ранга}.
    \end{enumerate}

  \item \emph{Диады и базисные диады}. Пусть $A = \mathbf{a}_i \mathbf{b}^{[i]}$. И 
    $\exists$ не более чем 1 пара ненулевых векторов $\mathbf{a}_i \mathbf{b}^{[i]}$:
    \[
      [\mathbf{a}_1 \mathbf{b}^{[1]} \mathbf{0} \mathbf{0} \dots \mathbf{0} \mathbf{0}] = \mathbf{a}_1 \otimes \mathbf{b}^{[1]}; \quad
      [\mathbf{0} \mathbf{0} \mathbf{a}_2 \mathbf{b}^{[2]} \dots \mathbf{0} \mathbf{0}] = \mathbf{a}_2 \otimes \mathbf{b}^{[2]}; \quad
      [\mathbf{0} \mathbf{0} \mathbf{0} \mathbf{0} \dots \mathbf{a}_1 \mathbf{b}^{[1]}] = \mathbf{a}_n \otimes \mathbf{b}^{[n]};
    \]

    То есть мы научились по любой паре векторов конструировать диаду. Пусть $\mathbf{e}_i$ и
    $\mathbf{h}_j$ -- базисы в $\mathcal{L}_n$. Первый выберем в качестве левых векторов, а
    второй -- правых. Набор $[\mathbf{e}_1 \mathbf{0} \mathbf{e}_2 \mathbf{0} \dots \mathbf{e}_i \mathbf{h}_j \dots \mathbf{e}_n \mathbf{0}] = \mathbf{e}_i \otimes \mathbf{h}_j$ -- базисная диада. (вместо всех $\mathbf{e}_k, k\neq i$ ожно было поставить нули). 

  \item диадный базис. % TODO чото тут // Вроде дописал (Сеня)
  
  	\begin{equation*}
  		\mathbf{h}_j = \mathbf{e}_j \longrightarrow \mathbf{e}_i \otimes \mathbf{e}_j
  	\end{equation*}
  
    \begin{theorem}
      $\forall T = [\mathbf{a}_i \mathbf{b}^{[i]}]$ можно представить в виде: 
      \begin{equation}\label{lec_2:eq:tensor_basis}
        T = T^{ij} \mathbf{e}_i \otimes \mathbf{e}_j,
      \end{equation}
      линейная комбинация базисных диад.

      Причем:
      \[
        T = (T^{ij} \mathbf{e}_i) \otimes \mathbf{e}_j
        = [\mathbf{a}^{[j]} \mathbf{e}_j]
        = [\mathbf{e}_i \mathbf{b}^{[j]}]
      \]
    \end{theorem}
    \begin{theorem}[Следствие]
      Множество всех тензоров образует линейное пространство, где сложение векторов:
      \begin{multline*}
        T_1 = [\mathbf{a}_i \mathbf{b}^{[i]}],
        T_2 = [\mathbf{c}_j \mathbf{d}^{[j]}]; \quad
        \mathbf{a}_i = a^j_i \mathbf{e}_j,
        \mathbf{b}^{[i]} = b^{ik} \mathbf{e}_k,
        \mathbf{c}_i = c^{j}_{i} \mathbf{e}_j, 
        \mathbf{d}^{[i]} = d^{ik} \mathbf{e}_k, \quad \\
        T_1 = [(a^i \mathbf{e}_i) (b^{[ik]} \mathbf{e}_k)] = a^j_i 
        T_1 + T_2 = (a^j_i b^{ik} + c^{j}_i d^{jk}_i) \mathbf{e}_j \otimes \mathbf{e}_k
        = (T_1^{ij} + T_2^{ij}) \mathbf{e}_i \otimes \mathbf{e}_j.
      \end{multline*}
      А базисные диады образуют базис в тензорном пространстве.
    \end{theorem}
  \end{enumerate}
\end{definition}

Далее можно работать только с формулой \eqref{lec_2:eq:tensor_basis}.

Введём скалярное умножение тензоров. Для этого рассматриваем не $\mathcal{L}_n$ в качестве
порождающего, а $\mathcal{E}_n$.
\begin{align*}
  T \cdot \mathbf{a} &= (T^{ij} \mathbf{e}_i \otimes \mathbf{e}_j) \cdot (a^k \mathbf{e}_k)
  = T^{ij} a^k \mathbf{e}_i \otimes (\mathbf{e}_j \cdot \mathbf{e}_k)
  = T^{ij} a^k \mathbf{e}_i \otimes g_{jk} = \\
  &\text{Соглашение: тензорное умножение между тензором и числом опускаем} \\
  &= T^{ij} a^k g_{jk} \mathbf{e}_i
\end{align*}

\[
  T \cdot B = (T^{ij} \mathbf{e}_i \otimes \mathbf{e}_j) \cdot (B^{kl} \mathbf{e}_k \otimes \mathbf{e}_l)
  = T^{ij} B^{kl} \mathbf{e}_i \otimes (\mathbf{e}_j \cdot \mathbf{e}_k) \otimes \mathbf{e}_l
  = T^{ij} B^{kl} \mathbf{e}_i \otimes g_{jk} \otimes \mathbf{e}_l
  = T^{ij} B^{kl} g_{jk} \mathbf{e}_i \otimes \mathbf{e}_l
\]
-- тоже тензор 2-го ранга.

Двойное скалярное произведение:
\[
  T \cdot \cdot B = T^{ij} g_{jk} B^{kl} g_{il}
\]

\subsection{Точечное евклидово пространство}

\begin{definition}
  Точечным евклидовым пространством $\mathcal{E}_n^a$ (аффинным) называют пространство, в котором
  введены два типа объектов:
  \begin{enumerate}
    \item векторы $\mathbf{a}, \mathbf{b}, \mathbf{c} \in \mathcal{E}_n$;
    \item точки $A, B, C \in \mathcal{E}_n^a$.
  \end{enumerate}
  которые удовлетворяют следующим аксиомам:
  \begin{enumerate}
    \item $\forall A, B \in \mathcal{E}_n^a \, \exists! \mathbf{x} \in \mathcal{E}_n : \mathbf{x} = \mathbf{AB} \in \mathcal{E}_n^a$;
    \item $\forall A, B \in \mathcal{E}_n^a \, \exists \mathbf{0} \in \mathcal{E}_n : \mathbf{AB} + \mathbf{BC} + \mathbf{CA} = \mathbf{0}$ (равенство Шаля).
  \end{enumerate}
  Это формализация векторов и точек из пространств элементарной геометрии.
\end{definition}

В обычном евклидовом пространстве вектора это классы эквивалентности, а здесь хорошо получается 
что есть точечки и можно к ним вектор присоединить.

Если есть точка $A \in \mathcal{E}_n^a$ и вектор $\mathbf{x} \in \mathcal{E}_n$ то $\exists! B \in \mathcal{E}_n^a : \mathbf{AB} = \mathbf{x}$.

\paragraph{Некоторые свойства этого пространства}
\begin{enumerate}
  \item Рассмотрим базис $\mathbf{e}_i \in \mathcal{E}_n$ и точку $O \in \mathcal{E}_n^a$:
    $\forall M \in \mathcal{E}_n^a \, \exists! \mathbf{x} \in \mathcal{E}_n : \mathbf{OM} = \mathbf{x}$.
    Такое соотвествие $M \to \mathbf{x}$ будем называть \emph{радиус-вектором} относительно системы координат $O\mathbf{e}_i$.
    \begin{definition}
      Система координат в точечно-евклидовом пространстве: это точка $O$ и любой присоединенный к ней
      базис $\mathbf{e}_i$. Обозначается $O\mathbf{e}_i$.
    \end{definition}
   
  \item Рассмотрим СК $O\mathbf{e}_i$ и тогда для любой точки $M$ существует радиус-вектор $\mathbf{OM} = \mathbf{x} \in \mathcal{E}_n$.
    Тогда мы можем разложить $\mathbf{x}$ по базису $\mathbf{e}_i$: $\mathbf{x} = x^i \mathbf{e}_i$.
    % TODO картинка базис чото там еще // Это??? (Сеня)
    
    \begin{figure}[H]
    	\centering
    	\includesvg[scale=0.6]{lec02_vecX}
    \end{figure}

  \item Длина вектора, соединяющего точки $A, B \in \mathcal{E}_n^a$. $\mathbf{AB} = \mathbf{a}$.
    \[
      l(A, B) = |\mathbf{AB}| = |\mathbf{a}| = \sqrt{\mathbf{a}^i \cdot \mathbf{a}_i} = \sqrt{a^i a^j g_{ij}}
    \]
    
  	\begin{figure}[H]
  		\centering
  		\includesvg[scale=0.6]{lec02_vecX2}
  	\end{figure}

  \item Пространство, в котором существует понятие длины (расстояния между точками A B), но в
    общем случае нет скалярного произведения) называентся метрическим пространсвтом.
    В точечно-евклидовом простарнство тоже можно ввести понятие расстояния между точками A B:
    \[
      l(A, B) = |\mathbf{AB}| = \sqrt{\mathbf{a} \cdot \mathbf{a}}
    \]
    поэтому точечно-евклидово пространство называют метризованным.

  \item Криволинейные координаты в $\mathcal{E}_n^a$. Рассмотрим СК $O\mathbf{e}_i$ и точку $M$ с
    координатами $x^i$ в этой СК -- эти координаты будем называть декартовыми (не обязательно
    ортонормированный базис).
    
    \begin{figure}[H]
    	\centering
    	\includesvg[scale=0.6]{lec02_curvedcoord}
    \end{figure}
    
    Рассмотрим теперь функции многих переменных: $X^i = X^i (x^j)$, где в некоторой области
    $V \subset \mathcal{E}_n^a$. Поскольку $\mathcal{E}_n^a$ метризовано, можно ввести понятие 
    области -- открытого множества (для любой точки существует ее окрестность, полностью
    принадлежащая области). $\varepsilon$-окрестностью точки будем называть множество:
    $U_\varepsilon(A) = \left\{ M \in \mathcal{E}_n^a | l(A, M) < \varepsilon \right\} $.
    
    \begin{figure}[H]
    	\centering
    	\includesvg[scale=0.6]{lec02_epsilon-area}
    \end{figure}
    
    Если заданы функции вида $X^i = X^i (x^j)$, которые
    \begin{enumerate}
      \item являются гладкими в области $V$;
      \item являются невырожденными в этой же области:
        \[
          \left| \dfrac{\partial X^i}{\partial x^j} \right| \neq 0
        \]
    \end{enumerate}
    то говорим что задана криволинейная система координат.
  
    Например, если $\bar{\mathbf{e}}_i$ -- ортонормированный базис, $O\bar{\mathbf{e}}_i$ -- прямоугольная
    декартова система координат. Через любую точку $M \in V$ можно провести 3 координатные линии
    $X^\alpha = var, X^{\beta} = \const, X^\gamma = \const, \alpha \neq \beta \neq \gamma, \alpha, \beta, \gamma = 1, 2, 3$.
    
    \begin{figure}[H]
    	\centering
    	\includesvg[scale=0.8]{lec02_curvedcoord2}
    \end{figure}

	\begin{example}
		Цилиндрическая система координат. 
		
		\begin{figure}[H]
			\centering
			\includesvg[scale=1.0]{lec02_cilindercoord}
		\end{figure}
		
		\[
		\begin{cases}
			X^1 = r, \\
			X^2 = \varphi, \\
			X^3 = z = x^3
		\end{cases}
		\]
		
		$X^1 = var$ -- луч
		$X^2 = var$ -- окружность
		$X^3 = var$ -- прямая
	\end{example}
    

  \item Локальные векторы базиса. У нас есть радиус-вектор точки $M$: $\mathbf{x} = x^i \mathbf{e}_i$. 
    Из криволинейных координат и условий, которые мы наложили на них, существует обратные функции:
    $x^i = x^i(X^j) \in V_x \subset \mathcal{E}_n^a$ -- тоже гладкая и невырожденная. Тогда 
    радиус-вектор точки $M$: $\mathbf{OM} = \mathbf{x} = x^i(X^k) \bar{\mathbf{e}}_k$. Дифференцированием
    получим:
    \[
      \dfrac{\partial \mathbf{x}}{\partial X^i} = \dfrac{\partial x^i (X^k)}{\partial X^i} \bar{\mathbf{e}}_i
      \equiv \mathbf{r}_i
    \]
    Следовательно, определили локальные векторы базиса: 
    $\mathbf{r}_i \equiv \dfrac{\partial \mathbf{x}}{\partial X^i} \bar{\mathbf{e}}_i
    = Q^j_{\, i} \bar{\mathbf{e}}_i$. Матрица $Q$ называется якобиевой.

    Так как криволиненйные координаты невырожденны ($\det Q \neq 0$) $\mathbf{r}_i$ образует 
    базис. Эти вектора направлены по касательным к соответствующим координатным линиям.
    
	\rule{0.95\textwidth}{0.4pt}
	
    % TODO кусок ниже непонятно куда вставить -- он хотел это отдельно раньше, но видимо забыл 
    % и дал только в этот момент
    % TODO: Проверить норм ли, что я так оформил?? (Сеня)
    В точечно-евклидовом простарснвте существует понятие единого ортонормированного базиса такого,
    что $\forall M \in \mathcal{E}_n^a \, \exists! \mathbf{OM} = \mathbf{x} = x^i \bar{\mathbf{e}}_i$.
    
    \begin{example} \label{cilinder_surface}
    	Например:
    	\begin{figure}[H]
    		\centering
    		\includesvg[scale=0.8]{lec02_cilindercoord2}
    		\caption{Поверхность цилиндра -- неевклидово двумерное пространство}
    	\end{figure}
    \end{example}
    
    % здесь этот кусок закончен
    \rule{0.95\textwidth}{0.4pt}

  \item Метрическая матрица для локальных векторов базиса: $g_{ij} = \mathbf{r}_i \cdot \mathbf{r}_j$. 
    Обратная метрическая матрица $g^{ij} g_{jk} = \delta^i_k$.

  \item $\mathbf{r}^i = g^{ij} \mathbf{r}_j$ -- вектора взаимного базиса.
    Утверждение: $\mathbf{r}^i \cdot \mathbf{r}_j = \delta^i_j$
    Доказательство: 
\end{enumerate}

