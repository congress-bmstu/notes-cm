% Лекция 2 -- 2024-02-14
\subsection{Тензоры в евклидовом провстранстве}
В евклидовом пространстве можно ввести, например, \emph{метрическую
матрицу}
\[
  g_{ij} = \mathbf{e}_i \cdot \mathbf{e}_j.
\]

% Также еще вот $\forall \mathbf{a} \in \mathcal{E}_n, \exists a^i \colon  \mathbf{a} = a^i \mathbf{e}_i$.

Кроме матриц нам понадобятся тензоры. Например, \emph{градиент деформации} имеет вид
\[
  F = \mathbf{r}_i \otimes \mathring{\mathbf{r}}^i,
\]
\begin{figure}[H]
	\centering
	\includesvg[scale = 0.8]{lec02_vecs1}
\end{figure}
\noindentгде $\mathring{\mathbf{r}}_i \cdot \mathring{\mathbf{r}}^j = \delta_i^j$, а
символом
  $\otimes$ обозначается тензорное произведение.
  

\begin{wrapfigure}{r}{0.5\textwidth}
	\centering
	\includesvg[scale=0.8]{lec02_RandLvecs}
\end{wrapfigure}

Если совместить базисы (перенести к началу координат), то получим объект, состоящий
из векторов $\mathring{\mathbf{r}}_1$, $\mathring{\mathbf{r}}_2$, $\mathring{\mathbf{r}}_3$, $\mathbf{r}_1$,
$\mathbf{r}_2$, $\mathbf{r}_3$. Запишем эти векторы в следующем порядке:
\[
  \mathbf{r}_1 \mathring{\mathbf{r}}_1 \mathbf{r}_2 \mathring{\mathbf{r}}_2 \mathbf{r}_3 \mathring{\mathbf{r}}_3 = \mathbf{r}_i \mathbf{r}^i,
\]
получив некоторый \emph{векторный набор}. В каждой паре первый назовём \emph{левым} вектором, а
второй --- \emph{правым}. В данном случае набор был построен из векторов базиса,
но в дальнейшем это не будет иметь значения.

%\begin{figure}[H]
%	\centering
%	\includesvg[scale=0.8]{RandLvecs}
%\end{figure}

Оказывается, что если взять такой векторный набор, то на нём можно построить все операции, которые
нас интересуют. Тогда $F = [\mathbf{r}_i \mathbf{r}^i]$ (суммирование не подразумевается).

\begin{wrapfigure}{r}{0.5\textwidth}
	\centering
	\includesvg[scale=0.6]{lec02_blobs}
\end{wrapfigure}

Назовём теперь один из способов определить тензор (второго ранга).
\begin{definition}
  \emph{Тензором второго ранга} мы называем класс эквивалентости векторных наборов.
\end{definition}

\[
  \mathbf{r}_i = F \cdot \mathbf{r}_i^0.
\]



\subsection{Основные подходы к определению тензоров (ранга 2)}
\begin{definition}[алгебраическое]
  \label{tensor-def}
  \emph{Тензором} называется оператор линейного преобразования векторов,
  который действует следующим образом:
  \[
    \mathbf{a}, \mathbf{b} \in \mathcal{E}_n\colon \mathbf{a} = T \mathbf{b}.
  \]
\end{definition}

\renewcommand{\thedefinition}{\ref{tensor-def}$'$}
\addtocounter{definition}{-1}
\begin{definition}
  Тензором называется некоторый инвариантный объект, который в некотором (\emph{диадном}) базисе
   $\mathbf{e}_i \otimes \mathbf{e}_j$ имеет компоненты $T^{ij}$ (матрица компонент).
  При переходе в другой базис $\mathbf{e}_i \to \mathbf{e}_i' = Q^j_{\, i} \mathbf{e}_j$ компоненты тензора
  будут преобразовываться по аналогии с компонентами вектора, то есть, как
  известно,
  \begin{gather*}
    \mathbf{a} = a^i \mathbf{e}_i = {a^i}' {\mathbf{e}_i}' = a^i P^k_{\,i} \mathbf{e}_k'
    \Rightarrow
    {a^k}' = P^k_{\, i} a^i, \quad \text{а}\\
    {T^{ij}}' = P^i_{\, k} P^j_{\, l} T^{kl},
  \end{gather*}
  где, конечно, $ P = Q^{-1} $.
\end{definition}
\renewcommand{\thedefinition}{\arabic{definition}}

Выходит, тензор мы вводим с помощью некоторого порождающего пространства
$\mathbf{a} \in \mathcal{L}_n$, $\mathbf{e}_i, \mathbf{e}_i' \in \mathcal{L}_n$.

В целом определение корректное, но оставляет некоторые вопросы.
\begin{enumerate}
  \item Что такое инвариантный объект? Это объект, который, подобно векторам, не
    зависит от базиса;
  \item Как найти координаты тензора хоть в каком-то базисе? В формуле
    $T = T^{ij} e_i \otimes e_j$ ничего не говорится о происхождении базисных
    диад;

  \item Порождающее пространство является линейным. Значит, что бы мы ни делали,
    мы придём к тому, что тензор является
    линейным оператором. Оба определения являются трактованием лишь части
    свойств (см. рис. \ref{fig:bred}).
    \begin{figure}[H]\label{fig:bred}
    	\centering
    	\includesvg[scale=0.8]{lec02_squares}
      \caption{}
    \end{figure}

  \item Нет самой конструкции тензора. Напротив, в геометрическом определении
    выше приведён чёткий алгоритм 
    построения.
\end{enumerate}

У следующих авторов тензоры строятся аналогично нашему курсу:
\begin{itemize}[label=--]
  \item Ефимов, Розендорнд,
  \item Победря Б.\,Е.,
  \item Тарлаковский Д.\,В.
\end{itemize}
Однако векторные наборы у нас вполне конкретной длины, в то время как согласно
данным
авторам они имеют произвольную длину.

\renewcommand{\thedefinition}{\ref{tensor-def}$''$}
\addtocounter{definition}{-1}
\begin{definition}[геометрическое]
Пусть
  \begin{enumerate}
    \item $\mathcal{L}_n$ --- линейное (векторное) пространство --- \emph{порождающее пространство}.
      Выберем две системы векторов в $\mathcal{L}_n$: $\mathbf{a}_i, \mathbf{b}^{[i]}, i = \overline{1, n}$.
      (квадратные скобки --- просто обозначение).

    \item Построим формальный векторный набор из $\mathbf{a}_i,
      \mathbf{b}^{[j]}$ длины $2n$.
      Векторы из $\mathbf{a}_i$ называем левыми, а из $\mathbf{b}^{[i]}$
      правыми. Таким образом\footnote{Далее скобки и запятые будут опускаться.}, 
      $A \equiv ((\mathbf{a}_1, \mathbf{b}^{[1]}), (\mathbf{a}_2,
      \mathbf{b}^{[2]}), \dots (\mathbf{a}_n, \mathbf{b}^{[n]}))
      \equiv ((\mathbf{a}_i, \mathbf{b}^{[i]}))$.
      Здесь у нас $ n $ пар векторов (и в каждой паре $ i $ есть левый $\mathbf{a}_i$ и правый вектор
      $\mathbf{b}^{[i]}$).

    \item Введем теперь операции с векторными наборами.
      \begin{enumerate}
        \item \textsc{Сложение} однотипных векторных наборов. \emph{Однотипными} будем называть такие наборы, у 
          которых совпадают либо все левые, либо все правые векторы:
          \[
            A_1 \equiv \mathbf{a}_i \mathbf{b}^{[i]} \leftrightarrow A_2 = \mathbf{a}_i \mathbf{c}^{[i]}; 
            \quad
            A_1 \leftrightarrow A_3 = \mathbf{d}_i \mathbf{b}^{[i]}.
          \]
          Тогда
          \[
            A_1+A_2 = \mathbf{a}_i (\mathbf{b}^{[i]} + \mathbf{c}^{[i]}); \quad
            A_1+A_3 = (\mathbf{a}_i + \mathbf{d}_i) \mathbf{b}^{[i]}.
          \]
          Это частичная операция, то есть такая, которая определена только на
          подмножестве элементов (векторов). К тому же, как легко видеть, она рефлексивна,
          симметрична, но не транзитивна, поэтому не может быть названа
          эквивалентностью.

        \item \textsc{Умножение} на число $s \in \mathbb{R}$.
          \[
            sA = (s \mathbf{a}_i) \mathbf{b}^{[i]} = \mathbf{a}_i (s \mathbf{b}^{[i]}).
          \]
          Данное выражение показывает, что мы называем равными не только
          полностью совпадающие наборы.
        
        \item \textsc{Эквивалентность} векторных наборов.
          Векторные наборы $A$ и $B$ называются \emph{эквивалентными}, если выполняется хотя бы одно 
          из следующих условий:
          \begin{enumerate}
            \item Векторные наборы $A$ и $B$ состоят из одних и тех же пар, но упорядоченных
              произвольным образом.
              
              Например,
              $A = \mathbf{a}_1 \mathbf{b}^{[1]} \mathbf{a}_2 \mathbf{b}^{[2]},
              \, B = \mathbf{a}_2 \mathbf{b}^{[2]} \mathbf{a}_1 \mathbf{b}^{[1]}$, откуда $A \sim B$.

            \item Набор $A$ может быть получен из другого набора с помощью согласованной операции
              умножения левых и правых векторов:
              \[
                A = \mathbf{a}_i \mathbf{b}^{[i]} \sim B = (s \mathbf{a}_i)
                (s^{-1}\mathbf{b}^{[i]})
                \quad \forall s \in \mathbb{R},\ s \neq 0.
              \]

            \item Если в $A$ и $B$ все векторы $\mathbf{a}_i$ и $\mathbf{b}^{[i]}$ совпадают, кроме тех пар,
              у которых хотя бы один вектор нулевой.
              Например,
              $A=\mathbf{a}_1 \mathbf{b}^{[1]} (\mathbf{a}_i \mathbf{0}) \sim B = \mathbf{a}_1 \mathbf{b}^{[1]} (\mathbf{0} \mathbf{b}^{[2]}) \sim \mathbf{a}_1 \mathbf{b}^{[1]} (\mathbf{c}_{2} \mathbf{0})$.
          \end{enumerate}

        \item Пусть теперь есть некоторый векторный набор $A = \mathbf{a}_i
          \mathbf{b}^{[i]}$. Введём 
          множество всех векторных наборов $B = \mathbf{c}_i \mathbf{d}^{[i]}$, эквивалентных $A$ и
          обозначим его $T = [A] = [\mathbf{a}_i \mathbf{b}^{[i]}]$. \end{enumerate}
Таким
          образом определён
          \emph{тензор второго ранга}.
    
  \item \textsc{Диады и базисные диады}. Пусть $A = \mathbf{a}_i
    \mathbf{b}^{[i]}$, где
    существует не более чем одна пара ненулевых векторов $\mathbf{a}_i
    \mathbf{b}^{[i]}$. Положим по определению
    \[
      [\mathbf{a}_1 \mathbf{b}^{[1]} \mathbf{0} \mathbf{0} \dots \mathbf{0} \mathbf{0}] = \mathbf{a}_1 \otimes \mathbf{b}^{[1]}; \quad
      [\mathbf{0} \mathbf{0} \mathbf{a}_2 \mathbf{b}^{[2]} \dots \mathbf{0} \mathbf{0}] = \mathbf{a}_2 \otimes \mathbf{b}^{[2]}; \quad
      [\mathbf{0} \mathbf{0} \mathbf{0} \mathbf{0} \dots \mathbf{a}_1 \mathbf{b}^{[1]}] = \mathbf{a}_n \otimes \mathbf{b}^{[n]};
    \]

    Иными словами, мы научились по любой паре векторов конструировать диаду. Пусть $\mathbf{e}_i$ и
    $\mathbf{h}_j$ -- базисы в $\mathcal{L}_n$. Первый выберем в качестве левых векторов, а
    второй --- правых. Набор $[\mathbf{e}_1 \mathbf{0} \mathbf{e}_2 \mathbf{0}
    \dots \mathbf{e}_i \mathbf{h}_j \dots \mathbf{e}_n \mathbf{0}] =
    \mathbf{e}_i \otimes \mathbf{h}_j$ --- базисная диада. (Вместо всех
    $\mathbf{e}_k, k\neq i$ можно было поставить нули.) 

  \item \textsc{Диадный базис.} % TODO чото тут // Вроде дописал (Сеня)
  	\begin{equation*}
  		\mathbf{h}_j = \mathbf{e}_j \longrightarrow \mathbf{e}_i \otimes
      \mathbf{e}_j.
  	\end{equation*}
  
    \begin{theorem}
      Любой тензор второго ранга $ T = [\mathbf{a}_i \mathbf{b}^{[i]}]$ можно представить в виде линейной комбинации базисных диад
      \begin{equation}\label{lec_2:eq:tensor_basis}
        T = T^{ij} \mathbf{e}_i \otimes \mathbf{e}_j,
      \end{equation}
       причём
      \[
        T = (T^{ij} \mathbf{e}_i) \otimes \mathbf{e}_j
        = [\mathbf{a}^{[j]} \mathbf{e}_j]
        = [\mathbf{e}_i \mathbf{b}^{[j]}].
      \]
    \end{theorem}
    \begin{corollary*}
      Множество всех тензоров (второго ранга) образует линейное пространство,
      где сложение векторов подчиняется правилам
      \begin{gather*}
        T_1 = [\mathbf{a}_i \mathbf{b}^{[i]}], \quad
        T_2 = [\mathbf{c}_j \mathbf{d}^{[j]}]; \\
        \mathbf{a}_i = a^j_i \mathbf{e}_j, \quad
        \mathbf{b}^{[i]} = b^{ik} \mathbf{e}_k,\quad
        \mathbf{c}_i = c^{j}_{i} \mathbf{e}_j, \quad
        \mathbf{d}^{[i]} = d^{ik} \mathbf{e}_k,  \\
        T_1 = [(a^i \mathbf{e}_i) (b^{[ik]} \mathbf{e}_k)] = a^j_i, \quad
        T_1 + T_2 = (a^j_i b^{ik} + c^{j}_i d^{jk}_i) \mathbf{e}_j \otimes \mathbf{e}_k
        = (T_1^{ij} + T_2^{ij}) \mathbf{e}_i \otimes \mathbf{e}_j,
      \end{gather*}
      а базисные диады образуют базис в тензорном пространстве.
    \end{corollary*}
  \end{enumerate}
\end{definition}
\renewcommand{\thedefinition}{\arabic{definition}}

Далее можно работать только с формулой \eqref{lec_2:eq:tensor_basis}.

Введём скалярное умножение для тензоров. Для этого в качестве порождающего
пространства рассматриваем не линейное $\mathcal{L}_n$, а евклидово $\mathcal{E}_n$.
\[
  T \cdot \mathbf{a} &= (T^{ij} \mathbf{e}_i \otimes \mathbf{e}_j) \cdot (a^k \mathbf{e}_k)
  = T^{ij} a^k \mathbf{e}_i \otimes (\mathbf{e}_j \cdot \mathbf{e}_k)
  = T^{ij} a^k \mathbf{e}_i \otimes g_{jk} = T^{ij} a^k g_{jk} \mathbf{e}_i.
\]
(\textsc{Соглашение.} тензорное умножение между тензором и числом опускаем.)

В свою очередь
\[
  T \cdot B = (T^{ij} \mathbf{e}_i \otimes \mathbf{e}_j) \cdot (B^{kl}
  \mathbf{e}_k \otimes \mathbf{e}_l)
  = T^{ij} B^{kl} \mathbf{e}_i \otimes (\mathbf{e}_j \cdot \mathbf{e}_k) \otimes \mathbf{e}_l
  = T^{ij} B^{kl} \mathbf{e}_i \otimes g_{jk} \otimes \mathbf{e}_l
  = T^{ij} B^{kl} g_{jk} \mathbf{e}_i \otimes \mathbf{e}_l.
\]
Получили снова тензор второго ранга.

Двойное скалярное произведение определяется выражением
\[
  T \cddot B = T^{ij} g_{jk} B^{kl} g_{il}.
\]


\subsection{Точечное евклидово пространство}
\begin{definition}
  \emph{Точечным евклидовым} (\emph{аффинным}) пространством $\mathcal{E}_n^a$ называют пространство, в котором
  введены два типа объектов: векторы и точки.
  % \begin{enumerate}
  %   \item Векторы $\mathbf{a}, \mathbf{b}, \mathbf{c} \in \mathcal{E}_n$;
  %   \item Точки $A, B, C \in \mathcal{E}_n^a$.
  % \end{enumerate}
  При этом эти объекты должны удовлетворять следующим аксиомам:
  \begin{enumerate}
    % \item Для любых точек $A, B \in \mathcal{E}_n^a$ существует единственный
    %   вектор $\mathbf{x} \in
    %   \mathcal{E}_n$, такой что $\mathbf{x} = \mathbf{AB} \in \mathcal{E}_n^a$;
    \item Для любых точек $A, B \in \mathcal{E}_n^a$ существует единственный
      вектор $\mathbf{AB} \in
      \mathcal{E}_n$, такой что $A + \mathbf{AB} = B \in \mathcal{E}_n^a$;
    \item Для любых точек $A, B, C \in \mathcal{E}_n^a$ выполняется \emph{равенство Шаля} $\mathbf{AB}
      + \mathbf{BC} + \mathbf{CA} = \mathbf{0}$.
  \end{enumerate}
  Этим мы формализовали векторы и точки из пространства элементарной геометрии.
\end{definition}

В обычном евклидовом пространстве векторы суть классы эквивалентности
(параллельных направленных отрезков), а здесь есть точки, к которым
можно присоединить векторы.

Если есть точка $A \in \mathcal{E}_n^a$ и вектор $\mathbf{x} \in \mathcal{E}_n$,
то существует единственная точка $B \in \mathcal{E}_n^a$, такая что $\mathbf{AB} = \mathbf{x}$.

\paragraph{Некоторые свойства точечного евклидова пространства.}
\begin{enumerate}
  \item \textsc{Радиус-вектор.} Рассмотрим базис $\mathbf{e}_i \in \mathcal{E}_n$ и точку $O \in
    \mathcal{E}_n^a$. Тогда для любой точки
    $\forall M \in \mathcal{E}_n^a$ существует единственный вектор $ \mathbf{x}
    \in \mathcal{E}_n$ такой, что $ \mathbf{OM} = \mathbf{x}$.
    Такое соотвествие $M \mapsto \mathbf{OM}$ будем называть
    \emph{радиусом-вектором} относительно системы координат $O\mathbf{e}_i$.
    \begin{definition}
      \emph{Системой координат} в точечно-евклидовом пространстве называется точка $O$ и любой присоединенный к ней
      базис $\mathbf{e}_i$. (Обозначается $O\mathbf{e}_i$.)
    \end{definition}
    Рассмотрим систему координат $O\mathbf{e}_i$. Тогда для любой точки $M$ существует радиус-вектор $\mathbf{OM} = \mathbf{x} \in \mathcal{E}_n$.
    Тогда мы можем разложить точку $ M $ (вектор $\mathbf{x}$) по базису $\mathbf{e}_i$: $\mathbf{x} = x^i \mathbf{e}_i$.
    % TODO картинка базис чото там еще // Это??? (Сеня)
    
    \begin{figure}[H]
    	\centering
    	\includesvg[scale=0.6]{lec02_vecX}
    \end{figure}

  \item \textsc{Длина} вектора $ \mathbf a := \mathbf{AB} $, соединяющего точки $A, B \in
    \mathcal{E}_n^a$, вычисляется по формуле
    \[
      l(A, B) = |\mathbf{AB}| = |\mathbf{a}| = \sqrt{\mathbf{a}^i \cdot
      \mathbf{a}_i} = \sqrt{a^i a^j g_{ij}}.

    \]
    
  	\begin{figure}[H]
  		\centering
  		\includesvg[scale=0.6]{lec02_vecX2}
  	\end{figure}

  \item \textsc{Расстояние.} Пространство, в котором существует понятие
    расстояния между точками,
    % длины (расстояния между
    % точками $A$ и $B$),
    но в котором в
    общем случае нет скалярного произведения, называется \emph{метрическим
    пространством}.
    В точечно-евклидовом простанство тоже можно ввести понятие расстояния между
    точками $A$, $B$ по формуле
    \[
      l(A, B) = |\mathbf{AB}| = \sqrt{\mathbf{a} \cdot \mathbf{a}},
    \]
    поэтому точечно-евклидово пространство называют \emph{метризованным}.

  \item \textsc{Криволинейные координаты в $\mathcal{E}_n^a$.} Рассмотрим репер
    $O\mathbf{e}_i$ и точку $M$ с
    координатами $x^i$ в этом репере. Эти координаты будем называть
    \emph{декартовыми} (базис не обязан быть ортонормированным).
    \begin{figure}[H]
    	\centering
    	\includesvg[scale=0.6]{lec02_curvedcoord}
    \end{figure}
    Рассмотрим теперь функцию многих переменных $X^i = X^i (x^j)$ в некоторой области
    $V \subset \mathcal{E}_n^a$. Поскольку $\mathcal{E}_n^a$ метризовано, можно ввести понятие 
    области --- открытого множества (для любой точки области существует ее окрестность, полностью
    принадлежащая области). При этом $\varepsilon$-окрестностью точки будем называть множество
    $U_\varepsilon(A) := \left\{ M \in \mathcal{E}_n^a \mid l(A, M) < \varepsilon \right\} $.
    \begin{figure}[H]
    	\centering
    	\includesvg[scale=0.6]{lec02_epsilon-area}
      \caption{Понятие области и $ \varepsilon $-окрестности.}
    \end{figure}
    
    Если заданы функции вида $X^i = X^i (x^j)$, которые
    \begin{enumerate}
      \item являются гладкими в области $V$;
      \item являются невырожденными в этой же области\footnote{Видимо, имеется в
        виду, что не равен нулю якобиан --- определитель матрицы Якоби.}:
        \[
          \left| \dfrac{\partial X^i}{\partial x^j} \right| \neq 0,
        \]
    \end{enumerate}
    то говорим, что задана \emph{криволинейная система координат}.
  
    Например, пускай $\bar{\mathbf{e}}_i$ --- ортонормированный базис,
    $O\bar{\mathbf{e}}_i$ --- прямоугольная
    декартова система координат. Через любую точку $M \in V$ можно провести три
    \emph{координатные линии}
    $X^\alpha = \var, X^{\beta} = \const, X^\gamma = \const$, $\alpha \neq \beta
    \neq \gamma$, $\alpha, \beta, \gamma \in \{1, 2, 3\}$.
    
    \begin{figure}[H]
    	\centering
    	\includesvg[scale=0.8]{lec02_curvedcoord2}
    \end{figure}

	\begin{example}
		Цилиндрическая система координат. 
		
		\begin{figure}[H]
			\centering
			\includesvg[scale=1.0]{lec02_cilindercoord}
		\end{figure}
		
		\[
		\begin{cases}
			X^1 = r, \\
			X^2 = \varphi, \\
			X^3 = z = x^3.
		\end{cases}
		\]
		
		Здесь $X^1 = \var$ --- луч,
		$X^2 = \var$ --- окружность,
		$X^3 = \var$ --- прямая.
	\end{example}
    

  \item Локальные векторы базиса. У нас есть радиус-вектор точки $M$: $\mathbf{x} = x^i \mathbf{e}_i$. 
    Из криволинейных координат и условий, которые мы наложили на них, существует
    обратные функции
    $x^i = x^i(X^j) \in V_x \subset \mathcal{E}_n^a$, причём тоже гладкие и
    невырожденные. Тогда 
    радиус-вектор точки $M$ $\mathbf{OM} = \mathbf{x} = x^i(X^k) \bar{\mathbf{e}}_k$. Дифференцированием
    получим
    \[
      \dfrac{\partial \mathbf{x}}{\partial X^i} = \dfrac{\partial x^i (X^k)}{\partial X^i} \bar{\mathbf{e}}_i
      \equiv \mathbf{r}_i.
    \]
    Следовательно, мы определили локальные векторы базиса
    $\mathbf{r}_i \equiv \frac{\partial \mathbf{x}}{\partial X^i} \bar{\mathbf{e}}_i
    = Q^j_{\, i} \bar{\mathbf{e}}_i$. Матрица $Q$ называется \emph{якобиевой}.

    Так как криволиненйные координаты невырожденные ($\det Q \neq 0$), векторы $\mathbf{r}_i$ образуют 
    базис. Эти векторы направлены по касательным к соответствующим координатным линиям.
    
	% \rule{0.95\textwidth}{0.4pt}
	
    % TODO кусок ниже непонятно куда вставить -- он хотел это отдельно раньше, но видимо забыл 
    % и дал только в этот момент
    % TODO: Проверить норм ли, что я так оформил?? (Сеня)
    В точечно-евклидовом простарснвте существует понятие единого
    ортонормированного базиса --- такого,
    что $\forall M \in \mathcal{E}_n^a \, \exists! \mathbf{OM} = \mathbf{x} = x^i \bar{\mathbf{e}}_i$.
    
    \begin{example} \label{cilinder_surface}
      Напротив, таким свойством не обладают неевклидовы пространства.
    	\begin{figure}[H]
    		\centering
    		\includesvg[scale=0.8]{lec02_cilindercoord2}
    		\caption{Поверхность цилиндра --- неевклидово двумерное пространство.}
    	\end{figure}
    \end{example}
    
    % % здесь этот кусок закончен
    % \rule{0.95\textwidth}{0.4pt}

  \item Метрическая матрица для локальных векторов базиса ищется по формуле $g_{ij} = \mathbf{r}_i \cdot \mathbf{r}_j$. 
    Обратная метрическая матрица $ g^{ij} $ в таком случае находится из соотношения $g^{ij} g_{jk} = \delta^i_k$.

  \item Векторы $\mathbf{r}^i = g^{ij} \mathbf{r}_j$ называют \emph{векторами взаимного
    базиса}.
    Утверждение: $\mathbf{r}^i \cdot \mathbf{r}_j = \delta^i_j$
    Доказательство: 
\end{enumerate}

