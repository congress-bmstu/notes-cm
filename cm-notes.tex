\documentclass[12pt, oneside]{book}

\usepackage[T2A]{fontenc}			% кодировка
\usepackage[utf8]{inputenc}			% кодировка исходного текста
\usepackage[english,russian]{babel}	% локализация и переносы

\usepackage{tensor}
\usepackage{amsmath}
\usepackage{amsthm, mathrsfs, mathtools, amssymb}
\usepackage{enumitem}

\usepackage{epigraph}

\usepackage{clrscode}[const] %это я (сеня) добавил
\usepackage{pdfpages}

\usepackage{geometry}
\geometry{verbose,a4paper,tmargin=2cm,bmargin=2cm,lmargin=2.5cm,rmargin=1.5cm}

\usepackage{svg}
\svgpath{{img/}}
\usepackage{wrapfig}
\usepackage{caption}
\usepackage{subcaption}
\usepackage{float}

\usepackage{tikz}

\newcommand{\dimus}[1]{\begin{quotation} \emph{#1} \flushright--- Димитриенко Ю.\,И.
 \end{quotation}}
%\usepackage{showframe}

% \usepackage{accents}
% \newlength{\dhatheight}
% \newcommand{\hhat}[1]{%
%     \settoheight{\dhatheight}{\ensuremath{\hat{#1}}}%
%     \addtolength{\dhatheight}{-0.35ex}%
%     \hat{\vphantom{\rule{1pt}{\dhatheight}}%
%     \smash{\hat{#1}}}}
% \newcommand{\ubar}[1]{\underaccent{\bar}{#1}}

% \usepackage{bm}

\usepackage[colorlinks=true, linkcolor=black, urlcolor=black]{hyperref}
% \newtheoremstyle{example}% name
% {0.7cm}% Space above
% {0.7cm}% Space below
% {\small}% Body font
% {}% Indent amount
% {\small\scshape}% Theorem head font
% {.}% Punctuation after theorem head
% {.5em}% Space after theorem head
% {}% Theorem head spec (can be left empty, meaning ‘normal’)
%\theoremstyle{example}
%\newtheorem{ex}{Пример}
%\numberwithin{ex}{section}

\newtheoremstyle{example}% name
{0.7cm}% Space above
{0.7cm}% Space below
{\small}% Body font
{}% Indent amount
{\small\scshape}% Theorem head font
{.}% Punctuation after theorem head
{.5em}% Space after theorem head
{}% Theorem head spec (can be left empty, meaning ‘normal’)

\theoremstyle{example}
\newtheorem{example}{Пример}
\numberwithin{example}{section}

\theoremstyle{plain}
\newtheorem{theorem}{Теорема}
\newenvironment{thmbis}[1]
  {\renewcommand{\thetheorem}{\ref{#1}$'$}%
   \addtocounter{thm}{-1}%
   \begin{theorem}}
  {\end{theorem}}
\newtheorem{corollary}{Следствие}
\newtheorem*{corollary*}{Следствие}
\newtheorem{lemma}{Лемма}
\newtheorem{utv}{Утверждение}
\newtheorem*{utv*}{Утверждение}

% \newcounter{dfnbis}
\theoremstyle{definition}
\newtheorem{definition}{Определение}
% \newenvironment{dfnbis}[2][\thedfnbis]
% {\renewcommand{\thedefinition}{\ref{#2}\repeat{#1}{$'$} }%
%    \addtocounter{definition}{-1}%
%    \begin{definition}}
%   {\end{definition}}
% \newenvironment{dfnbis}[2][$'$]
% {\renewcommand{\thedefinition}{\ref{#2}#1}%
%    \addtocounter{definition}{-1}%
%    \begin{definition}}
%    {\end{definition}\renewcommand{\thedefinition}{\ref{#2}}}
% \newenvironment{dfnbis}[1]
% {\renewcommand{\thedefinition}{\ref{#1}$'$}%
%    \addtocounter{definition}{-1}%
%    \begin{definition}}
%    {\end{definition}}
\newenvironment{dfnbis}
{\addtocounter{definition}{-1}%
  \renewcommand{\thedefinition}{\arabic{definition}$'$}%
   \begin{definition}}
   {\end{definition}}
\newtheorem*{definition*}{Определение}
\newtheorem{question}{Вопрос}

\theoremstyle{remark}
\newtheorem{remark}{Замечание}
\newtheorem*{remark*}{Замечание}
\numberwithin{remark}{section}


% \newcommand{\toP}{\xrightarrow[]{\mathsf P}} % сходимость по вероятности
% \newcommand{\toPN}{\xrightarrow[]{\text{п.н.}}} % сходимость почти наверное
% \newcommand{\toD}{\xRightarrow[]{d}} % сходимость по распределению (слабая)
% \newcommand{\toR}{\xrightarrow[]{r}} % сходимость по распределению (слабая)
% \newcommand{\P}{\mathsf P} 
% \newcommand{\D}{\mathsf D} 
% \newcommand{\M}{\mathsf M} 

% \DeclareMathOperator{\cov}{cov}

% \newenvironment{solution}
% {
% 	\vspace{0.5em}
	
% 	% \noindent\textsc{Решение.}
% 	\textsc{Решение.}
% }
% {
%   \renewcommand{\qedsymbol}{$\square$}
% 	\qed
% }
% \renewcommand{\qedsymbol}{$\blacksquare$}

\begin{document}
  %\includepdf[scale=1.1]{img/dim.pdf}

  \tableofcontents

  \chapter{Модуль 1}

  % \section{Лекция 1 -- 2024-02-07 -- Механика сплошных сред}\label{sec:lec1}

% TODO исправить L_n на \mathcal{L}_n

Програмные комплексы:

\dots

План курса МСС:
\begin{enumerate}
  \item Введение:
    \begin{enumerate}
      \item Объекты и методы МСС;
      \item Основные задачи МСС;
    \end{enumerate}

  \item Элементы тензорного анализа;
  \item Основополагающие аксиомы МСС;
  \item Кинематика МСС;
  \item Законы сохранения МСС:
    \begin{enumerate}
      \item закон сохранения массы;
      \item закон изменения количества движения (закон сохранения импульса);
      \item закон изменения момента количества движения;
      \item первый закон термодинамики;
      \item второй закон термодинамики;
      \item нулевой закон термодинамики;
    \end{enumerate}
  \item Определяющие соотношения;
  \item Замкнутые системы уравнений МСС;
  \item Соотношения на поверхностях сильных разрывов;

    След сем:
  \item Основы механики деформируемого твердого тела (МДТТ);
  \item Основы механики жидкостей и газов (МЖГ).
\end{enumerate}

\section{Введение}

\subsection{Объекты МСС}

\paragraph{Что изучает МСС?}
Рассматривается физические (материальные) объекты. Выберем \underline{геометрический способ}
описания объектов:
\begin{itemize}
  \item $L$ --- характерный размер (длина) материального объекта (м);
    
    % TODO рисунок
    Объекты:
    \begin{itemize}
      \item естественного происхождения;
      \item искусственного происхождения;
    \end{itemize}

    Верхний предел применимости МСС где-то между $10^7$ и $10^9$ -- там нарушается первый принцип
    МСС, т.е. исчезает сплошность -- между планетками пустота. Однако после этой границы всё равно 
    возникает возможность применения МСС для релятивистких явлений (это такое обобщение МСС).

    Композиты -- особый вид твердых сред (материя).

    Итоги:
    \begin{enumerate}
      \item $L_{min} \leqslant L \leqslant L_{max}$ -- границы применимости МСС. При больших
        длинах -- астрофизика, при меньших -- физика микромира.

      \item Сплошность: $\exists$ самоподобных характерных объемов в которых $\exists$ <<много>>
        вещества.

      \item Основные области применения законов МСС:
        \begin{itemize}
          \item автомобилестроение;
          \item двигателестроение;
          \item авиастроение;
          \item ракетостроение;
          \item атомо техника;
          \item строительство;
          \item биомеханика; % TODO чото в скобочках (!!!!!!!!!!!!!)
          \item геомеханика (прогнозирование климата, тектоника и т.д.);
          \item композиты (прогнозирование свойств композитов)

            тервер говно, потому что использует грубые методы, не применяя информацию о внутренних
            законах исследуемого объекта. Поэтому он не применим в композитах -- углеродные волокна 
            валяются просто, есть что-то жидкое (связующее вещество) и оно не валяется и из этого 
            всего возникает крыло -- вот как это предказать.
        \end{itemize}
    \end{enumerate}
\end{itemize}

\subsection{Методы МСС}

Рассматриваются следующие науки: МСС, физика, химия, математика.

\begin{enumerate}
  \item По объектам, которые изучает МСС -- это просто \underline{часть физики}.

  \item Если изучается движение сред с химическими реакциями, то это раздел МСС -- механика
    многокомпонентых сред.

  \item Методы построения законов -- МСС - это раздел математики. Математика построена на аксиомах.
    МСС близка к математике, потому что она основана на аксиомах:
    \begin{itemize}
      \item аксиома сплошности;
      \item Евклидово пространство $\mathcal{E}_3^a$;
      \item $\exists$ абсолютное время.
    \end{itemize}
\end{enumerate}


\section{Элементы тензорного анализа}

Основные пространства в МСС:
\begin{enumerate}
  \item $L_n$ -- векторное (линейное пространство) пространство;
  \item $\mathcal{E}_n$ -- евклидово простраство;
  \item $\mathcal{E}_n^a$ -- точечно-евклидово простраство (афинное);
  \item $\chi$ -- метрическое пространство.
\end{enumerate}

\subsection{Векторное (линейное) пространство}

\begin{definition}
  $L_n$ -- множество, в котором введены 2 операции:
  \begin{itemize}
    \item сложение
    \item умножение на число
  \end{itemize}
\end{definition}

\begin{ex}
  Пространство элементарной геометрии $E_2$ и $E_3$.

  Множество геометрических объектов, состоящих из точек, прямых, плосткостей и из ... Геометрический
  вектор -- направленный отрезок.

  На множестве геометрических векторов введём отношение эквивалентности, такое что два вектора
  эквивалентны, если один получаются параллельным переносом. Тогда сложение векторов определим
  как сложение классов эквивалентости по правилу параллелограма. 

  $V_2$ -- пространство свободных векторов в $E_2$.
  $V_3$ -- пространство свободных векторов в $E_3$
\end{ex}

\begin{ex}
  Арифметическое пространство координатных столбцов $\mathbb{R}_n$.
  \[
    \begin{pmatrix}
      x_1 \\
      \vdots \\
      x_n
    \end{pmatrix} \in \mathbb{R}_n; \quad

    \begin{pmatrix}
      x_1 \\
      \vdots \\
      x_n
    \end{pmatrix} + \begin{pmatrix}
      y_1 \\
      \vdots \\
      y_n
    \end{pmatrix} = 
    \begin{pmatrix}
      x_1 + y_1 \\
      \vdots \\
      x_n + y_n
    \end{pmatrix}; \quad

    \lambda \begin{pmatrix}
      x_1 \\
      \vdots \\
      x_n
    \end{pmatrix} 
  \]
\end{ex}

В любом $L_n$ $\exists$ \underline{базис} (конечный), т.е. такая система векторов, что
\[
  \exists \vec{e_1}, \dots, \vec{e_n}: \forall \vec{a} \in L_n: \exists a^i :  \vec{a} = a^i \vec{e}_i;
\]
Не существует такого набора чисел $s^1, \dots, s^n$, такого что не все $s^i = 0$, так что
$s^i \vec{e}_i = 0$.

\begin{ex}
  Базис в $E_2$ ( $V_2$ ). 
\end{ex}

\paragraph{Замена базиса}

Рассмотрим базис $\vec{e}_i$ и $\vec{e}_i' = Q_i^j \vec{e}_j = Q^k_i \vec{e}_k$. $Q^j_i$ -- матрица 
замены базисов ($\det Q^j_i \neq 0$), к которой $\exists P^j_i$ -- обратная матрица, т.е. $P^i_j Q^j_k = \delta^i_k$; 

Преобразование компоненты вектора при замене базиса
\[
  \vec{a} = a^i \vec{e}_j = a^i' \vec{e}_j' = (a'^j Q^i_j) \vec{e}_i
\]

\subsection{Евклидово пространство}

\begin{definition}
  $\mathcal{L}_n$, в котором дополнительно введена операция скалярного умножения: $\varphi: \mathcal{E}_n^2 \to \mathbb{R}$. 
  Свойства:
  \begin{itemize}
    \item $\vec{a} \cdot \vec{b} = \vec{b} \cdot \vec{a}$;
    \item $(\vec{a}+\vec{b}) \cdot \vec{c} = \vec{a} \cdot \vec{c} + \vec{b} \cdot \vec{c}$;
    \item $\vec{a} \cdot \vec{a} \leqslant 0$.
  \end{itemize}
  
  Факты:
  \begin{itemize}
    \item $|\vec{a}| = \sqrt{ \vec{a} \cdot \vec{a}}$;
    \item $g_{ij} = \vec{e}_i \cdot \vec{e}_j$ -- метрическая матрица;
    \item $g^{ij} g_{jk} = \delta^i_k$;
    \item $\vec{e}^i = g^{ik} \vec{e}_k$ -- векторы взаимного базиса;
    \item $\vec{e}^i \cdot \vec{e}_j = (g^{ik} \vec{e}_k) \cdot \vec{e}_j = g^{ik} (\vec{e}_k \cdot \vec{e}_j) = g^{ik} g_{kj} = \delta^i_j$;
    \item Частный случай: если $\bar{\vec{e}}_i$ -- ортонормированный, т.е.
      $\vec{e}_i \cdot \vec{e}_j = \delta_{ij}$. Тогда $g_{ij} = \delta_{ij}$, тогда
      $g^{ij} g_{jk} = \delta^i_k$ и $g^{ij} = \delta^{ij}$. Тогда $\bar{\vec{e}}^i = \delta^{ik} \bar{\vec{e}}_k$
  \end{itemize}
\end{definition}

Векторное произведение в $\mathcal{E}_3$:
\[
  \vec{a} \times \vec{b} = \dfrac{1}{\sqrt{g}} \varepsilon^{ijk} a_i b_j \vec{e}_k
  = \sqrt{g} \varepsilon_{ijk} a^i b^j \skew{-7}\vec{e}^k.
\]

  % Лекция 2 -- 2024-02-14
\subsection{Тензоры в евклидовом провстранстве}

В евклидовом пространстве, где есть скалярное произведение, можно ввести, например, метрическую
матрицу:
\[
  g_{ij} = \vec{e}_i \cdot \vec{e}_j
\]

Также еще вот $\forall \vec{a} \in \mathcal{E}_n, \exists a^i :  \vec{a} = a^i \vec{e}_i$.

Еще вот есть тензоры. Например, градиент деформации:
\[
  F = \vec{r}_i \otimes \mathring{\vec{r}}^i,
\]

\begin{figure}[H]
	\centering
	\includesvg[scale = 0.8]{vecs1}
\end{figure}
где $\mathring{\vec{r}}_i \cdot \mathring{\vec{r}}^j = \delta_i^j$,
  $\otimes$ -- тензорное произведение.

Если совместить базисы (перенести к началу координат перенести), то получим объект, состоящий
из векторов: $\mathring{\vec{r}}_1$, $\mathring{\vec{r}}_2$, $\mathring{\vec{r}}_3$, $\vec{r}_1$,
$\vec{r}_2$, $\vec{r}_3$. Запишем эти вектора в таком порядке:
\[
  \vec{r}_1 \mathring{\vec{r}}_1 \vec{r}_2 \mathring{\vec{r}}_2 \vec{r}_3 \mathring{\vec{r}}_3 = \vec{r}_i \vec{r}^i,
\]
это -- \emph{векторный набор}. В каждой паре первый назовём \emph{левым} вектором, а
второй -- \emph{правым}. Сейчас мы этот векторный набор собрали из двух базисов, но вот потом
разберёмся как его построить по каким-то просто наборам векторов. 

\begin{figure}[H]
	\centering
	\includesvg[scale=0.8]{RandLvecs}
\end{figure}

Оказывается, что если взять такой векторный набор, то на нём можно построить все операции, которые
нас интересуют. Тогда $F = [\vec{r}_i \vec{r}^i]$ (суммирование не подразумевается).

Один из способов определить тензор (второго ранга):
\begin{definition}[тензора второго ранга]
  Тензор второго ранга -- это класс эквивалентости векторных наборов.
\end{definition}

\[
  \vec{r}_i = F \cdot \vec{r}_i^0.
\]

\begin{figure}[H]
	\centering
	\includesvg[scale=0.8]{blobs}
\end{figure}

\subsection{Наиболее основные подходы к определению тензоров (2 ранга)}

\begin{definition}
  Тензор -- такой опрератор линейного преобразования векторов, который действует таким образом:
  \[
    \vec{b}, \vec{a} \in \mathcal{E}_n : \vec{a} = T \cdot \vec{b}.
  \]
\end{definition}

\begin{definition}
  Тензор -- некоторый инвариантный объект, который в некотором базисе (диадном) $\vec{e}_i \otimes \vec{e}_j$ имеет компоненты $T^{ij}$ (матрица компонент).
  При переходе в другой базис $\vec{e}_i \to \vec{e}_i' = Q^j_{\, i} \vec{e}_j$ компоненты тензора
  будут преобразовываться по аналогии с компонентами вектора:
  \[
    \vec{a} = a^i \vec{e}_i = {a^i}' {\vec{e}_i}' = a^i P^k_{\,i} \vec{e}_k'
    \Rightarrow
    {a^k}' = P^k_{\, i} a^i.
  \]

  \[
    {T^{ij}}' = P^i_{\, k} P^j_{\, l} T^{kl}.
  \]

\end{definition}

Получается, тензор мы вводим с помощью некоторого порождающего пространства:
$\vec{a} \in \mathcal{L}_n, \vec{e}_i, \vec{e}_i' \in \mathcal{L}_n$.

Впринцепе определение правильное, но вызывает некоторые вопросы:
\begin{enumerate}
  \item что такое инвариантный объект? -- ну то, что он не зависит от базиса (так же как и векторы).

  \item где разложение тензора по базису? т.е. как найти его координаты хоть в каком-то базисе? 
    $T = T^{ij} e_i \otimes e_j$ -- здесь непонятно что это за базисные диады.

  \item порождающее пространство является линейным. Должны прийти к тому, что тензор является
    линейным оператором. таким образом, оба определения являются трактованием части свойств
    
    \begin{figure}[H]
    	\centering
    	\includesvg[scale=0.8]{squares}
    \end{figure}

  \item нет самой конструкции тензора -- вот в геометрическом определении выше четко алгоритм 
    построения приведён.
\end{enumerate}

Вот у этих дядей тензоры строятся аналогично нашему построению:
\begin{itemize}
  \item Ефимов, Розендорнд
  \item Победря Б.Е.
  \item Тарлаковский Д.В.
\end{itemize}
однако, векторные наборы у нас вполне конкретной длины, а у них он произвольной длины.

\begin{definition}[геометрическое определение]
  \begin{enumerate}
    
    \item $\mathcal{L}_n$ -- линейное (векторное) пространство -- \emph{порождающее пространство}.
      Выберем 2 системы векторов в $\mathcal{L}$: $\vec{a}_i, \vec{b}^{[i]}, i = \overline{1, n}$.
      (квадратные скобки -- просто обозначение).

    \item Построим формальный векторный набор из $\vec{a}_i, \vec{b}^{[j]}$ длины $2n$:
      Вектора из $\vec{a}_i$ называем левыми, а из $\vec{b}^{[i]}$ -- правыми: 
      $A \equiv \vec{a}_1 \vec{b}^{[1]} \vec{a}_2 \vec{b}^{[2]} \dots \vec{a}_n \vec{b}^{[n]}
      \equiv \vec{a}_i \vec{b}^{[i]}$.
      Здесь у нас n пар векторов (и в каждой паре есть левый $\vec{a}_i$ и правый вектор
      $\vec{b}^{[j]}$).

    \item Введем операции с векторными наборами:
      \begin{enumerate}
        \item сложение однотипных векторных наборов. Однотипными будем называть такие наборы, у 
          которых совпадают хотя бы либо все левые либо все правые векторы:
          \[
            A_1 \equiv \vec{a}_i \vec{b}^{[i]} \leftrightarrow A_2 = \vec{a}_i \vec{c}^{[i]}; 
            \quad
            A_1 \leftrightarrow A_3 = \vec{d}_i \vec{b}^{[i]}.
          \]
          тогда:
          \[
            A_1+A_2 = \vec{a}_i (\vec{b}^{[i]} + \vec{c}^{[i]}); \quad
            A_1+A_3 = (\vec{a}_i + \vec{d}_i) \vec{b}^{[i]}.
          \]
          Это частичная операция -- то есть такая, которая определена только на подмножестве всех.

        \item Умножение на число $s \in \mathbb{R}$.
          \[
            sA = (s \vec{a}_i) \vec{b}^{[i]} = \vec{a}_i (s \vec{b}^{[i]}).
          \]
        
        \item \emph{Эквивалентность} векторных наборов:
          Векторные наборы $A$ и $B$ называются эквивалентными, если выполняется хотя бы одно 
          из следующих условий:
          \begin{enumerate}
            \item векторные наборы $A$ и $B$ состоят из одних и тех же пар, но упорядоченных
              произвольным образом.
              
              Например,
                $A = a_1 b^{[1]} a_2 b^{[2]}, \, B = a_2 b^{[2]} a_1 b^{[1]}$ тогда $A \sim B$.

            \item Набор $B$ (А) может быть получен из другого набора с помощью согласованной операции
              умножения левых и правых векторов:
              \[
                A = \vec{a}_i \vec{b}^{[i]} \sim B = (s \vec{a}_i) (\dfrac{1}{s} \vec{b}^{[i]})
                \forall s \in \mathbb{R}, s \neq 0.
              \]

            \item если в A и B все векторы $\vec{a}_i$ и $\vec{b}^{[i]}$ совпадают, кроме тех пар,
              у которых хотя бы один вектор нулевой.
              Например,
              $A=\vec{a}_1 \vec{b}^{[1]} (\vec{a}_i \vec{0}) \sim B = \vec{a}_1 \vec{b}^{[1]} (\vec{0} \vec{b}^{[2]}) \sim \vec{a}_1 \vec{b}^{[1]} (\vec{c}_{2} \vec{0})$.
          \end{enumerate}

        \item Пусть теперь есть некоторый векторный набор $A = \vec{a}_i \vec{b}^{[i]}$ тогда введем 
          множество всех векторных наборов $B = \vec{c}_i \vec{d}^{[i]}$, эквивалентных $A$ и
          обозначим его $T = [A] = [\vec{a}_i \vec{b}^{[i]}]$. Таким образом определён
          \emph{тензор 2-го ранга}.
    \end{enumerate}

  \item \emph{Диады и базисные диады}. Пусть $A = \vec{a}_i \vec{b}^{[i]}$. И 
    $\exists$ не более чем 1 пара ненулевых векторов $\vec{a}_i \vec{b}^{[i]}$:
    \[
      [\vec{a}_1 \vec{b}^{[1]} \vec{0} \vec{0} \dots \vec{0} \vec{0}] = \vec{a}_1 \otimes \vec{b}^{[1]}; \quad
      [\vec{0} \vec{0} \vec{a}_2 \vec{b}^{[2]} \dots \vec{0} \vec{0}] = \vec{a}_2 \otimes \vec{b}^{[2]}; \quad
      [\vec{0} \vec{0} \vec{0} \vec{0} \dots \vec{a}_1 \vec{b}^{[1]}] = \vec{a}_n \otimes \vec{b}^{[n]};
    \]

    То есть мы научились по любой паре векторов конструировать диаду. Пусть $\vec{e}_i$ и
    $\vec{h}_j$ -- базисы в $\mathcal{L}_n$. Первый выберем в качестве левых векторов, а
    второй -- правых. Набор $[\vec{e}_1 \vec{0} \vec{e}_2 \vec{0} \dots \vec{e}_i \vec{h}_j \dots \vec{e}_n \vec{0}] = \vec{e}_i \otimes \vec{h}_j$ -- базисная диада. (вместо всех $\vec{e}_k, k\neq i$ ожно было поставить нули). 

  \item диадный базис. % TODO чото тут // Вроде дописал (Сеня)
  
  	\begin{equation*}
  		\vec{h}_j = \vec{e}_j \longrightarrow \vec{e}_i \otimes \vec{e}_j
  	\end{equation*}
  
    \begin{theorem}
      $\forall T = [\vec{a}_i \vec{b}^{[i]}]$ можно представить в виде: 
      \begin{equation}\label{lec_2:eq:tensor_basis}
        T = T^{ij} \vec{e}_i \otimes \vec{e}_j,
      \end{equation}
      линейная комбинация базисных диад.

      Причем:
      \[
        T = (T^{ij} \vec{e}_i) \otimes \vec{e}_j
        = [\vec{a}^{[j]} \vec{e}_j]
        = [\vec{e}_i \vec{b}^{[j]}]
      \]
    \end{theorem}
    \begin{theorem}[Следствие]
      Множество всех тензоров образует линейное пространство, где сложение векторов:
      \begin{multline*}
        T_1 = [\vec{a}_i \vec{b}^{[i]}],
        T_2 = [\vec{c}_j \vec{d}^{[j]}]; \quad
        \vec{a}_i = a^j_i \vec{e}_j,
        \vec{b}^{[i]} = b^{ik} \vec{e}_k,
        \vec{c}_i = c^{j}_{i} \vec{e}_j, 
        \vec{d}^{[i]} = d^{ik} \vec{e}_k, \quad \\
        T_1 = [(a^i \vec{e}_i) (b^{[ik]} \vec{e}_k)] = a^j_i 
        T_1 + T_2 = (a^j_i b^{ik} + c^{j}_i d^{jk}_i) \vec{e}_j \otimes \vec{e}_k
        = (T_1^{ij} + T_2^{ij}) \vec{e}_i \otimes \vec{e}_j.
      \end{multline*}
      А базисные диады образуют базис в тензорном пространстве.
    \end{theorem}
  \end{enumerate}
\end{definition}

Далее можно работать только с формулой \eqref{lec_2:eq:tensor_basis}.

Введём скалярное умножение тензоров. Для этого рассматриваем не $\mathcal{L}_n$ в качестве
порождающего, а $\mathcal{E}_n$.
\begin{align*}
  T \cdot \vec{a} &= (T^{ij} \vec{e}_i \otimes \vec{e}_j) \cdot (a^k \vec{e}_k)
  = T^{ij} a^k \vec{e}_i \otimes (\vec{e}_j \cdot \vec{e}_k)
  = T^{ij} a^k \vec{e}_i \otimes g_{jk} = \\
  &\text{Соглашение: тензорное умножение между тензором и числом опускаем} \\
  &= T^{ij} a^k g_{jk} \vec{e}_i
\end{align*}

\[
  T \cdot B = (T^{ij} \vec{e}_i \otimes \vec{e}_j) \cdot (B^{kl} \vec{e}_k \otimes \vec{e}_l)
  = T^{ij} B^{kl} \vec{e}_i \otimes (\vec{e}_j \cdot \vec{e}_k) \otimes \vec{e}_l
  = T^{ij} B^{kl} \vec{e}_i \otimes g_{jk} \otimes \vec{e}_l
  = T^{ij} B^{kl} g_{jk} \vec{e}_i \otimes \vec{e}_l
\]
-- тоже тензор 2-го ранга.

Двойное скалярное произведение:
\[
  T \cdot \cdot B = T^{ij} g_{jk} B^{kl} g_{il}
\]

\subsection{Точечное евклидово пространство}

\begin{definition}
  Точечным евклидовым пространством $\mathcal{E}_n^a$ (аффинным) называют пространство, в котором
  введены два типа объектов:
  \begin{enumerate}
    \item векторы $\vec{a}, \vec{b}, \vec{c} \in \mathcal{E}_n$;
    \item точки $A, B, C \in \mathcal{E}_n^a$.
  \end{enumerate}
  которые удовлетворяют следующим аксиомам:
  \begin{enumerate}
    \item $\forall A, B \in \mathcal{E}_n^a \, \exists! \vec{x} \in \mathcal{E}_n : \vec{x} = \vec{AB} \in \mathcal{E}_n^a$;
    \item $\forall A, B \in \mathcal{E}_n^a \, \exists \vec{0} \in \mathcal{E}_n : \vec{AB} + \vec{BC} + \vec{CA} = \vec{0}$ (равенство Шаля).
  \end{enumerate}
  Это формализация векторов и точек из пространств элементарной геометрии.
\end{definition}

В обычном евклидовом пространстве вектора это классы эквивалентности, а здесь хорошо получается 
что есть точечки и можно к ним вектор присоединить.

Если есть точка $A \in \mathcal{E}_n^a$ и вектор $\vec{x} \in \mathcal{E}_n$ то $\exists! B \in \mathcal{E}_n^a : \vec{AB} = \vec{x}$.

\paragraph{Некоторые свойства этого пространства}
\begin{enumerate}
  \item Рассмотрим базис $\vec{e}_i \in \mathcal{E}_n$ и точку $O \in \mathcal{E}_n^a$:
    $\forall M \in \mathcal{E}_n^a \, \exists! \vec{x} \in \mathcal{E}_n : \vec{OM} = \vec{x}$.
    Такое соотвествие $M \to \vec{x}$ будем называть \emph{радиус-вектором} относительно системы координат $O\vec{e}_i$.
    \begin{definition}
      Система координат в точечно-евклидовом пространстве: это точка $O$ и любой присоединенный к ней
      базис $\vec{e}_i$. Обозначается $O\vec{e}_i$.
    \end{definition}
   
  \item Рассмотрим СК $O\vec{e}_i$ и тогда для любой точки $M$ существует радиус-вектор $\vec{OM} = \vec{x} \in \mathcal{E}_n$.
    Тогда мы можем разложить $\vec{x}$ по базису $\vec{e}_i$: $\vec{x} = x^i \vec{e}_i$.
    % TODO картинка базис чото там еще // Это??? (Сеня)
    
    \begin{figure}[H]
    	\centering
    	\includesvg[scale=0.6]{vecX}
    \end{figure}

  \item Длина вектора, соединяющего точки $A, B \in \mathcal{E}_n^a$. $\vec{AB} = \vec{a}$.
    \[
      l(A, B) = |\vec{AB}| = |\vec{a}| = \sqrt{\vec{a}^i \cdot \vec{a}_i} = \sqrt{a^i a^j g_{ij}}
    \]
    
  	\begin{figure}[H]
  		\centering
  		\includesvg[scale=0.6]{vecX2}
  	\end{figure}

  \item Пространство, в котором существует понятие длины (расстояния между точками A B), но в
    общем случае нет скалярного произведения) называентся метрическим пространсвтом.
    В точечно-евклидовом простарнство тоже можно ввести понятие расстояния между точками A B:
    \[
      l(A, B) = |\vec{AB}| = \sqrt{\vec{a} \cdot \vec{a}}
    \]
    поэтому точечно-евклидово пространство называют метризованным.

  \item Криволинейные координаты в $\mathcal{E}_n^a$. Рассмотрим СК $O\vec{e}_i$ и точку $M$ с
    координатами $x^i$ в этой СК -- эти координаты будем называть декартовыми (не обязательно
    ортонормированный базис).
    
    \begin{figure}[H]
    	\centering
    	\includesvg[scale=0.6]{curvedcoord}
    \end{figure}
    
    Рассмотрим теперь функции многих переменных: $X^i = X^i (x^j)$, где в некоторой области
    $V \subset \mathcal{E}_n^a$. Поскольку $\mathcal{E}_n^a$ метризовано, можно ввести понятие 
    области -- открытого множества (для любой точки существует ее окрестность, полностью
    принадлежащая области). $\varepsilon$-окрестностью точки будем называть множество:
    $U_\varepsilon(A) = \left\{ M \in \mathcal{E}_n^a | l(A, M) < \varepsilon \right\} $.
    
    % TODO: Щас ток подумал, что возможно эту и другие более мелкие картинки стоит не по центру, а как-нибудь вокруг текста чтоб обтекало, чтонее чтобы текст вокруг. Остальные картинки ЗАВТРА!!
    \begin{figure}[H]
    	\centering
    	\includesvg[scale=0.6]{epsilon-area}
    \end{figure}
    
    Если заданы функции вида $X^i = X^i (x^j)$, которые
    \begin{enumerate}
      \item являются гладкими в области $V$;
      \item являются невырожденными в этой же области:
        \[
          \left| \dfrac{\partial X^i}{\partial x^j} \right| \neq 0
        \]
    \end{enumerate}
    то говорим что задана криволинейная система координат.
  
    Например, если $\bar{\vec{e}}_i$ -- ортонормированный базис, $O\bar{\vec{e}}_i$ -- прямоугольная
    декартова система координат. Через любую точку $M \in V$ можно провести 3 координатные линии
    $X^\alpha = var, X^{\beta} = \const, X^\gamma = \const, \alpha \neq \beta \neq \gamma, \alpha, \beta, \gamma = 1, 2, 3$.

    Пример: цилиндрическая система координат. 
    % TODO рисунок.
    \[
      \begin{cases}
        X^1 = r, \\
        X^2 = \varphi, \\
        X^3 = z = x^3
      \end{cases}
    \]

    $X^1 = var$ -- луч
    $X^2 = var$ -- окружность
    $X^3 = var$ -- прямая

  \item локальные векторы базиса. У нас есть радиус-вектор точки $M$: $\vec{x} = x^i \vec{e}_i$. 
    Из криволинейных координат и условий, которые мы наложили на них, существует обратные функции:
    $x^i = x^i(X^j) \in V_x \subset \mathcal{E}_n^a$ -- тоже гладкая и невырожденная. Тогда 
    радиус-вектор точки $M$: $\vec{OM} = \vec{x} = x^i(X^k) \bar{\vec{e}}_k$. Дифференцированием
    получим:
    \[
      \dfrac{\partial \vec{x}}{\partial X^i} = \dfrac{\partial x^i (X^k)}{\partial X^i} \bar{\vec{e}}_i
      \equiv \vec{r}_i
    \]
    Следовательно, определили локальные векторы базиса: 
    $\vec{r}_i \equiv \dfrac{\partial \vec{x}}{\partial X^i} \bar{\vec{e}}_i
    = Q^j_{\, i} \bar{\vec{e}}_i$. Матрица $Q$ называется якобиевой.

    Так как криволиненйные координаты невырожденны ($\det Q \neq 0$) $\vec{r}_i$ образует 
    базис. Эти вектора направлены по касательным к соответствующим координатным линиям.

    % TODO кусок ниже непонятно куда вставить -- он хотел это отдельно раньше, но видимо забыл 
    % и дал только в этот момент
    в точечно-евклидовом простарснвте существует понятие единого ортонорвированного базиса такого,
    что $\forall M \in \mathcal{E}_n^a \, \exists! \vec{OM} = \vec{x} = x^i \bar{\vec{e}}_i$.
    Пример неевклидова двумерного пространства -- поверхность цилиндра.
    % TODO рисунок
    % здесь этот кусок закончен

  \item Метрическая матрица для локальных векторов базиса: $g_{ij} = \vec{r}_i \cdot \vec{r}_j$. 
    Обратная метрическая матрица $g^{ij} g_{jk} = \delta^i_k$.

  \item $\vec{r}^i = g^{ij} \vec{r}_j$ -- вектора взаимного базиса.
    Утверждение: $\vec{r}^i \cdot \vec{r}_j = \delta^i_j$
    Доказательство: 
\end{enumerate}


  \section{Основополагающие аксиомы МСС}
% Лекция 3 -- 2024-02-21

\subsection{Определение сплошной среды}

\begin{figure}[H]
	\centering
	\includesvg[scale=0.8]{mss_in_a_nutshell}
\end{figure}

МСС изучает <<не очень маленькие>> и <<не очень большие>> объекты -- тела, которые состоят
из материальных точек.

\begin{figure}[h!]
  \centering
  \includesvg[scale=1]{continuum}
  \caption{В любом бесконечно малом объеме всегда много вещества}
\end{figure}

\begin{definition}
  Сплошной средой называется тело B, для которого введено взаимно-однозначное соотвествие
  или отображение в некоторое метрическое простанство $\chi$.
\end{definition}

\begin{figure}[H]
	\centering
	\includesvg[scale=1]{twoplanes}
\end{figure}

\begin{definition}
  Множество всех тел $B$ называется \emph{вселенной}.
\end{definition}

\paragraph{Аксиома 1. (аксиома сплошности)} Образ $W(B)$ всякой сплошной среды образует континуальное
множество в $\chi$. Открытое континуальное множество = \textbf{область} $V$ в $\chi$.

\begin{figure}[H]
	\centering
	\includesvg[scale=1.1]{continuumaxiom}
\end{figure}

\[
  \forall M \in W(B) : \exists \mathring{U}_\delta (M) \subset W(B),
\]
где $\mathring{U}_\delta$ -- проколотая окрестность.

\paragraph{Аксиома 2. (евклидовость пространства)} В качестве метрического пространства $\chi$,
в котором рассматривается все тела $B \subset$ вселенной можно выбрать трёхмерное 
точечно-евклидово пространство $\mathcal{E}^a_3 = \chi$.

В $\mathcal{E}_3^a$ существует много различных <<конструкций>>: точки (= материальные точки
$M \in W(B)$), вектора ($M_1, M_2 \in \mathcal{E}_3$), единая декартова система координат
($O\vec{e}_1 : \forall M \in W(B)=V : \exists \vec{x} \text{-- радиус-вектор } :
\vec{x} = x^i \bar{\vec{e}}_i$).

\begin{figure}[H]
	\centering
	\includesvg[scale=1]{euclidspace}
\end{figure}

Контрпример, т.е. случай, когда не бывает единой декартовой СК, встретился в Примере \ref{cilinder_surface}.

\paragraph{Аксиома 3. (существование абсолютного времени)} Для всякого тела $B$ и для всякого
параметра $t \in \mathbb{R}_{+0}$ $\exists V(t) \in \mathcal{E}_3^a$, где $\mathbb{R}_{+0}$ --
неотрицательные вещественные числа.

\begin{figure}[H]
	\centering
	\includesvg[scale=1.1]{planes}
\end{figure}

Есть самолёт, зависящий от какого-то параметра и при каких-то разных значениях этого параметра 
образы этого самолёта будут разными. И в разные моменты времени одна и та же материальная точка
будет иметь разные радиус-вектора.

Абсолютизм времени состоит в том, что при движении с большими скоростями возникают релятивистские 
эффекты, но в классической механике сплошных сред $|\vec{v}| << C$, чтобы время текло одинаково
для всех тел.

\begin{definition}
  \emph{Движение} тел и материальных точек -- это изменение радиус-векторов материальных точек в
  единой декартовой системе координат $O \bar{\vec{e}}_1$.
\end{definition}


\subsection{Кинематика сплошных сред}

\begin{definition}
  Кинематика сплошных сред -- это раздел, в котором изучается изучается движение тел без анализа
  причин, которые вызывают это движение.
\end{definition}

\paragraph{Лагранжевы и Эйлеровы координаты}

Рассмотрим произвольную сплошную среду в некоторый момент времени $t_1 \geqslant 0$ и в некоторый
другой момент времени $t_2 \geqslant 0$.

\begin{figure}[H]
	\centering
	\includesvg[scale=1.1]{lagrang_and_eulier}
\end{figure}

Эта сплошная среда $B$ состоит из одних и тех же материальных точек $M$ во всех рассматриваемых 
моментах времени. \footnote{имеется ввиду, что мы рассматриваем такой класс тел, то есть существуют
и другие}
  
Благодаря евклидовости пространства, всегда существует единая декартова система координат. В
ней мы можем следить за радиус-вектором какой-либо материальной точки.

Рассмотрим положение тела в $O\bar{\vec{e}}_i$ в некоторый фиксированный момент времени, 
в котором удобно выбрать $t=0$.

\begin{figure}[H]
	\centering
	\includesvg[scale=1.1]{lagrang_and_eulier2}
\end{figure}

Положение тела $\mathring{V}$ в $O\bar{\vec{e}}_i$ при $t=0$ называется \emph{отсчетной}
(лагранжевой) конфигурацией $\mathring{\mathcal{K}}$.

Положение того же самого тела $V$ в момент времени $t > 0$ называется актуальной (эйлеровой)
конфигурацией $\mathcal{K}$.

Укажем далее способ <<паспортизации>> всех материальных точек. В момент времени $t = 0 :
\vec{x}(M) = \mathring{\vec{x}} = \mathring{x}^{i} \bar{\vec{e}}_i$, т.е. $\mathring{x}^{i}$
-- декартовы коодинаты материальной
точки $M$ при $t=0$. Введём криволинейные координаты матириальной точки при $t=0$:
$X^i = X^i(\mathring{x}^j) \Leftrightarrow \mathring{V} \to \mathring{V}_x$. Из аксиомы 3 следует, что для той же 
материальной точки $M$ сплошной среды $B$ в моменты времени $t > 0$ соотвествует $\vec{x}(t) = x^i \bar{\vec{e}}_i$.

Введём такие криволинейные координаты $X^i$ материальных точек, которые введены в
$\mathring{\mathcal{K}}$, а дальше \textbf{движутся вместе с материальными точками}. В частном
случае эти криволинейные координаты могут совпадать с обычными координатами $X^i = \mathring{x}^i$.

\begin{figure}[H]
	\centering
	\includesvg[scale=0.8]{balka}
\end{figure}


Координаты движутся вместе с материальными точками означает, что для любого момента времени $t$ 
все материальные точки $M$ тела имеют одни и те же координаты $X^i$. Такие координаты $X^i$ называют
\emph{лагранжевыми координатами} материальной точки $M$.

Введённые лагранжевы координаты $X^i$ -- это способ <<паспортизации>> материальных точек (то есть
способ их различения).

Полагая, что $X^i(x^i)$ -- это гладкие функции, определённые в области $\mathring{V}$ и невырождены:
\[
  \forall \mathring{x} \in \mathring V : \det \left( \dfrac{\partial X^i}{\partial x^j}  \right) \neq 0.
\]

Тогда из теоремы об обратной функции из курса ТФНП существует обратная функция $\mathring{x}^i
= \mathring{x}^i (X^j), \forall X^j \in V_x$. Причём
\[
  \det \left( \dfrac{\partial \mathring{x}^i}{\partial X^{j}}  \right)
  = \dfrac{1}{\det \left( \dfrac{\partial X^i}{\partial \mathring{x}^i}  \right) } \neq 0.
\]
Тогда $\mathring{\vec{x}} = \mathring{x} \bar{\vec{e}}_i = \mathring{x}^i (X^j) \bar{\vec{e}}_i$.
($\bar{\vec{e}}_i$ -- не зависит от системы координат). Тогда получается, что
$\mathring{\vec{x}} = \mathring{\vec{x}} ( X^i) = \mathring{x}^i \bar{\vec{e}}_i$.

Рассмотрим теперь момент времени $t = 0$: $\vec{x} = \vec{x}(X^i, t)$ -- из введения лагранжевых 
координат и способа паспортизации материальных точек. Если зафиксировать $t$ (но берём разные
криволинейные координаты), то получим множество радиус-векторов:

\begin{figure}[H]
	\centering
	\includesvg[scale=0.7]{hedg}
\end{figure}

Если зафиксируем криволинейные координаты, а $t$ будем менять, то получим параметрическое
представление кривой $\vec{x} = \vec{x} ( X^i, t )$ -- траектория материальной точки (с координатами
$X^i$). Причём траектории не пересекаются в один и тот же момент времени. Уравнение $\vec{x} = \vec{x}(X^i, t)$ называется законом движения материальных точек.

Если бы был известен закон движения всех материальных точек, то задача кинематики была бы решена.
$x^{i}$, $\mathring{x}^i$ -- эйлеровы координаты материальной точки $M$. Из уравнений каких-то
следует, что $\vec{x} = \vec{x} (X^i, t) = x^j ( X^i, t) \bar{\vec{e}}_j$. $x^j = x^j(X^j, t)$
-- связь эйлеровых координат с лагранжевыми.

\subsection{Материальное и пространственное описание движения тела}

В МСС состояние тел описывается неокоторыми:
\begin{itemize}
  \item скалярными величинами: $\theta, \rho, \dots$;
  \item векторными величинами: $\vec{x}, \vec{v}, \vec{a}, \dots$;
  \item тензорными величинами: $E, F, T, \dots$.
\end{itemize}

Эти тензоры являются функциями от $X^i$ и $t$:
\begin{itemize}
  \item $\vec{x} = \vec{x} (X^i, t)$;
  \item $\vec{b} = \vec{b} (X^i, t) \forall \vec{b} \in \mathcal{E}_3$;
  \item $\xi = \xi(X^i, t) \forall $ скаляра;
  \item $A = A(X^i, t) \forall$ тензора.
\end{itemize}
Везде $t \in \mathbb{R}_{+0}$. Эти все штуки называются \emph{полями} скаляров, тензоров, векторов.

Существуют обратные $X^i = X^i ( x^j, t )$, если они гладкие и невырожденные: $\det \dfrac{\partial X^i}{\partial x^j} \neq 0 \forall x^i \in V \times [0, +\infty)$.

Тогда любое поле: $\xi = \xi(X^i, t) = \xi(X^i(x^j, t), t) = \tilde{\xi} (x^j, t)$ (аналогично и с
векторами и с тензорами). В силу взаимной однозначности функции соответствия между эйлеровыми и 
декартовыми координатами, всегда можно использовать запись полей, которые описывают состояние
тела, либо через $(X^i, t)$ (если используется такое описание, то говорят, что используется
материальное (Лагранжево) описание), либо через $(x^i, t)$ (если используется это описание, то 
говорят, что применяется пространственное (Эйлерово) описание).

Рассмотрим некоторые примеры, когда удобней использовать какое-либо описание

\begin{example} 
  Для твёрдых тел чаще используется Лагранжево описание движения тела.
  
  \begin{figure}[H]
  	\centering
  	\includesvg[scale=0.8]{drummer}
  \end{figure}
  
  При использовании Лагранжевого описания, мы точно знаем область, в которой нам нужны тензоры
  всякие, потому что область совпадает с изначальной.

  При использовании Эйлерового описания, область неизвестна и основная задача будет в её
  нахождении.
\end{example}

\begin{example}
  Газовая динамика.
  
  \begin{figure}[H]
  	\centering
  	\includesvg[scale=0.8]{engine}
  \end{figure}
  
  Область, в которой ищуться характеристики движения, постоянна, потому что нас не интересуют
  области, в которых жидкость была до данного момента, и в которые она попадёт после интересующего 
  момента, поэтому рассматриваем

  Нам не нужно находить радиус-вектора точек (хотя мы бы могли), нам нужно находить скалярные поля
  в интересующей области -- плотность, давление и т.д. Уравнения газовой динамики позволяют разделить
  задачи нахождения радиус-векторов и этих полей (в твёрдых телах такого не мы сделать не можем).
\end{example}

\subsection{Локальные базисы в $\mathring{\mathcal{K}}$ и $\mathcal{K}$}

Рассмотрим уравнение движения материальных точек $\vec{x} = \vec{x} (X^i, t)$ -- оно всегда
сущестует, пусть даже мы его не находим. Предполагаем, что эти функции гладкие, продифференцируем
их по $X^j$:
\[
  \vec{r}_i = \dfrac{\partial \vec{x}}{\partial X^i}; \quad
  \mathring{\vec{r}}_i = \dfrac{\partial \mathring{\vec{x}}}{\partial X^i},
\]
локальные базисы в $\mathcal{K}$ и $\mathring{\mathcal{K}}$ (то, что это базис, следует из
невырожденности).

\begin{figure}[H]
	\centering
	\includesvg{localbasis}
\end{figure}

$\vec{r}_i$ и $\mathring{\vec{r}}_i$ направлены по касательным к Лагранжевым координатам $X^i$. 
Локальные базисы $\vec{r}_i$ движуться вместе с материальными точками. 
\[
  \vec{r}_i
  = \dfrac{\partial \vec{x}}{\partial X^i}
  = \dfrac{\partial x^j \bar{\vec{e}}_j}{\partial X^i}  
  = \tensor{Q}{^j_i} \bar{\vec{e}}_j,
\]
где $\tensor{Q}{^j_i}$ -- якобиевая матрица.

\[
  \mathring{\vec{r}}_i = \dfrac{\partial \mathring{\vec{x}}}{\partial X^i}
  = \dfrac{\partial x^{0j}}{\partial X^i} \bar{\vec{e}}_j
  = \tensor{\mathring{Q}}{^j_i} \bar{\vec{e}}_j
\]

Так как предполагаются невырожденными, то существуют обратные матрицы:
\[
  \tensor{P}{^j_i} \tensor{Q}{^i_k} = \delta^j_{\, k}; \quad
  \tensor{\mathring{P}}{^j_i} \tensor{\mathring{Q}}{^i_k} = \delta^j_{\, k}
\]

$P$ -- обратная якобиевая матрица.

\[
  g_{ij} = \vec{r}_i \cdot \vec{r}_j; 
  \quad
  \mathring{g}_{ij} = \mathring{\vec{r}}_i \cdot \mathring{\vec{r}}_j;
\]
-- метрические матрицы отсчетного и произвольного состояния (причём $\det g_{ij} \neq 0$, что
доказывается с помощью подстановки вместо $\vec{r}_i = \tensor{Q}{^j_i} \bar{\vec{e}}_j$, попутно
получаем еще одну формулу для метрической матрицы: 
$g_{ij} = \tensor{Q}{^k_i} \tensor{Q}{^l_j} \delta_{kl}$, $\det g_{ij} = (\det Q )^2$).

Обозначим также $\mathring{g} = \det \mathring{g}_{ij} \neq 0$ и $g = \det g_{ij} \neq 0$.
Тогда введём $\vec{r}^i  = g^{ij} \vec{r}_j, \mathring{\vec{r}}^i = \mathring{g}^{ij}
\mathring{\vec{r}}_j$ --
векторы взаимных базисов в $\mathcal{K}$ и $\mathring{\mathcal{K}}$.

Их свойства:
\[
  \vec{r}_i \cdot \vec{r}^j = \delta^j_i, \quad \mathring{\vec{r}}_i \cdot \mathring{\vec{r}}^j = \delta^j_i
\]
\begin{proof}
  \[
    \vec{r}_i \cdot (g^{jk} \vec{r}_k)
    = g^{jk} \vec{r}_i \cdot \vec{r}_k
    = g^{jk} \cdot g_{ik} = \delta_i^j.
  \]
\end{proof}

Следствие: $\vec{r}_i \times \vec{r}_j = \sqrt{g} \varepsilon_{ijk} \vec{r}^k$.

\begin{proof}
  $\vec{a} \times \vec{b} = \sqrt{g} \varepsilon_{ijk} a^i b^j \vec{r}^k$.
  Из $\vec{a} = a^i \vec{r}_i$ и $\vec{b} = b^j \vec{r}_i$, поэтому 
  \[
    \vec{r}_i \times \vec{r}_j
    % TODO дописать доказательство
  \]
\end{proof}

Аналогично всё с ноликовыми векторами.

Следствие еще одно:
\[
  \vec{r}_1 \cdot (\vec{r}_2 \times \vec{r}_3)
  = \vec{r}_1 \cdot \sqrt{g} \vec{r}^1
  = \sqrt{g} \cdot 1 = \sqrt{g}.
\]

Таким образом, $\sqrt{g} = \vec{r}_1 \cdot (\vec{r}_2 \times \vec{r}_3)$.


\subsection{Ковариантная производная в $\mathcal{K}$ и $\mathring{\mathcal{K}}$}

Введем набла оператор в $\mathcal{K}$ -- символический дифференциальный оператор:
\[
  \nabla \equiv \vec{r}^i \dfrac{\partial }{\partial X^i} 
  = \vec{r}^1 \dfrac{\partial }{\partial X^1} + \vec{r}^2 \dfrac{\partial }{\partial X^2} 
  + \vec{r}^3 \dfrac{\partial }{\partial X^3};
\]

Аналогично с наблей в $\mathring{\mathcal{K}}$.

Применение его:
\begin{enumerate}
  \item К скаляру: градиент скаляра -- вектор
    \[
      \nabla \varphi = \vec{r}^i \dfrac{\partial \varphi}{\partial X^i}.
    \]
    Из свойств отметим, что он инвариантен, то есть 
    \[
      \nabla \varphi = \bar{\vec{e}}^i \dfrac{\partial \varphi}{\partial x^i} 
      = \dfrac{\partial \varphi}{\partial x^1} \bar{\vec{e}}_1 + \dfrac{\partial \varphi}{\partial x^2} \bar{\vec{e}}_2 + \dfrac{\partial \varphi}{\partial x^3} \bar{\vec{e}}_3.
    \]

  \item К вектору:
    \begin{itemize}
      \item тензорно: градиент вектора -- тензор второго ранга:
        \[
          \nabla \otimes \vec{a}
          = \vec{r}^i \dfrac{\partial }{\partial X^i} \otimes \vec{a}
          = \vec{r}^i \otimes \dfrac{\partial \vec{a}}{\partial X^i} 
        \]
        Рассмотрим
        \[
          \dfrac{\partial \vec{a}}{\partial X^i} 
          = \dfrac{\partial a^j \vec{r}_j}{\partial X^i} 
          = \dfrac{\partial a^j}{\partial X^i} \vec{r}_j
            + a^j \dfrac{\partial \vec{r}_j}{\partial X^i} 
            = \left( \dfrac{\partial a^k}{\partial X^i} + a^j \Gamma^k_{ji} \right) \vec{r}_k,
        \]
        где введено обозначение:
        $ \dfrac{\partial \vec{r}_j}{\partial X^i} 
          = \Gamma^k_{ji} \vec{r}_k $
        -- символы Кристоффеля.

        Обозначим: $\nabla_i a^k \equiv \dfrac{\partial a^k}{\partial X^i} + a^j \Gamma^k_{ji}$
        -- ковариантная производная от контравариантных компонент вектора.
        Тогда $ \dfrac{\partial \vec{a}}{\partial X^i} = (\nabla_i a^k) \vec{r}_k$.

        Если $\vec{r}_i \equiv \bar{\vec{e}}_i$, то $\Gamma^k_{ji} \equiv 0$. То есть
        ковариантная производная совпадёт с частной производной.

        Подставим полученное в начало:
        \[
          \nabla \otimes \vec{a} = \vec{r}^i \otimes (\nabla_i a^k) \vec{r}_k 
          = (\nabla_i a^k) \vec{r}^i \otimes \vec{r}_k
        \]
        Компоненты этого тензора в смешанном диадном локальном базисе.

        \textbf{Теорема Риччи}: $\nabla_i g_{jk} \equiv 0$, из нее следует, что можно опускать и 
        поднимать индексы под знаком контравариантной производной.

        \[
          \nabla^i a_k \equiv g^{ij} \nabla_j a_k
        \]
        -- контравариантая производная от ковариантных компонент вектора.

\[
          \nabla \otimes \vec{a}
          = (\nabla_i a^k) \vec{r}^i \otimes \vec{r}_k
          = (\nabla^i a^k) \vec{r}_i \otimes \vec{r}_k
        \]

        Отметим, что $ \dfrac{\partial a_k}{\partial X^i} $ -- не являются компонентами какого-то
        тензора, но $\nabla_i a_k$, $\nabla^i a^k$ -- являются компонентами тензора 2-го ранга.

    \end{itemize}
\end{enumerate}



  % \section{Жопа жопа жопа}

% TODO ковариантные производные тензоров, ротор тензора и тд

\subsection{Градиент деформации}

Рассмотрим движение материальных точек сплошной среды.
% TODO картинка две амёбы в одну попала стрела и она умерла нахуй

\[
  \mathring{\vec{r}}_i = \dfrac{\partial \mathring{\vec{x}}}{\partial X^i};
  \quad 
  \vec{r}_i = \dfrac{\partial \vec{x}}{\partial X^i},
  \quad
  \vec{x} = \vec{x}(X^i, t).
\]

Нужен новый объект, с помощью которого можно было сравнивать конфигурации $\mathcal{K}$ и $\mathring{\mathcal{K}}$.

Введём тензор $E = \vec{r}_i \otimes \vec{r}^i
= \mathring{\vec{r}}_i \otimes \mathring{\vec{r}}^i
= g_{ij} \vec{r}^i \otimes \vec{r}^j = \mathring{g}_{ij} \mathring{\vec{r}}^i \otimes \mathring{\vec{r}}^j$

Возьмём два базиса из разных конфигураций: $F = \vec{r}_i \otimes \mathring{\vec{r}}^j$ -- градиент деформации.

Свойства этого тензора:
\begin{enumerate}
  \item Градиент деформации -- тензор преобразования векторов локального базиса $\mathring{\vec{r}}_i$ в $\vec{r}_i$ из $\mathring{\mathcal{K}}$ в $\mathcal{K}$:
    \[
      F \cdot \mathring{\vec{r}}_i
      = (\vec{r}_j \otimes \mathring{\vec{r}}^j) \cdot \vec{r}_i
      = \vec{r}_j \otimes (\mathring{\vec{r}}^j \cdot \mathring{\vec{r}}_i)
      = \vec{r}_j \otimes \delta^j_i = \vec{r}_i.
  \]

  \item Введем элементарный радиус-вектор в $\mathring{\mathcal{K}}$:
    $d\mathring{\vec{x}} = \vec{MM_1}$ -- вектор сдвига между двумя близкими точками.
    Тогда $\mathring{\vec{x}} = \mathring{\vec{x}} (X^i)$.
    Поскольку эта функция предполагается всегда гладкой в нужной нам области, то существует
    дифференциал: $d\mathring{\vec{x}} = \dfrac{\partial \mathring{\vec{x}}}{\partial X^i} dX^i
    = \mathring{\vec{r}}_i dX^i$ -- по сути разложение по базису, а его координатами является
    приращения локальных координат.

    % TODO рисунок разложения по базису

    Аналогично определим приращение в $\mathcal{K}$ при фиксированном $t$: 
    $d\vec{x} = \dfrac{\partial \vec{x}}{\partial X^i} dX^i = \vec{r}_i dX^i.$

    % TODO тот же самый рисунок только в другой конфигурации

    Одни и те же локальные координаты $\Leftrightarrow$ одна и та же материальная точка: $M = M_1$.

    Важно измерять расстояния в МСС, в домах можно поставить маячки, чтобы смотреть развивается ли
    трещина:
    % TODO самолёт и домик

    Рассмотрим теперь $F \cdot d\mathring{\vec{x}}$:
    \[
      F \cdot d\mathring{\vec{x}}
      = (\vec{r}_i \otimes \mathring{\vec{r}}^i) \cdot d\mathring{\vec{x}}
      = \vec{r}_i \otimes (\mathring{\vec{r}}^i \cdot \mathring{\vec{r}}_j dX^j)
      = \vec{r}_i \otimes \delta^i_j dX^j
      = \vec{r}_j dX^j = d\vec{X}
    \]
    Таким образом, получили, что $d\vec{X} = F \cdot d\mathring{\vec{x}}$.

    Оказывается, градиент деформации преобразует вектора из начальной конфигурации в настоящую.

    % TODO рисунок локальная окрестность точки M 

    Следовательно, $F$ является тензором линейного преобразования малой окрестности $d\mathring{V}$ в
    $dV$.

    Свойства линейного преобразования:
    \begin{enumerate}
      \item Если окрестность некоторой точки $M$ $d\mathring{V} = \text{куб}$. В силу свойств
        линейности, куб может перейти только в косоугольный параллелепипед. Можно доказать и
        строго, но мы ограничимся словами: уравнение плоскостей -- линейные, они переходят в
        линейные при линейном преобразовании.
        % TODO рисунок

      \item Если $d\mathring{V} = \text{шар}$, то $dV$ -- эллипсоид. Не решая никакой задачи, только
        из свойств непрерывных гладких преобразований мы очень много можем сказать о любых возможных
        состояниях, мы знаем чем они будут являться.
        % TODO рисунок
    \end{enumerate}

  \item $F = \vec{r}_i \otimes \mathring{\vec{r}}^i = [\vec{r}_i \, \mathring{r}^i]$ -- тензор, а
    что такое тензор? -- Это класс эквивалентности векторных наборов. Одно из условий попасть в этот
    класс эквивалентности -- можно умножить одну из троек векторов на число, а другую на неё поделить
    или более общо умножить на матрицу, а другую -- поделить.
    $F = \tensor{Q}{^j_i} \vec{r}_i \otimes P_{kj} \mathring{\vec{r}}^k = [\tilde \vec{r}_i \otimes
    \mathring{\tilde \vec{r}}^i]$.

  \item Детерминантом тензора называется детерминант матрицы компонент тензора в смешанном базисе:
    $\operatorname{det} A = \operatorname{det} (\tensor{A}{^i_j})$, $A = A^{ij} \vec{r}_i \otimes \vec{r}_j = A_{ij} \vec{r}^i \otimes \vecc{r}^j = \tensor{\mathring{A}}{^i_j} \mathring{\vec{r}}_i \otimes \mathring{\vec{r}}^j$.
    Посчитаем детерминант тензора градиента деформации:
    \[
      F
      = \vec{r}_i \otimes \mathring{\vec{r}}^i
      = \tensor{Q}{^j_i} \bar{\vec{e}} \otimes \tensor{\mathring{P}}{^i_k} \bar{\vec{e}},
      \quad \tensor{Q}{^j_i} \equiv \dfrac{\partial x^j}{\partial X^i},
      \quad \tensor{\mathring{P}}{^i_k} = \left( \dfrac{\partial \mathring{x}^i}{\partial X^j}  \right) = \dfrac{\partial X^i}{\partial \mathring{x}^k}.
    \]
    \[
      F = \left( \dfrac{\partial x^j}{\partial X^i}  \right) \left( \dfrac{\partial X^i}{\partial \mathring{x}^j}  \right) \bar{\vec{e}}_j \otimes \bar{\vec{e}}_k
      = \dfrac{\partial x^j}{\partial \mathring{x}^k} \bar{\vec{e}}_j \otimes \bar{\vec{e}}_k.
    \]
    \[
      \det F = \det \left( \dfrac{\partial x^j}{\partial \mathring{x}^k}  \right) 
    \]
  
  \item Транспонированный тензор: $F^T = \mathring{\vec{r}}_i \otimes \vec{r}^i$.
    Обратный тензор: $F^{-1} \cdot F = E$.
    $F^{-1}$ -- существует, т.к. $\det F \neq 0$.
    Тогда обратным к тензору градиента деформации будет $F^{-1} = \mathring{\vec{r}}_i \otimes \vec{r}^i$.
    Проверим, что это действительно обратный тензор:
    \[
      F^{-1} \cdot F
      = (\mathring{\vec{r}}_i \otimes \vec{r}^i) \cdot (\vec{r}_i \otimes \mathring{\vec{r}}^i)
      = \mathring{\vec{r}}_i \otimes \delta^i_j \otimes \mathring{\vec{r}}^j
      = \mathring{\vec{r}}_i \otimes \mathring{\vec{r}}^i = E
    \]

    Причем, $F^{-1 T} = \vec{r}^i \otimes \mathring{\vec{r}}_i$.

  \item $\nabla \otimes \vec{a}$ и $\mathring{\nabla} \otimes \vec{a}$:
    \[
      \nabla \otimes \vec{a} = \vec{r}^i \otimes \dfrac{\partial \vec{a}}{\partial X^i}
      = \vec{r}^j \delta^i_j \otimes \dfrac{\partial \vec{a}}{\partial X^i} 
      = \vec{r}^j \otimes (\mathring{\vec{r}} )
      = \dots
      = F^{-1 T} \cdot \mathring{\nabla} \otimes \vec{a}.
    \]
    \[
      \nabla \otimes \vec{a} = F^{-1 T} \cdot \mathring{\nabla} \vec{a}.
      \quad
      F^T \cdot \nabla \otimes \vec{a} = F^T \cdot F^{-1 T} \cdot \mathring{\nabla} \otimes \mathring{\vec{a}} = E \cdot \mathring{\nabla} \otimes \vec{a}.
    \]
    Таким образом, получили, что тензор градиента деформации полностью задаёт все преобразования 
    из начальной конфигурации:
    \[
      \mathring{\nabla} \otimes \vec{a} = F^T \cdot \nabla \otimes \vec{a}.
    \]
\end{enumerate}

\subsection{Тензоры и меры деформации}

% TODO переписать (здесь пока некрасиво совсем) и проверить на опечатки

Ввведём еще один способ сравнения двух конфигураций $\mathring{\mathcal{K}}$ и $\mathcal{K}$.
Сравним напрямую метрические матрицы:
\[
  \varepsilon_{ij} = \dfrac{1}{2} \left( g_{ij} - \mathring{g}_{ij} \right) 
\]
-- компоненты тензора деформации.
\[
  \varepsilon^{ij} = \dfrac{1}{2} (g^{ij} - \mathring{g}^{ij})
\]
--  <<контравариантные>> компоненты тензора деформации. Это просто обозначение не является очень
удачным, потому что они не получаются стандартной операцией поднятия индексов.

Построим следующие 4 тензора:
\[
  C = \varepsilon_{ij} \mathring{\vec{r}}^i \otimes \mathring{\vec{r}}^j; \quad
\]
-- правый тензор деформации Коши-Грина.

\[
  A = \varepsilon_{ij} \vec{r}^i \otimes \vec{r}^j
\]
--правы тензор деформации Альманзи.

\[
  \varepsilon^{ij} \mathring{\vec{r}}_i \otimes \mathring{\vec{r}}_j
\]
-- левый тензор деформация Альманзи.

\[
  J = \varepsilon^{ij} \vec{r}_i \otimes \vec{r}_j
\]
-- левый тензор деформации Коши-Грина.

\begin{theorem}
  \begin{align*}
    C &= \dfrac{1}{2} (F^T \cdot F - E); \\
    A &= \dfrac{1}{2} (E - F^{-1 T} \cdot F^{-1}); \\
    \Lambda &= \dfrac{1}{2} (E - F^{-1} \cdot F^{-1 T}); \\
    J &= \dfrac{1}{2}(F \cdot F^T - E)
  \end{align*}
\end{theorem}
\begin{proof}
  \begin{equation*}
    C
    = \varepsilon_{ij} \mathring{\vec{r}}^i \otimes \mathring{\vec{r}}^j
    = \dfrac{1}{2} (g_{ij} - \mathring{g}_{ij}) \mathring{\vec{r}}^i \otimes \mathring{\vec{r}}^j
    = \dfrac{1}{2} (g_{ij} \mathring{\vec{r}}^i \otimes \mathring{\vec{r}}^j - \mathring{g}_{ij} \mathring{\vec{r}}^i \otimes \mathring{\vec{r}}^j)
    = \dfrac{1}{2} ( \mathring{\vec{r}}^i (\vec{r}_i \cdot \vec{r}_j) \otimes \mathring{\vec{r}}^j) - E)
    = \dfrac{1}{2} (F^T \cdot F - E).
  \end{equation*}
  Остальные формулы доказываются аналогично.
\end{proof}

Полезным оказывается введение меры деформации. 
\[
  G = g_{ij} \mathring{\vec{r}}^i \otimes \mathring{\vec{r}}^j = F^T \cdot F
\]
-- право мера деформации Коши-Грина.

\[
  \mathbf{g} = \mathring{g}_{ij} \vec{r}^i \otimes \vec{r}^j
\]
-- левая мера деформации Альманзи.

\[
  \mathbf{g}^{-1} = \mathring{g}^{ij} \vec{r}_i \otimes \vec{r}_j
\]
-- правая мера деформации Альманзи.

\[
  G^{-1} = g^{ij} \mathring{\vec{r}}_i \otimes \mathring{\vec{r}}_j
\]
-- правая мера деформации Коши-Грина.

\subsection{Вектор перемещений}

Рассмотрим еще один способ сравнения двух конфигураций $\mathring{\mathcal{K}}$ и $\mathcal{K}$.

% TODO рисунок

Введем вектор $\vec{u} = \vec{x} - \mathring{\vec{x}},$ где $\vec{x}$ и $\mathring{\vec{x}}$ --
радиус-векторы одной и той же материальной точки $M$. Обычно используется в самом конце решения задачи, например в ANSYS сначала решается задача, потом строятся эти вектора и вот чото дальше туда сюда.

Как связать $\vec{u}$ и $F$?
Утверждение: $F = E + (\mathring{\nabla} \otimes \vec{u})^T$
\begin{proof}
  Рассмотрим тензор $F^T = \mathring{\vec{r}}^i \otimes \vec{r}_i = \mathring{\vec{r}}^i \otimes \dfrac{\partial \vec{x}}{\partial X^i} = \left( \mathring{\vec{r}}^i \dfrac{\partial }{\partial X^i}  \right) \otimes \vec{x} = \mathring{\vec{r}}_i \mathring{\nabla} \otimes \vec{x}$.
  \[
    F^T = \mathring{\nabla} \otimes \vec{x} = \mathring{\nabla} \otimes (\mathring{\vec{x}} + \vec{u}) = \mathring{\nabla} \otimes \mathring{\vec{x}} + \mathring{\nabla} \otimes \vec{u},
  \]
  но $\mathring{\nabla} \otimes \mathring{\vec{x}} = \dots = E$, тогда: % TODO дописать
  $ F^T = E + \mathring{\nabla} \otimes \vec{u} $, или $F = E + (\mathring{\nalba} \otimes \vec{u})^T$.
\end{proof}

\begin{corollary}[Связь тензора деформации с вектором перемещений]
  \[
    C = \dfrac{1}{2} (F^T \cdot F - E)
    = \dfrac{1}{2} ( (E + \mathring{\nabla} \otimes \vec{u})(E + \mathring{\nabla} \otimes u^T) - E)
    = \dfrac{1}{2} ( \mathring{\nabla} \otimes \vec{u} + \mathring{\nabla} \otimes \vec{u}^T + \mathring{\nabla} \otimes \vec{u} \cdot \mathring{\nabla} \otimes \vec{u}^T )
  \]
  Первая часть называется симметричной частью, а вторая -- квадратичной.
\end{corollary}

Для так называемых сред с малыми деформациями (подавляющее большинство твердых веществ такие) --
где можно пренебречь квадратичной частью. Тогда тензор $C$:
\[
  C = \dfrac{1}{2} (\mathring{\nabla} \otimes \vec{u} + \mathring{\nabla} \otimes \vec{u}^T) \equiv \varpepsilon,
\]
где $\varepsilon$ называется тензором линейных деформаций.

Квадратичное слагаемое -- один из основных источников нелинейности. На ней основаны такие важные эффекты, как потеря устойчивости конструкции. На рисунке представлено некоторое здание, стоящее на тонких колоннах. Прочность будет оставаться такой же как если бы это были не колонны, а цельное что-то,
но у колонн может возникнуть деформации, изображенные на рисунке.
% TODO  

Нелинейность является источником неединственности.

  \subsection{Физический (геометрический) смысл компонент тензора деформаций}

Определение компонент тензора деформаций:
\[
  \varepsilon_{ij} = \dfrac{1}{2} \left( g_{ij} -  \mathring{g}_{ij} \right).
\]

По определению $g_{ij}$ и $\mathring{g}_{ij}$ (переходим к греческим индексам):
\begin{align*}
  g_{\alpha\beta} &= \mathbf{r}_\alpha \cdot \mathbf{r}_\beta \\
  \varepsilon_{\alpha \beta} &= \dfrac{1}{2} \left( \mathbf{r}_\alpha \mathbf{r}_\beta - \mathring{\mathbf{r}}_\alpha \mathring{\mathbf{r}}_\beta \right) 
  = 
\end{align*}
Было введено понятие угла $\psi_{\alpha\beta}$ между векторами:
в $\mathbb{E}_3$ введеино понятие угла как такое число, что в скалярном произведении
будет его косинус, т.е. $\mathbf{r}_\alpha \cdot \mathbf{r}_\beta = |\mathbf{r}_\alpha| \cdot |\mathbf{r}_\beta| \cos \psi$, тогда:
\[
  \varepsilon_{\alpha\beta} = \dfrac{1}{2} \left( |r_\alpha| |r_\beta| \cos \psi - |\mathring{r}_\alpha| |\mathring{r}_\beta| \cos \mathring{\psi} \right) 
\]

% TODO рисунок

Рассмотрим элементарный радиус-вектор в $\mathcal{K}$ и $\mathcal{\mathring{K}}$:
$d\mathbf{x} = \vec{M\tilde M_1} = \mathbf{r}_i dX^i$

В частном случае выбор $d\mathbf{x}$, ориентированные по какому-нибудь координатному направлению:
$d\mathbf{x}_\alpha = \vec{MM_\alpha} = \mathbf{r}_\alpha dX^\alpha$ (нет суммирования).
В начальный момент времени: $d\mathring{\mathbf{x}}_\alpha = \vec{MM_\alpha} = \mathring{\mathbf{r}}_\alpha dX^\alpha$.
% TODO рисунок

Введем длины этих векторов:
\begin{align*}
  d\mathring{s}_\alpha &\equiv |d\mathring{x}_\alpha|
  = (d\mathring{x}_\alpha \cdot d\mathring{x}_\alpha)^{1/2}
  =  |\mathring{\mathbf{r}}_\alpha| |dX^\alpha|, \\
  ds_\alpha &\equiv |dx_\alpha|
  = (dx_\alpha \cdot dx_\alpha)^{1/2}
  =  |\mathbf{r}_\alpha| |dX^\alpha|.
\end{align*}

Как их сравнивать? Введем отклонение:
\[
  ds_\alpha - d\mathring{s}_\alpha = (|\mathbf{r}_\alpha| - |\mathbf{\mathring{r}}_\alpha|) |dX^\alpha|,
\]
\[
  \dfrac{ds_\alpha}{d\mathring{s}_\alpha}
  = \dfrac{|\mathbf{r}_\alpha| |dX^\alpha|}{|\mathbf{\mathring{r}}_\alpha| |dX^\alpha| }
  = \dfrac{|\mathbf{r}_\alpha|}{|\mathbf{\mathring{r}}_\alpha|}.
\]
\[
  \delta_\alpha
  \equiv \dfrac{ds_\alpha - d\mathring{s}_\alpha}{d\mathring{s}_\alpha}
  = \dfrac{|\mathbf{r}_\alpha|}{|\mathbf{\mathring{r}}_\alpha|} - 1.
\]
-- относительное удлинение.

Из определения относительного удлинения следует, что
\[
  |\mathbf{r}_\alpha| = (1 + \delta_\alpha) \cdot |\mathring{\mathbf{r}}_\alpha|.
\]

Тогда тензор деформаций:
\[
  \varepsilon_{\alpha\beta} = \dfrac{1}{2} \left( (1+\delta_\alpha) (1+\delta_\beta) \cos\psi - \cos\mathring{\psi} \right).
\]

Рассмотрим что у нас получилось:
Если $\alpha=\beta$, при этом $\psi_{\alpha\alpha} = \mathring{\psi}_{\alpha\alpha} = 0$, тогда 
\[
  \varepsilon_{\alpha\alpha}
  = \dfrac{1}{2} |\mathring{\mathbf{r}}_\alpha|^2 ((1+\delta_\alpha)^2 - 1)
  = \dfrac{1}{2} \mathring{g}_{\alpha\alpha} ((1+\delta_\alpha)^2 - 1)
\]

Выберем теперь такие координаты, чтобы в начальный момент времени они совпадали с декартовыми:
$X^\alpha = \mathring{x}^\alpha$.
% TODO рисунок бруска (уже был)
Тогда $\mathring{g}_{\alpha\alpha} = 1$, тогда
\[
  \varepsilon_{\alpha\alpha}
  = \dfrac{1}{2} ((1+\delta_\alpha)^2 - 1)
\]

Рассмотрим еще случай, когда $|\delta_\alpha| \ll 1$ во всех точках тела, тогда
\[
  \varepsilon_{\alpha\alpha} \approx \delta_\alpha.
\]

\begin{example}
  % TODO рисунок брусочек сплющивается.

  Так как удлинение бруска однородно в данной задаче, то для всех точек тела верно:
  \[
    \delta_1 = \dfrac{l_1 - \mathring{l}_1}{\mathring{l}_1} = \dfrac{ds_\alpha - d\mathring{s}_\alpha}{d\mathring{s}_\alpha}
  \]

  Удлинение элементарных отрезоков, ориентированных вдоль координатных прямых.
\end{example}

Для большинства твердых тел ($\sim 95\%$) оказывается, что $|\delta_\alpha| \ll 1$!

Диаграмма деформации:
% TODO график $T_{aa}$ от $delta_a$. -- можно не рисовать, всё равно еще не проходили.

% TODO еще был пример с рисунком, но он говно.

Если $|\delta_\alpha| \ll 1$, то среду называют средой с малыми деформациями.
Иначе, среду называют с конечными деформации.

Рассмотрим теперь случай, когда $\alpha\neq\beta$. Выберем начальную систему координат 
совпадающей с декартовой: $X^\alpha = \mathring{x}^\alpha$, т.е.
$\mathbf{\mathring{r}}_\alpha = \bar{\mathbf{e}}$. Тогда $\psi_{\alpha\beta} = \pi/2$.
Тогда
\[
  \varepsilon_{\alpha\beta} = \dfrac{1}{2} (1+\delta_\alpha)(1+\delta_\beta) \cos\psi_{\alpha\beta}.
\]

Введем обозначение: $\chi_{\alpha\beta} = \psi_{\alpha\beta} - \mathring{\psi}_{\alpha\beta}
= \psi_{\alpha\beta} - \dfrac{\pi}{2}$ -- угол скашивания.
\[
  \cos\psi_{\alpha\beta} = \cos\left(\chi_{\alpha\beta} + \dfrac{\pi}{2}\right)
  = \sin\chi_{\alpha\beta}
\]

Тогда
\[
  \varepsilon_{\alpha\beta} = \dfrac{1}{2} (1+\delta_\alpha)(1+\delta_\beta) \sin\chi_{\alpha\beta}
\]

Если $|\delta_\alpha| \ll 1$ (вероятно, имеется ввиду еще и $\delta_\beta \ll 1$), т.е.
рассматривается случай малых деформаций, тогда и углы деформаций малы: $\chi_{\alpha\beta} \ll 1$,
тогда:
\[
  \varepsilon_{\alpha\beta} = \dfrac{1}{2} \chi_{\alpha\beta}
\]

\begin{example}
  % TODO рисунок
\end{example}

Таким образом, получили, что в случае малых деформаций:
\[
  \begin{cases}
    \varepsilon_{\alpha\beta} = \dfrac{1}{2} \chi_{\alpha\beta}, \\
    \varepsilon_{\alpha\alpha} = \delta_\alpha.
  \end{cases}
\]

\subsection{Преобразование ориентированной площадки}

Рассмотрим в $\mathring{\mathcal{K}}$ такой объект, как $d\mathring{\mathbf{x}}_\alpha$.
% TODO рисунок
Взяв пару из тройки элементарных радиус-векторов, вычислим их векторное произведение:
\[
  d\mathring{\mathbf{x}}_\alpha \times d\mathring{\mathbf{x}}_\beta
  = \mathring{\mathbf{n}} d\mathring{\Sigma}_\gamma
  \equiv \mathring{\mathbf{n}} d\gamma,
\]
где $d\mathring{\Sigma}_\gamma$ -- площадь элементарной площадки, образованной элементарными
радиус-векторами $d\mathring{\mathbf{x}}_\alpha$ и $d\mathring{\mathbf{x}}_\beta$.

Как преобразуется $\mathring{\mathbf{n}} d\mathring{\Sigma}_\gamma$ при переходе из
$\mathcal{\mathring{K}}$ в $\mathcal{K}$?
Воспользуемся тем, что вектор $d\mathring{\mathbf{x}}_\alpha = \vec{MM_1}$ соединяет те же самые
точки, что и $d\mathbf{x}_\alpha = \vec{MM_1}$.
\[
  \mathbf{n} d\Sigma_\gamma = d\mathbf{x}_\alpha \times d\mathbf{x}_\beta
  = \mathbf{r}_\alpha dX^\alpha \times \mathbf{r}_\beta dX^\beta
  = \sqrt{g} \epsilon_{\alpha\beta\gamma} \mathbf{r}^\gamma dX^\alpha dX^\beta.
\]
Аналогично, $\mathring{\mathbf{n}} d\mathring{\Sigma} = \sqrt{\mathring{g}} \epsilon_{\alpha\beta\gamma} \mathring{\mathbf{r}}^\gamma dX^\alpha dX^\beta$.

Вспомним о градиенте деформации, а точнее об его свойстве: $\mathbf{r}_i = F\cdot\mathring{\mathbf{r}}_i$.

Подставим это в выражение выше:
\[
  \mathbf{n} d\Sigma_\gamma
  = \sqrt{g} \epsilon_{\alpha\beta\gamma} \mathbf{r}^\gamma dX^\alpha dX^\beta
  = \sqrt{g} \epsilon_{\alpha\beta\gamma} F^{-1T} \mathring{\mathbf{r}}^\gamma dX^\alpha dX^\beta
  = \dots
  = \sqrt{\dfrac{g}{\mathring{g}}} F^{-1T} \mathring{\mathbf{n}} d\mathring{\Sigma}.
\]


\subsection{Полярное разложение}

\begin{theorem}[о полярном разложении]
  Всякий невырожденный тензор $F$, то есть такой, который $\det F \neq 0 \forall x^i \in V_x \forall t \geqslant 0$ можно представить в виде:
  \begin{equation}\label{lec_5:theorem_polar}
    F = O\cdot U = V\cdot O,
  \end{equation}
  где $O$ -- тензор поворота сопровождающей деформации, $O$ -- ортогональный тензор,
    $U$ и $V$ -- правый и левый тензоры искажения -- симметричные ($U^T = U$), положительно определённые
    тензоры ($\forall \mathbf{a} \neq \mathbf{0} : \mathbf{a} \cdot U \cdot \mathbf{a} > 0$),
    причём разложение \eqref{lec_5:theorem_polar} единственное.
\end{theorem}

% TODO сделать окружение remark
\begin{remark*}
  Теорема может быть применена к любому невырожденному тензору, в частности к градиенту деформации
  $F$.
\end{remark*}

\begin{remark*}
  Покажем, что $\det F \neq 0$, действительно, вспомним, что такое детерминант тензора --
  это детерминант матрицы компонент тензора в любом смешанном базисе:
  \[
    \det F = \det (\mathbf{r}_i \otimes \mathring{\mathbf{r}}^i)
    = \tensor{F}{_i^j} \bar{\mathbf{e}}_i \otimes \bar{\mathbf{e}}^j % TODO ошибка блять
  \]

  \[
    \det F = \det \left( \dfrac{\partial x^i}{\partial \mathring{x}^j}  \right) \neq 0 
    \forall \mathring{x}^i \in \mathring{V}.
  \]
  -- верно для любого невырожденного тела.
\end{remark*}

\begin{remark*}
  Вот такие тензоры мы уже знаем: $E, F, C, A, \Lambda, J$.

  $C = \dfrac{1}{2} \left( F^T \cdot F - E \right) \forall t \geqslant 0$, но $F(0) = \mathbf{r}_i(t) |_{t=0} \otimes \mathring{\mathbf{r}}^i = \mathring{\mathbf{r}}_i \otimes \mathring{\mathbf{r}}^i = E \Rightarrow C|_{t=0} = 0 \Rightarrow \det C |_{t=0} = 0$.

  Пусть для некоторого невырожденного тензора $B$ применим теорему о полярном разложении:
  $B = O_B \cdot U_B = V_B \cdot O_B$.

  А что для тензора $E$? Оказывается, что $O_E = U_E = V_E = E$
\end{remark*}

\begin{corollary}
  \begin{enumerate}
    \item $F^T \cdot F = (O\cdot U)^T \cdot (O \cdot U) = U^T \cdot O^T \cdot U \cdot U = U^T \cdot U= U \cdot U = U^2$, и $F \cdot F^T = (V O) (V O)^T = \dots = V^2$;

    \item $F^{-1T} \cdot F^{-1} = (V \cdot O)^{-1T} \cdot (V \cdot O)^{-1}
      = (O^{-1} \cdot V^{-1})^T \cdot O^{-1} \cdot V^{-1}
      = V^{-1T} \cdot O^{-1T} \cdot O^{-1} \cdot V^{-1}
      = V^{-2}$;

    \item Тогда тензор Коши-Грина: $C = \dfrac{1}{2} (F^T \cdot F - E) = \dfrac{1}{2} \left( U^2 - E \right)$ -- поэтому называется правый тензор деформации Коши-Грина.

    \item $A = \dfrac{1}{2} \left( E - F^{-1T} \cdot T^{-1} \right)  = \dfrac{1}{2} \left( E - V^{-2} \right) $ -- левый тензор деформации Альмандзи.

    \item $\Lambda = \dfrac{1}{2} (E - F^{-1} \cdot F^{-1T}) = \dfrac{1}{2} (E - U^{-2})$ --
      правый тензор деформации Альмандзи.

    \item $J = \dfrac{1}{2} \left( F \cdot F^T - E \right) = \dfrac{1}{2} (V^2 - E)$
      -- левый тензор деформации Коши-Грина.
  \end{enumerate}
\end{corollary}

\begin{definition}
  Введём понятие собственный векторов и собственных значений:
  Будем говорить, что $\mathbf{p}_{A\alpha}$ -- правый собственный вектор для тензора второго ранга $A$, если
  он удовлетворяет уравнению $A \cdot \mathbf{p}_{A\alpha} = \lambda_{A\alpha} \cdot \mathbf{p}_{A\alpha}$, а числа $\lambda_{A\sigma}$ -- правое собственное значение.

  Аналогично, левым собственным вектором называется такой вектор $\mathbf{p}^*_{A\alpha}$, что:
  $\mathbf{p}^*_{A\alpha} \cdot A = \lambda^*_{A\alpha} \cdot \mathbf{p}^*_{A\alpha}$.
\end{definition}

\begin{corollary}
  Если $A$ -- симметричный, то:
  \begin{itemize}
    \item все собственные значения вещественные;
    \item все собственные векторы вещественнозначные, т.е. все компоненты в декартовом базисе
      вещественные.
    \item $\mathbf{p}_{A\alpha} = \mathbf{p}^*_{A\alpha}$
    \item $\mathbf{p}_{A\alpha} \cdot \mathbf{p}_{A\beta} = \delta_{\alpha\beta}.$
  \end{itemize}

  Если он еще и положительно определенный, то все собственные значения еще и положительны.
\end{corollary}

\begin{corollary}
  Применим эту теорему для $U$ и $V$, так как они симметричны и положительно определенные, то для них существуют собственные значения и собственные векторы.
\end{corollary}

Утверждение: $\lambda_\alpha = \mathring{\lambda}_\alpha$, но в общем случае 
% TODO дописать

\begin{corollary}
  Представим симметричный тензор в собственном базисе. Если $A$ -- симметричный,
  $\mathbf{p}_{A\alpha}$ -- собственные векторы, $\lambda_{A\alpha}$ -- собственные значения,
  тогда
  \[
    A = \sum_{\alpha=1}^3 \lambda_{A\alpha} \mathbf{p}_{A\alpha} \otimes \mathbf{p}_{A\alpha}
    = A^{ij} \mathbf{r}_i \otimes \mathbf{r}_j.
  \]
\end{corollary}
\begin{proof}
  Для доказательства подставим в определение собственных векторов для симметричного тензора:
  \[
    A \cdot \mathbf{p}_{A\alpha}
    = \sum_{\beta=1}^3 \lambda_{A\beta} (\mathbf{p}_{A\alpha}\otimes \mathbf{p}_{A\beta}) \cdot \mathbf{p}_{A\alpha}
    = \sum_{\beta=1}^3 \lambda_{A\beta} \mathbf{p}_{A\alpha} \otimes \delta_{\alpha\beta}
    = \sum_{\beta=1}^3 \lambda_{A\beta} 
    = \lambda_{A\alpha} \mathbf{p}_{A\alpha}
  \]
\end{proof}

Применим теперь это следствие к $U$ и $V$:
\[
  U= \sum_{\alpha=1}^3 \lambda_\alpha \mathring{\mathbf{p}}_\alpha \otimes \mathring{\mathbf{p}}_\alpha
\]
\[
  V= \sum_{\alpha=1}^3 \lambda_\alpha \mathbf{p}_\alpha \otimes \mathbf{p}_\alpha

\]

\begin{corollary}
  \[
    U^2 = \sum_{\alpha=1}^3 \lambda_\alpha \mathring{\mathbf{p}}_\alpha \otimes \mathring{\mathbf{p}}_\alpha \cdot 
    \left( \sum_{\beta=1}^3 \lambda_\beta \mathring{\mathbf{p}}_\beta \otimes \mathring{\mathbf{p}}_\beta \right)
    = \sum_{\alpha, \beta = 1}^3 \lambda_\alpha \lambda_\beta \mathring{\mathbf{p}}_\alpha \otimes (\mathring{\mathbf{p}}_\alpha \cdot \mathring{\mathbf{p}}_\beta) \otimes \mathring{\mathbf{p}}_\beta
    = \sum_{\alpha=1}^3 \lambda_\alpha \mathring{\mathbf{p}}_\alpha \otimes \mathring{\mathbf{p}}_\alpha.
  \]

  Аналогичное можно доказать для любой степени, а также для функции от тензора, определение которой 
  будет дано ниже.
\end{corollary}

\begin{definition}
  Функцией от тензора называется:
  \[
    f(U) \equiv \sum_{\alpha=1}^3 f(\lambda_\alpha) \mathring{\mathbf{p}}_\alpha \otimes \mathring{\mathbf{p}}_\alpha.
  \]
\end{definition}


Собственные значения тензоров деформации:
$C = \dfrac{1}{2} (F^T \cdot F - E) = \dfrac{1}{2} \left( U^2 - E \right) = \sum_{\alpha=1}^3 \dfrac{1}{2} (\lambda^2 - 1) \mathring{\mathbf{p}}_\alpha \otimes \mathring{\mathbf{p}}_\alpha$.

\begin{corollary}
  Как связаны $\mathring{\mathbf{p}}_\alpha$ и $\mathbf{p}_\alpha$? ответ убил:
  \[
    \mathbf{p}_\alpha = O \cdot \mathring{\mathbf{p}}_\alpha.
  \]

  % TODO рисунок с большой буквой О
\end{corollary}

Тензор градиента деформации можно представить в своём <<собственном>> базисе, состоящем из
собственных векторов $U$ и $V$:
\[
  F = \sum_{\alpha=1}^3 \lambda_{\alpha} \mathbf{p}_\alpha \otimes \mathring{\mathbf{p}}_\alpha,
\]
а тензор $O$:
\[
  O = \mathbf{p}_i \otimes \mathring{\mathbf{p}}^i
\]

\begin{corollary}
  Выберем $d\mathring{\mathbf{x}}_\alpha = d\mathring{K} \mathring{\mathbf{p}}_\alpha = |d\mathring{\mathbf{x}}_\alpha| \mathring{\mathbf{p}}_\alpha$. Тогда
  \[
    d\mathbf{x}_\alpha = F \cdot d\mathring{\mathbf{x}}_\alpha = \sum_{\alpha=1}^3 \lambda_\beta \mathbf{p} \otimes \delta_{\alpha\beta} |d\mathring{\mathbf{x}}_\alpha| = \dots = \lambda_\alpha \mathbf{p}_\alpha |d\mathring{\mathbf{x}}_\alpha|.
  \]

  То есть $d\mathbf{x}_\alpha = \lambda_\alpha \mathbf{p}_\alpha |d\mathring{\mathbf{x}}_\alpha|$.

  Если выбирать собственные вектора единичной длины, то $|d\mathbf{x}_\alpha| = \lambda_\alpha |d\mathring{\mathbf{x}}_\alpha|$, отсюда следует геометрический смысл собственных значений -- 
  если в отсчетной конфигурации выбрать векторы по специальным направлениям  (которые мы сможем узнать только после появления тензора градиента деформации), то в актуальной конфигурации этот вектор останется коллинеарным этому направлению, а длина увеличится на собственное значение, соответствующее этому направлению.
\end{corollary}

  \subsection{Геометрическая картина преобразования малой окрестности произвольной материальной
точки $M$}

% TODO рисунок

Согласно теореме, полярное разложение градиента деформации: 
\[
  F = O \cdot U = V \cdot O
\]

Тогда
\[
  d\mathbf{x} = O \cdot U \cdot d\mathring{\mathbf{x}} = V \cdot O \cdot d \mathring{\mathbf{x}}.
\]

Введём обозначения:
\[
  \begin{cases}
    d\mathbf{x}' = U \cdot d\mathring{\mathbf{x}}, \\
    d\mathbf{x} = O \cdot d\mathbf{x}',
  \end{cases}
\]
\[
  \begin{cases}
    d\mathbf{x}' = O \cdot d\mathring{\mathbf{x}}, \\
    d\mathbf{x} = V \cdot d\mathbf{x}',
  \end{cases}
  
\]

таким образом, с помощью полярного разложения преобразование малой окрестности
$dV$ точки $M$ при переходе из $\mathring{K}$ в $K$ можно представить как суперпозицию
двух преобразований. Причём это можно сделать двумя способами:
\begin{itemize}
  \item преобразование с помощью $U$ или $V$ (изменение углов и преобразование длин);
  \item преобразование поворота как жесткого целого.
\end{itemize}

\begin{example}
  Геометрическая картина преобразования малой окрестности произвольной материальной
точки $M$:
  \begin{itemize}
    \item Первый способ:
      Если $d\mathring{V}$ -- куб. На рисунке представлено действие тензора $U$ (или $V$),
      а далее действие тензора $O$ -- просто поворот.
      % TODO рисунок три кубика
      Меняются углы и длины ребёр -- так действует $U$. После этого действует тензор $O$, который 
      сохраняет углы и длины.

    \item Второй способ:
      сначала выполнен поворот, потом применён тензор $V$.
  \end{itemize}
\end{example}



\subsection{Скоростные характеристики движения сплошной среды}

Введём понятие \emph{скорости} материальной точки $M$.

Рассмотрим закон движения материальной точки сплошной среды:
\begin{equation}\label{lec_6:eq_zakon_dv}
  \mathbf{x} = \mathbf{x} (X^i, t), X^i \in V_{x}, t \in [0, t_{max})
\end{equation}
По предплоложению, закон движения гладкий в каждой точке, а также невырожденный.

Дифференцируя \eqref{lec_6:eq_zakon_dv}:
\[
  \mathbf{v}(X^i, t) \equiv \left. \dfrac{\partial \mathbf{x}}{\partial t} (X^i, t) \right|_{X^i}
\]

Разложим закон движения по декартовому базису:
$\mathbf{x} = x^i \bar{\mathbf{e}}_i = x^i(X^j, t) \bar{\mathbf{e}}_i$.
Тогда: $\mathbf{v} = \dfrac{\partial }{\partial t} \left( x^i \bar{\mathbf{e}}_i \right) 
  = \dfrac{\partial x^i}{\partial t} \bar{\mathbf{e}}_i = \bar{v}^i \bar{\mathbf{e}}_i$.
Декартовы компоненты вектора скорости:
$\bar{v}^i (X^j, t) = \dfrac{\partial x^i}{\partial t} (X^j, t)$.

Рассмотрим произвольное гладкое векторное поле, переменное по $X^i$ и $t$:
$\mathbf{a} = \mathbf{a}(X^i, t) = \mathbf{a}(X^j(x^k, t), t) = \tilde{\mathbf{a}} (x^k, t)$.
Тогда рассмотрим производную:
$ \dfrac{d\mathbf{a}}{dt} \equiv \left.\dfrac{\partial \mathbf{a}}{\partial t} (X^j, t) \right|_{X^i}$ -- полная производная от векторного поля по времени $t$ (aka \emph{субстанциональная}).

Рассмотрим теперь представление этой полной производной:
\[
  \dfrac{d\mathbf{a}}{dt} = \dfrac{d}{dt} \tilde{\mathbf{a}}(x^k, t)
  = \dfrac{d}{dt} \tilde{\mathbf{a}} (x^k(X^i, t), t)
  = \dfrac{\partial \hat{\mathbf{a}}}{\partial x^k} \cdot \dfrac{\partial x^k}{\partial t}
  + \left. \dfrac{\partial \hat{\mathbf{a}}}{\partial t} \right|_{x^k}
\]

Преобразуем первое слагаемое:
\begin{multline*}
  \dfrac{\partial \tilde{\mathbf{a}}}{\partial x^k} \cdot \dfrac{\partial x^k}{\partial t} 
  = \dfrac{\partial \mathbf{a}}{\partial X^i} \cdot \dfrac{\partial X^i}{\partial x^k} \cdot \dfrac{\partial x^k}{\partial t}
  = \dfrac{\partial \mathbf{a}}{\partial X^i} \cdot \tensor{P}{^i_k} \cdot \mathbf{v}^k (x^i, t)
  = \mathbf{v}^j \cdot \delta^k_j \cdot \tensor{P}{^i_k} \cdot \dfrac{\partial \mathbf{a}}{\partial X^i} = \\
  = \mathbf{v}^j \cdot \bar{\mathbf{e}}_j \cdot \bar{\mathbf{e}}^k \cdot \tensor{P}{^i_k} \cdot \dfrac{\partial \mathbf{a}}{\partial X^i}
  = \mathbf{v} \cdot \mathbf{r}^i \otimes \dfrac{\partial \mathbf{a}}{\partial X^i} 
  = \mathbf{v} \cdot \nabla \otimes \mathbf{a}
\end{multline*}

Таким образом:
\begin{equation}\label{lec_6:full_derivative}
  \dfrac{d \mathbf{a}}{dt}
  = \mathbf{v} \cdot \nabla \otimes \mathbf{a}
  + \left. \dfrac{\partial \hat{\mathbf{a}}}{\partial t} \right|_{x^k}
\end{equation}

% \[
%   \dfrac{d}{dt} \mathbf{a} = \left. \dfrac{\partial \mathbf{a}}{\partial t} (x^i, t) \right|_{x^i}
%     + \mathbf{v} \cdot \nabla \otimes \mathbf{a}
% \]


$\mathbf{v} \cdot \nabla \otimes \mathbf{a}$ -- конвективная производная от векторного поля $\mathbf{a}$.

% TODO рисунок какие-то пауки чото палки синусоида чиво

Обобщением формулы \eqref{lec_6:full_derivative} для скаляров будет:
\[
  \dfrac{d\varphi}{dt} (x^i, t) = \left. \dfrac{\partial \varphi}{\partial t} \right|_{x^i}
  + \mathbf{v} \cdot \nabla \varphi.
\]

Для тензорного поля:
\[
  \dfrac{d\Omega}{dt} = \left. \dfrac{\partial \Omega}{\partial t} (x^i, t) \right|_{x^i}
    + \mathbf{v} \cdot \nabla \otimes \Omega.
\]

Еще введём интересный объект, рассмотрение свойств которого даже не входит в курс:

\begin{definition}
  $D = \dfrac{1}{2} \left( \nabla \otimes \mathbf{v} + \nabla \otimes \mathbf{v}^T \right) $
  называется \emph{тензором скоростных деформаций} (симметричный).
\end{definition}



\section{Законы сохранения}

\subsection{Закон сохранения массы}

\paragraph{Аксиома 4. Закон сохранения массы}
Для всякой сплошной среды $V$ в $\mathcal{K}$ в любой момент времени $t>0$
существует скалярная функция $M(V, t)$, называемая \emph{массой тела} (массой
сплошной среды), и обладающая следующими свойствами:
\begin{itemize}
  \item Положительность: $M>0, M \in \mathbb{R}_{+}$;
  \item Аддитивность: $M(V_1 \bigcup V_2, t) = M(V_1, t) + M(V_2, t)$
    для любого разбиения области $V$, в том числе и на континуальное разбиение.
  \item Инвариантность по отношению к любому движению: $M(V_1, t) = \operatorname{const}$
    $\Leftrightarrow$ $ \dfrac{dM}{dt} = 0 $, если тело состоит из одних и тех 
    же материальных точек.
\end{itemize}

Массу нельзя определить ни через какие введёные выше понятия. Впервые масса появляется
в этой аксиоме.

% TODO рисунок

\begin{corollary}
  В силу аддитивности массы массу сплошной среды $V$ можно представить следующим образом:
  $V = \int_V dV,$ так как каждая сплошная среда, в том числе $dV$ обладает
  массой, то обозначим массу элементарного объёма $dV$ как $dm$ (следует из аксиомы и из
  принципов дифференциального исчисления). Тогда в силу аддитивности:
  $M(V, t) = \int_V dm$.
\end{corollary}

\begin{definition}
  % TODO рисунок

  Введём отношение $ \dfrac{dm}{dV} \equiv \rho > 0$ и назовём его \emph{плотностью}
  вещества (согласно принятым обозначениям, $dV$ -- объём области $dV$).

  В отличие от массы, плотность определена в точке -- это предел отношения
  массы к объёму окрестности: $\rho = \lim_{|\Delta V| \to 0} \dfrac{\Delta m}{|\Delta V|} = \dfrac{dm}{dV}$.
  Таким образом, $\rho(X^i, t)$ -- скалярное поле.
\end{definition}

\begin{corollary}
  Следовательно, $dm = \rho dV$. Тогда масса тела: $M = \int_V \rho \, dV$, $\rho(x^i, t), x^i \in V(t), t\in[0, t_{max})$
\end{corollary}

Если $ \dfrac{dM}{dt} = 0 $, тогда закон сохранения массы в интегральной форме:
\[
  \dfrac{d}{dt} \int\limits_{V(t)} \rho dV = 0
\]

\begin{corollary}
  Так как масса тел сохраняется для любых моментов времени, если тела состоят 
  из одних и тех же точек, то для элементарного объёма $d\mathring{V}$ и $dV$
  масса должна сохраняться: если $d\mathring{V} \to d\mathring{m} \mathring{\rho} d\mathring{V}$
  и $dV \to dm = \rho dV$, то
  \begin{equation}\label{lec_6:eq_lagrange_1form}
    dm = d\mathring{m} \Leftrightarrow
    \mathring{\rho} d\mathring{V} = \rho dV.
  \end{equation}
  -- закон сохранения массы в дифференциальной форме, или уравнение неразрывности в 
  переменных Лагранжа в форме №1.
  .
  % TODO рисунок
\end{corollary}



\subsection{Формы №2 и №3 уравнения неразрывности}

Вспомним, что такое $d\mathring{V}$: $d\mathring{V} = d\mathbf{x}_1 \cdot \left( d\mathring{\mathbf{x}}_1 \times d\mathring{\mathbf{x}}_3 \right) $ -- смешанное произведение, если $d\mathring{\bar{x}}_\alpha = \vec{MM}_\alpha$, то $d\mathbf{x}_\alpha = \vec{MM}_\alpha$.

Так как $d\mathring{\mathbf{x}}_\alpha = \mathring{\mathbf{r}}_\alpha dX^\alpha$,
$d\mathbf{x}_\alpha = \mathbf{r}_\alpha dX^\alpha$, то $d\mathring{V} = \mathring{\mathbf{r}}_1 \cdot \left( \mathring{\mathbf{r}}_2 \times \mathring{\mathbf{r}}_3 \right) dX^1 dX^2 dX^3 $,
и, аналогично: $dV = \mathbf{r}_1 \cdot \left(\mathbf{r}_2 \times \mathbf{r}_3 \right) dX^1 dX^2 dX^3$, тогда $ \dfrac{dV}{d\mathring{V}} = \sqrt{\dfrac{g}{\mathring{g}}} $.

Вспомним \eqref{lec_6:eq_lagrange_1form}: $ \dfrac{\rho}{\mathring{\rho}} = \sqrt{\dfrac{g}{\mathring{g}}} $, тогда
\[
  \rho \sqrt{g} = \mathring{\rho} \sqrt{\mathring{g}}
\]
-- вторая форма уравнения неразрывности в переменных Лагранжа.

$\dots$

\[
  \sqrt{\dfrac{g}{\mathring{g}}}
  = \det\left(\dfrac{\dfrac{\partial x^i}{\partial X^j}}{\dfrac{\partial \mathring{x}^k}{\partial X^j}}\right)
  = \det\left({\dfrac{\partial x^i}{\partial X^j} \cdot \dfrac{\partial X^j}{\partial \mathring{x}^k}}\right)
  = \det \left( \dfrac{\partial x^i}{\partial \mathring{x}^k}  \right) 
\]

Но $F = \left( \dfrac{\partial x^i}{\partial \mathring{x}^j}  \right) \bar{\mathbf{e}_i} \otimes \bar{\mathbf{e}}^j$,
и $\det F = \det \left( \dfrac{\partial x^i}{\partial \mathring{x}^j}  \right) $.

Таким образом: $\sqrt{\dfrac{g}{\mathring{g}}} = \det F$.

\[
  \dfrac{\rho}{\mathring{\rho}} = \sqrt{\dfrac{\mathring{g}}{g}} = \dfrac{1}{\det F}
  \Rightarrow
  \mathring{\rho} = \rho \cdot \det F.
\]

То есть именно геометрия определяет изменение плотности (но не начальную плотность), несмотря
на то, что масса возникает независимо от геометрии.



\subsection{Дифференцирование интегралов по подвижному объёму}

\[
  M(t) = \int_{V(t)} \rho dV
  = \int_{V(0)} \rho(X^i, 0) \, d\mathring{V}
  = \int_{\mathring{V}} \mathring{\rho} \, d\mathring{V}
  = \mathring{M}
\]
.

Таким образом, всегда можно сказать, что $\int_V \rho \, dV = \int_\mathring{V} \mathring{\rho} \, d\mathring{V}$

Рассмотрим далее следующий интеграл: $\int_{V(t)} \mathbf{a} (x^i, t) \, dV$
% TODO рисунок ежа

Вычислим $\dfrac{d}{dt} \int_{V(t)} \mathbf{a}(x^i, t) \, dV$. Согласно правилам
замены переменных от $x^i$ к $\mathring{x}^i$: $x^i = x^i(\mathring{x}^j, t)$
при фиксированном $t$. Тогда:
\begin{multline*}
  \dfrac{d}{dt} \int_{V(t)} \mathbf{a}(x^i, t) \, dV
  = \dfrac{d}{dt} \int_{\mathring{V}} \mathbf{a}(\mathring{x}^i, t) \det \left( \dfrac{\partial x^i}{\partial \mathring{x}^j}  \right) \, d\mathring{V}
  = \dfrac{d}{dt} \int_{\mathring{V}} \mathbf{a}(\mathring{x}^i, t) \sqrt{\dfrac{g}{\mathring{g}}} \, d\mathring{V} = \\
  = \int_{\mathring{V}} \dfrac{d}{dt} \left( \mathbf{a} \sqrt{\dfrac{g}{\mathring{g}}} \right) \, d\mathring{V}
  = \int_{\mathring{V}} \dfrac{1}{\sqrt{\mathring{g}}} \left( \dfrac{d\mathbf{a}}{dt}\sqrt{g} + \mathbf{a} \dfrac{d\sqrt{g}}{dt} \right) \, d\mathring{V} = 
\end{multline*}
Имеет место следующая формула (без доказательства): $\dfrac{d}{dt}\sqrt{g} = \sqrt{g} \, \nabla \cdot \mathbf{v}$.
Тогда продолжая:
\begin{multline*}
  = \int_{\mathring{V}} \dfrac{\sqrt{g}}{\sqrt{\mathring{g}}} \left( \dfrac{d\mathbf{a}}{dt} + \mathbf{a} \cdot \nabla \cdot \mathbf{\mathring{v}} \right) \, d\mathring{V}
  = \int_{V(t)} \left( \dfrac{d \mathbf{a}}{dt} + \mathbf{a} \nabla \cdot \mathbf{v} \right) \, dV = \\
  = \int_{V(t)} \left( \left. \dfrac{\partial \mathbf{a}}{\partial t} \right|_{x^i} 
    + \mathbf{v} \cdot \nabla \otimes \mathbf{a} + \mathbf{a} \nabla \cdot \mathbf{v}\right) dV
  = \int_{V(t)} \left( \dfrac{\partial \mathbf{a}}{\partial t} + \nabla \cdot \left( \mathbf{v} \otimes \mathbf{a} \right)  \right) dV
\end{multline*}

Таким образом:
\[
  \dfrac{d}{dt} \int_{V(t)} \mathbf{a}(x^i, t) \, dV
  = \int_{V(t)} \left( \dfrac{\partial \mathbf{a}}{\partial t} + \nabla \cdot \left( \mathbf{v} \otimes \mathbf{a} \right)  \right) dV
\]
-- формула дифференцирования интеграла по переменному объёму (обобщение формулы
дифференцирования интеграла с переменным верхним пределом).



\subsection{Уравнение неразрывности в переменных Эйлера}

Применим эту формулу к частному случаю $\mathbf{a} = \varphi \bar{\mathbf{e}}_1$, где
$\varphi(x^i, t)$ -- скалярное поле.
\[
  \dfrac{d}{dt} \int_V \varphi \bar{\mathbf{e}}_1 \, dV
  = \int_V \left( \dfrac{\partial \varphi}{\partial t} \bar{\mathbf{e}}_1 + \nabla \cdot \left( \mathbf{v} \otimes \varphi \bar{\mathbf{e}}_1 \right)  \right) \, dV
  = \int_V \left( \dfrac{\partial \varphi}{\partial t}  + \nabla \cdot (\varphi \mathbf{v}) \right) dV
\]
В качестве $\varphi$ выберем плотность:
\[
  \dfrac{d}{dt} \int_V \rho dV
  = \int_V \left( \dfrac{\partial \rho}{\partial t} + \nabla (\rho \mathbf{v}) \right) \, dV
\]

Из закона сохранения массы в интегральной форме следует, что 
\begin{equation}\label{lec_6:eq_mass_integ}
  \int_V \left( \dfrac{\partial \rho}{\partial t} + \nabla \cdot (\rho \mathbf{v}) \right) \, dV = 0
\end{equation}

% TODO чтото тут

\[
  \int_{V(t)} \Omega dV = 0
\]
-- если $V(t)$ -- некоторая конкретная область, тогда это уравнение -- интегральное
(Вольтерры), если $V(t)$ -- произвольная (т.е. ищем решения, верные для целого класса
областей), тогда $\Omega = 0$. Вот таким образом можно перейти от интегрального
представления закона к дифференциальному.

Применим такую процедуру к \eqref{lec_6:eq_mass_integ}:
\[
  \dfrac{\partial \rho}{\partial t}  + \nabla \cdot (\rho \mathbf{v}) = 0, 
  \rho, \mathbf{v} // (x^i, t), x^i \in V, t \in [0, t_{max}]
\]
-- уравнение неразрывности в переменных Эйлера. 

Это одно из уравнений газовой динамики. Оно имеет дивергентную форму --
это такая форма, когда перед всеми частными производными коэффициенты единица:
в декартовом базисе это уравнение принимает вид:
\[
  \dfrac{\partial \rho}{\partial t}  + \dfrac{\partial \rho \mathbf{v}^i}{\partial x^i} = 0.
\]
также оно первого порядка, гиперболического типа.

Отметим, что все законы сохранения будут иметь похожую форму.

Преобразуем:
\begin{align*}
  \dfrac{\partial \rho}{\partial t} + \nabla \rho \mathbf{v} + \rho \nabla \cdot \mathbf{v} &= 0 \\
  \left( \dfrac{\partial \rho}{\partial t} + \mathbf{v} \nabla \rho \right) &= - \rho \nabla \mathbf{v} \\
  \dfrac{d \rho}{dt} &= - \rho \nabla \cdot \mathbf{v}
\end{align*}
последнее называют уравнением неразнывности в полных дифференциалах.

\subsection{Закон изменения количества движения (закон сохранения импульса)}

Обоснование этого закона (нестрогое). Рассмотрим точки массами $m_i$ (в сплошной среде такого
быть не может), каждая со скоростью $\mathbf{v}_i$. Какие-то точки пусть будут соединены
какими-то связями. Вектор $m_i \mathbf{v}_i$ назовём вектором импульса (количества движения)
материальной точки. Рассмотрим $\dfrac{d}{dt} \sum_{i=1}^N m_i \mathbf{v}_i = \sum_{i=1}^N \mathbf{F}_i^{(e)}$ ($\mathbf{F}_i^{(e)}$ -- внешние силы, а внутренние силы компенсируются).

% TODO рисунок

Пока нету понятия силы, и оказывается, что нельзя ввести силу из того, что мы уже имеем.

\begin{definition}
  % TODO рисунок
  Векторная величина:
  \[
    \mathbf{I} = \int_V \mathbf{v} \, dm
  \]
  называется вектором \emph{количества движения} (вектор импульса) сплошной среды.
\end{definition}


\paragraph{Аксиома 5. Закон изменения количества движения}

Любой паре тел сопоставляется вектор-функция $\mathcal{F}(V_1, V_2, t)$, называемая
вектором силы взаимодействия тел $V_1$ и $V_2$, обладающая следующими свойствами:
\begin{itemize}
  \item Аддитивность: $\mathcal{\mathbf{F}}(V_1 \bigcup V_1', t)
    = \mathcal{\mathbf{F}}(V_1, V_2, t) + \mathcal{\mathbf{F}}(V_1', V_2, t)$
    (аналогично для разбиения тела $V_2$). Это условие выполняется для любого
    (в том числе континуального) разбиения.
  \item Для любого тела $V$, изменение его вектора количества движения $\mathbf{I}$
    определяется его суммарным вектором сил $\mathcal{\mathbf{F}}$ взаимодействия
    данного тела со всеми окружающими его телами:
    \[
      \dfrac{d\mathbf{I}}{dt} = \mathcal{\mathbf{F}}.
    \]
    Причем некоторые части этого вектора сил могут быть нулевым.
\end{itemize}

  \begin{remark}
  \begin{enumerate}

    \item В этой аксиоме введено понятие силы $\mathcal{\mathbf{F}}$, которое не может 
      быть определено через какие-либо введенные ранее понятия.

    \item <<О первом законе Ньютона>>. Инерциальные системы отсчета. В 
      аксиоме 2 о евклидовости пространства $\mathbb{E}_3^a$ предполагается, что
      всегда можно ввести единую (единую для всех тел и для любого момента времени)
      декартовую ортонормированную систему координат. Получается, что
      первая часть закона Ньютона содержится в аксиоме евклидовости.
      Вторая часть первого закона Ньютона содержиться в аксиоме 5.

      Фактически, второй закон Ньютона -- частный случай \eqref{axiom-5}.
      Абсолютно твёрдое тело -- такая сплошная среда, для которой рассмотренный
      закон для любой точки в любой момент времени ???
      Если рассмотреть абсолютно твёрдое тело, то 
      \[
        \mathbf{v} = \mathbf{v}(t) \forall X^j \in V_X
      \]
      -- одинаковая скорость любой точки тела (не рассматривается вращение).
      Тогда $\mathbf{I} = M \cdot \mathbf{v}$, а, следовательно, \eqref{axiom-5}
      принимает вид:
      \[
        \dfrac{d\mathbf{I}}{dt} = M \dfrac{d\mathbf{v}}{dt} = \mathbf{\mathcal{F}}.
      \]
      Тогда обозначая ускорение $\mathbf{a} = \dfrac{d\mathbf{v}}{dt}$, 
      получаем классический второй закон Ньютона
      $M\mathbf{a} = \mathbf{\mathcal{F}}$.

    \item <<О неинерциальных системах отсчёта>>. Существуют разные системы 
      отсчёта, в том числе подвижные. В таких системах отсчёта законы меняются.
      Отсюда следует, что уравнение \eqref{axiom-5} -- необъективно, то есть зависит
      от системы отсчета (закон сохранения массы, к примеру, не зависел от
      системы отсчёта). Причём указанная аксиома никак не говорит о том, какой
      именно базис является инерциальным. Оказывается, что данная проблема 
      не решаема. В каждой задаче необязательно учитывать все эти неинерциальности,
      например, в случае расчёта машины не обязательно учитывать вращение Земли.
      % TODO рисунок базисы чото туда сюда
      % TODO рисунок машинки
  \end{enumerate}
\end{remark}

\subsection{Силы в МСС}

Разделим силы на внешние и внутренние. Про внешние силы вроде понятно, но что
такое внутренние? Рассмотрим некоторое тело. Разделим его
мысленно на части, каждая из частей действует на другую с некоторой силой.
Эти силы и назовём внутренними.

Также силы деляться на массовые и поверхностные.

Перечислим основные силы:
\begin{itemize}
  \item Массовые силы:
    Внешние:
    \begin{itemize}
      \item гравитационные силы;
      \item силы 
      \item электромагнитные силы;
    \end{itemize}
    Внутренние:
    \begin{itemize}
      \item силы инерции;
    \end{itemize}

  \item Поверхностные силы:
    \begin{itemize}
      \item силы трения -- внешние;
      \item <<силы упругости>> -- внутренние.
    \end{itemize}
\end{itemize}

Рассмотрим отдельно массовые и поверхностные силы.

\paragraph{Массовые силы (внешние)}
Разобьём тело на континуальное множество областей.
Для тела в актуальной конфигурации и точки $M$ с некоторой окрестностью $dV$ массы
$dm$. Согласно аксиоме 5, существует сила взаимодействия $dV$ с \emph{внешностью}, 
по отношению ко всему телу, обозначим эту силу $\mathbf{dF}_m$ -- внешняя массовая сила.

Введём новое понятие $\mathbf{f} = \dfrac{\mathbf{dF}_m}{dm}, dm > 0$ -- 
\emph{вектор плотности массовых сил}.

Для внешних массовых сил суммарный вектор внешних массовых сил:
\[
  \mathbf{\mathcal{F}}_m = \int\limits_V d\mathbf{\mathcal{F}}_m
  = \int\limits_V \rho \mathbf{f} \, dV.
\]

\paragraph{Поверхностные силы (внешние)}
Поверхность тела в актуальной конфигурации обозначим $\Sigma$. Для точки $M \in \Sigma$
выберем некоторую площадку $d\Sigma$. На часть объёма $d\tilde{V} = dV \bigcap V$
действует некоторая сила $d\mathbf{F}_\Sigma$ -- \emph{внешняя поверхностная сила}.
Для неё введём вектор $\mathbf{s} = \dfrac{d\mathbf{F}_\Sigma}{d\Sigma}$
-- \emph{вектор плотности поверхностных сил}.

Тогда в силу аддитивности сил можно сказать, что вектор суммарных внешних
поверхностных сил, действующих на поверхность тела $V$: (так как
$\Sigma = \int_\Sigma d\Sigma$)
\[
  \mathbf{\mathcal{F}} = \int\limits_\Sigma d\mathbf{\mathcal{F}}_\Sigma 
  = \int\limits_\Sigma \mathbf{s}' d\Sigma.
\]

Суммарная внешняя сила, действующая на тело $V$ со стороны окружающих его тел:
\[
  \mathbf{\mathcal{F}} = \mathbf{\mathcal{F}}_m + \mathbf{\mathcal{F}}_\Sigma.
\]


\subsection{Интегральная формулировка закона изменения количества движения}
Подставим полученное в \eqref{axiom-5}:
\begin{equation}\label{integral-form-I}
  \dfrac{d}{dt} \int\limits_V \rho \mathbf{v} dV =
  \int\limits_V \rho \mathbf{f} \, dV + \int\limits_\Sigma \mathbf{s} \, d\Sigma.
\end{equation}

Красота заключается в том, что законы сохранения имеют одинаковый вид, для
сравнения:
\[
  \dfrac{d}{dt} \int\limits_V \rho \, dV = 0.
\]

Используем правило дифференцирования интеграла по подвижному объёму. Для этого 
приведём без доказательства следующую формулу:

Упражение: доказать:
\[
  \dfrac{d}{dt} \int\limits_{V(t)} \rho \mathbf{a} \, dV =
  \int\limits_{V(t)} \dfrac{d\mathbf{a}}{dt} \, dV
\]

Тогда:
\begin{align*}
  &\int\limits_V \rho \dfrac{d\mathbf{v}}{dt} \, dV = \int\limits_V \rho \mathbf{f} \, dV
  + \int\limits_\Sigma \mathbf{s} \, d\Sigma, \\
  &\int\limits_V \rho \left( \mathbf{f} - \dfrac{d\mathbf{v}}{dt} \right) \, dV
  + \int\limits_\Sigma \mathbf{s} \, d\Sigma = 0.
\end{align*}
-- \emph{уравнение равновесия всех сил, действующих на тело}, где 
$\mathbf{\mathcal{F}}^{(i)}_{m} = - \int\limits_V \rho \dfrac{d\mathbf{v}}{dt} \, dV$
-- внутренняя массовая сила.



\subsection{Внутренние поверхностные силы}

Рассмотрим сплошную среду $V$. Разделим её на две части $V_1, V_2$. Выберем
произвольную точку $M \in \Sigma_0 \bigcap V$. Введем области $V_1$ и $V_2$
и вектор нормали $\mathbf{n}$, внешний по отношению к $V_2$. $\mathbf{n}$
отнесём в точку $M$ с площадкой $d\Sigma$.
% TODO рисунок
Тогда область $V_1$ действует на $V_2$ с силой $d\mathbf{\mathcal{F}}_2$, и
наоборот $V_2$ на $V_1$ с силой $d\mathbf{\mathcal{F}}_1$. Поскольку появление
этих сил связано с разбиением тел некоторой поверхностью, то эти силы будем
считать поверхностными, а не массовыми. Тогда можно ввести плостности этих
поверхностных сил $\mathbf{s}$: $\mathbf{s}_1 = \dfrac{d\mathbf{\mathcal{F}}_1}{d\Sigma}, \mathbf{s}_2 = \dfrac{d\mathbf{\mathcal{F}_2}}{d\Sigma}$.

Рассмотрим $\mathbf{t}_n = \dfrac{d\mathbf{\mathcal{F}}_1}{d\Sigma}$ и
$\mathbf{t}_{-n} = \dfrac{d\mathbf{\mathcal{F}}}{d\Sigma}$.
Их можно трактовать так: через данную точку можно провести некоторую поверхность
разделения тела, и вдоль элементарной площадки вокруг точки $M$ будут появляться
силы, а значит и $\mathbf{t}_n$ и $\mathbf{t}_{-n}$ -- \emph{векторы напряжений}.


\subsection{Первая теорема Коши}

\begin{theorem}[Коши, 1]
  Если $\mathbf{t}_n$ определена для поверхности $\Sigma$, не являющейся 
  поверхностью разрыва, то:
  \[
    \mathbf{t}_n = - \mathbf{t}_{-n}.
  \]

  Даже если поверхность $\Sigma$ сама гладкая, то это не значит, что решение
  будет само непрерывным (поверхности со скачками называем поверхностями
  разрыва).
\end{theorem}
\begin{proof}
  Рассмотрим область $V$ и разобьём её на две части $V_1$ и $V_2$ с помощью 
  поверхности $\Sigma$. И применим к $V_1$ и $V_2$ закон об изменении 
  количества движения в интегральной форме \eqref{integral-form-I}:
  \[
    \begin{cases}
      \int\limits_{V_1} {\rho \left(\mathbf{f} - \dfrac{d\mathbf{v}}{dt}\right) \, dV}
      + \int\limits_{\Sigma_1} \mathbf{s} \, d\Sigma
      + \int\limits_{\Sigma_0} \mathbf{t}_n \, d\Sigma = 0, \\

      \int\limits_{V_2} \rho \left(\mathbf{f} - \dfrac{d\mathbf{v}}{dt}\right) \, dV
      + \int\limits_{\Sigma_2} \mathbf{s} \, d\Sigma
      + \int\limits_{\Sigma_0} \mathbf{t}_{-n} \, d\Sigma = 0, \\

      \int\limits_{V_1 \bigcup V_2} \rho \left(\mathbf{f} - \dfrac{d\mathbf{v}}{dt}\right) \, dV
      + \int\limits_{\Sigma_1 \bigcup \Sigma_2} \mathbf{s} \, d\Sigma = 0
    \end{cases}
    \Rightarrow
    \int\limits_{\Sigma_0} (\mathbf{t}_n + \mathbf{t}_{-n}) d\Sigma &= 0.
  \]

  Поскольку $\Sigma_0$ -- произвольная, то её можно заменить на $d\Sigma_0$, 
  или по-другому, можно сказать, что в силу произвольности, 
  равенство нулю интеграла означает равенство нулю подинтегральной функции:
  \[
    \mathbf{t}_n + \mathbf{t}_{-n} = 0.
  \]

  Неявно использовали условие о гладкости $\mathbf{t}_n$.
\end{proof}


\subsection{Вторая теорема Коши}

Рассмотрим элементарный объём $dV$ в виде тетраэдра, который построен на
элементарных векторах $d\mathbf{x}_\alpha = \mathbf{r}_\alpha \cdot dX^\alpha$
(суммирования нет). У тетраэдра 4 грани, 3 из них назовём \emph{координатными}
-- $d\Sigma_\alpha, \alpha=1, 2, 3$, а оставшуюся \emph{наклонной} $d\Sigma_0$.
Определим вектора нормалей на этих гранях
$\mathbf{n}_\alpha = - \dfrac{\mathbf{r}^\alpha}{|\mathbf{r}^\alpha|}$ 
(т.к. $\mathbf{r}_\alpha \cdot \mathbf{r}^\beta = \delta^\beta_\alpha
\Rightarrow \mathbf{r}_\alpha \times \mathbf{r}_\beta = \sqrt{g} r^\gamma$).
А к $d\Sigma_0$ -- $\mathbf{n}$.

Рассмотрим следующее соотношение для 
любой поверхности $\Sigma$, ограничивающей замкнутую область:
\[
  \int\limits_\Sigma \mathbf{n} d\Sigma
  = \int\limits_\Sigma \mathbf{n} \cdot E \, d\Sigma
  = \int\limits_V \nabla \cdot E \, dV
  = 0,
\]
(использована формула Гаусса-Остроградского, а значит предполагается гладкость 
поверхности $\Sigma$).

Применим эту формулу к тетраэдру:
\[
  \int\limits_V \mathbf{n} \, d\Sigma
  = \sum_{\alpha=1}^3 \mathbf{n}_\alpha d\Sigma_\alpha + \mathbf{n} d\Sigma_0
  = 0.
\]

Подставим выражения для нормалей:
\begin{align*}
  \mathbf{n} d\Sigma_0
  &= \sum_{\alpha=1}^3 \dfrac{\mathbf{r}^\alpha}{|\mathbf{r}^\alpha|} d\Sigma_\alpha \\
  \mathbf{n} \cdot \mathbf{r}_\beta
  &= \sum_{\alpha=1}^3 \dfrac{\mathbf{r}^\alpha\cdot\mathbf{r}_\beta}{|\mathbf{r}^\alpha|} d\Sigma_\alpha
  = \dfrac{d\Sigma_\beta}{|\mathbf{r}^\beta|} \\
  d\Sigma_\beta = |\mathbf{r}^\beta| \mathbf{n} \cdot \mathbf{r} d\Sigma_0.
\end{align*}

Применим к тетраэдру закон об изменении количества движения в форме
\eqref{integral-form-I}:
\[
  \int\limits_\Sigma \mathbf{t}_n \, d\Sigma
  = \int\limits_V \rho \left( \dfrac{d\mathbf{v}}{dt} - \mathbf{f} \right) \, dV
\]
Введём обозначение: на каждую площадку $d\Sigma_\alpha$ и $d\Sigma_0$
действуют силы $d\mathbf{\mathcal{F}}_\alpha$ и $d\mathbf{\mathcal{F}}_0$.
Тогда на наклонной площадке с нормалью $\mathbf{n}$ можно ввести 
$\mathbf{t}_n = \dfrac{d\mathbf{\mathcal{F}}_0}{d\Sigma_0}$. Для остальных
площадок $\mathbf{t}_\alpha = \dfrac{d\mathbf{\mathcal{F}}_\alpha}{d\Sigma_\alpha}$.

\begin{align*}
  \mathbf{t}_n d\Sigma_0 + \sum_{\alpha=1}^3 \mathbf{t}_\alpha d\Sigma_\alpha
  &= \rho \left( \dfrac{d\mathbf{v}}{dt} - \mathbf{f} \right) \, dV \\
  \mathbf{t}_n d\Sigma_0
  + \sum_{\alpha=1}^3 \mathbf{t}_\alpha (\mathbf{n} \cdot \mathbf{r}_\alpha) |\mathbf{r}^\alpha| d\Sigma_0 &= \rho \left( \dfrac{d\mathbf{v}}{dt} - \mathbf{f} \right) \, dV \\
  \mathbf{t}_n - \sum_{\alpha=1}^3 \mathbf{t}_\alpha (\mathbf{n} \cdot \mathbf{r}_\alpha) |\mathbf{r}^\alpha| &= \rho \left( \dfrac{d\mathbf{v}}{dt} - \mathbf{f} \right) \dfrac{dV}{d\Sigma_0}
\end{align*}
Совершим предельный переход, стягивая $dV$ в точку, тогда
$h=\max_\alpha \left\{ h_\alpha \right\} , d\Sigma_0 = O(h^2);
dV = O(h^3) \Rightarrow \dfrac{dV}{d\Sigma_0} = O(h)$. Тогда получается, что
правая часть -- бесконечно малая по сравнению с левой:
\[
  \mathbf{t}_n = \sum_{\alpha=1}^3 (\mathbf{n} \cdot \mathbf{r}_\alpha) |\mathbf{r}^\alpha| \mathbf{t}_\alpha.
\]
так как $|\mathbf{r}^\alpha| = \sqrt{g^{\alpha\alpha}}$, то введём обозначение:
$\mathbf{t}^\alpha \equiv \mathbf{t}_\alpha \sqrt{g^{\alpha\alpha}}$. Тогда
\begin{align*}
  \mathbf{t}_n
  &= \sum_{\alpha=1}^3 (\mathbf{n} \cdot \mathbf{r}_\alpha) \mathbf{t}^\alpha, \\
  \mathbf{t}_n
  &= \mathbf{n} \sum_{\alpha=1}^3 \mathbf{r}_\alpha \otimes \mathbf{t}^\alpha.
\end{align*}

Тензор $T = \sum_{\alpha=1}^3 \mathbf{r}_\alpha \otimes \mathbf{t}^\alpha
= \mathbf{r}_i \otimes \mathbf{t}^i$ называется \emph{тензором истинных
деформаций Коши}.

Тогда $\mathbf{t}_n = \mathbf{n} \cdot T$ -- \emph{формула Коши}.

\begin{remark}
  Ранее мы ввели тензор градиента деформации:
  $F = \mathbf{r}_i \otimes \mathring{\mathbf{r}}^i$: $\mathbf{r}_i = F \cdot \mathring{\mathbf{r}}_i$.
  Теперь мы получили тензор истинных деформаций Коши:
  $T = \mathbf{r}_i \otimes \mathbf{t}^i$: $\mathbf{t}_n = T^T \cdot \mathbf{n}$.
\end{remark}

\begin{remark}
  Для тела $V$ и произвольной точки $M$ и проходящей через эту точку площадки
  $d\Sigma$ (для каждой площадки будет свой вектор напряжений $\mathbf{t}_n$),
  но существует такой тензор, что можно узнать любое такое напряжение 
  с помощью действия этого тензора на $\mathbf{n}$.
\end{remark}

\begin{remark}
  Тензор напряжений $T$ введён в актуальной конфигурации. Для него имеет место
  формула:
  \[
    d\mathbf{\mathcal{F}} = \mathbf{t}_n d\Sigma = \mathbf{n} \cdot T d\Sigma.
  \]
\end{remark}

\subsection{Тензор напряжений Теолы-Кирхгофа}

Рассмотрим $\mathring{\mathbf{t}} = \dfrac{d\mathbf{\mathcal{F}}}{d\Sigma_0}$.
\[
  d\mathbf{\mathcal{F}} = \mathring{\mathbf{t}}_n d\Sigma_0
  = \mathbf{n} \cdot T d\Sigma = T^T \cdot \mathbf{n} d\Sigma,
\]
но $\mathbf{n} d\Sigma
= \sqrt{g/\mathring{g}} F^{-1T} \mathring{\mathbf{n}} d\mathring{\Sigma}$, 
тогда
\[
  d\mathbf{\mathcal{F}}
  = \sqrt{\dfrac{g}{\mathring{g}}} T^T \cdot F^{-1T} \cdot \mathbf{n} d\Sigma
\]

\[
  \mathbf{t}_n
  = \sqrt{\dfrac{g}{\mathring{g}}} \left( F^{-1} \cdot T \right)^T \mathring{\mathbf{n}}
\]

\[
  \mathring{\mathbf{t}}_n = \mathbf{\mathring{n}} \cdot \sqrt{\dfrac{g}{\mathring{g}}} F^{-1} T
\]

Введем тензор Теолы-Кирхгофа: $P=\sqrt{\dfrac{g}{\mathring{g}}} F^{-1} T$:
\[
  \begin{cases}
    \mathring{\mathbf{t}}_n = \mathring{\mathbf{n}} \cdot P, \\
    \mathbf{t}_n = \mathbf{n} \cdot T.
  \end{cases}
\]

\[
  \begin{cases}
    d\mathbf{F} = \mathbf{t}_n d\Sigma = \mathbf{n} \cdot T d\Sigma, \\
    d\mathbf{F} = \mathring{t}_n d\mathring{\Sigma} = \mathring{\mathbf{n}} \cdot P d\mathring{\Sigma}
  \end{cases}
  
\]

  \subsection{Физический смысл компонент тензора напряжений Коши}

\paragraph{Физический смысл самого тензора Коши}
Тензор напряжений Коши $T$ вводится в актуальной конфигурации $\mathcal{K}$.
Какую бы площадку $d\Sigma$ мы не взяли в теле вокруг некоторой точки
-- на неё всегда действует сила ($d\mathcal{F}$) -- внутренняя поверхностная сила. 
Построим объект $\mathbf{t}_n = \dfrac{d\mathcal{F}}{d\Sigma}$ -- вектор напряжений.
Теорема №2 Коши говорит о том, что $\mathbf{t}_n = \mathbf{n}\cdot T$,
где тензор $T$ -- тензор напряжений Коши $T = \mathbf{r}_i \otimes \mathbf{t}^i$.

\paragraph{Физический смысл компонент}
Рассмотрим базис $\mathbf{r}_i$, по нему образуем базис $\hat{\mathbf{r}}_i$ --
процессом ортогонализации и ортонормирования. Тогда $T$ можно представить в этом
базисе $T = T^{ij} \mathbf{r}_i \otimes \mathbf{r}_j
= \hat{T}^{ij} \hat{\mathbf{r}}_i \otimes \hat{\mathbf{r}}_j$.

Для раскрытия физического смысла, рассмотрим элементарный объём $dV$ в форме куба,
рёбра которого ориентированы по $\hat{\mathbf{r}}_\alpha$. Тогда на каждой
грани $d\Sigma_\alpha$ ($d\Sigma_\alpha$ -- грань, ортогональная ребру $\hat{\mathbf{r}}_\alpha$)
этого куба действует внутренняя поверхностная сила $d\mathbf{\mathcal{F}}_\alpha$.
Тогда можно ввести вектор напряжений $\mathbf{t}_\alpha$ на $d\Sigma_\alpha$:
$\mathbf{t}_\alpha = \dfrac{d\mathbf{\mathcal{F}}_\alpha}{d\Sigma_\alpha}$.
По формуле Коши, $\mathbf{t}_\alpha = \mathbf{n}_\alpha \cdot T$, тогда:
$\dfrac{d\mathbf{\mathcal{F}}_\alpha}{d\Sigma_\alpha} = \mathbf{n}_\alpha \cdot T$.

Запишем это соотношение в базисе $\hat{\mathbf{r}}_\alpha$:
$d\mathbf{\mathcal{F}}_\alpha = d\hat{\mathbf{\mathcal{F}}}^i_\alpha \cdot\hat{\mathbf{r}}_i, \mathbf{n}_\alpha = \hat{\mathbf{n}}^i_\alpha \cdot \hat{\mathbf{r}}_i$.

Тогда:
\[
  \dfrac{d\hat{\mathbf{\mathcal{F}}}^i_\alpha \hat{\mathbf{r}}_i }{d\Sigma_\alpha}
= \mathbf{n}^i_\alpha \cdot \hat{\mathbf{r}}_i \cdot \hat{T}^{kj} \hat{\mathbf{r}}_j \otimes \hat{\mathbf{r}}_j.
\]
\[
  \dfrac{d\hat{\mathbf{\mathcal{F}}}^i_\alpha \hat{\mathbf{r}}_i }{d\Sigma_\alpha}
  = \hat{\mathbf{n}}^i_\alpha ^{kj} \delta_{ik} \hat{\mathbf{r}}_j
\]
\[
  \dfrac{d\hat{\mathbf{\mathcal{F}}}^j_\alpha \hat{\mathbf{r}}_i }{d\Sigma_\alpha}
  = \hat{\mathbf{n}}_{k\alpha} \hat{T}^{kj}
\]
-- выражение компонент тензора деформаций в данном базисе.
Так как $dV$ -- куб, ориентированный вдоль базиса $\hat{\mathbf{r}}_\alpha$,
поэтому компоненты нормалей $\hat{\mathbf{n}}_{k\alpha} = \delta_{k\alpha}$.
\[
  \delta_{k\alpha} \hat{T}^{kj} = \dfrac{d\mathbf{\mathcal{F}}^j_\alpha}{d\Sigma_\alpha}
\]
\[
  \hat{T}^{\alpha j} = \dfrac{d\mathbf{\mathcal{F}}^j_\alpha}{d\Sigma_\alpha}
\]

Рассмотрим отдельно диагональные компоненты:
\[
  \hat{T}^{\alpha \alpha} = \dfrac{d\mathbf{\mathcal{F}}^\alpha_\alpha}{d\Sigma_\alpha}
\]
$d\mathbf{\mathcal{F}}^\alpha_\alpha$ -- проекции на нормаль к грани $d\Sigma_\alpha$
-- $\mathbf{n}_\alpha$, поэтому $\hat{T}^{\alpha\alpha}$ -- \emph{нормальные напряжения}
-- проекция силы на нормаль, делённая на площадь площадки, перпендикулярной 
к этой нормали.

% TODO рисунок-иллюстрация, поясняющая почему это проекции и какие именно это
% проекции

Рассмотрим теперь $\hat{T}^{\alpha\beta} = \dfrac{d\hat{\mathbf{\mathcal{F}}}^\beta_\alpha}{d\Sigma_\alpha}$.
\[
  \begin{cases}
    \hat{T}^{\alpha\beta} = \dfrac{d\hat{\mathbf{\mathcal{F}}}^{\beta_\alpha}}{d\Sigma_\alpha},\\
    \hat{T}^{\alpha\gamma} = \dfrac{dd\hat{\mathbf{\mathcal{F}}}^{\gamma}_\alpha}{d\Sigma_\alpha}.
  \end{cases}
\]
-- касательные напряжения.

Всего для такого кубика получилось
3 нормальных напряжения (на каждую площадку) и 6 касательных напряжений (по две
на каждую площадку). На данном этапе. ничего не известно про симметрию (к сожалению).
В общем случае тензор $T$ -- не симметричен.

\paragraph{Размерность}
Рекомендация: почитайте второй том МСС, если интересуетесь размерностями.
Вообще есть 4 размерности: секунды, метры, килограммы, кельвины -- почемы их 4?
Оказывается, что их 4, потому что они все следуют из математики! То есть
оказывается, что есть ровно 4 масштабных коэффициента, которые допускают 
масштабирование системы координат. Оказывается, что всех размерностей 
в электродинамике не существует, они все выводятся из механических размерностей.
(правда все они в дробных степенях, чего другие обычные механические размерности
никогда не делают)

Понятие размерности относят только к скалярам и компонентам векторов, но не к тензорам.
Те размерности величин компонент векторов и тензоров обычно рассматриваются в 
ортонормированном базисе.

У компонент тензора напряжения Коши в декартовом ортонормированном базисе:
\[
  \left[ \hat{T}^{\alpha\beta} \right] = \dfrac{\text{Н}}{\text{м}^2} = \text{Па}.
\]
Также часто используется $\text{кгс} / \text{мм}^2 = \dots$.


\subsection{Физический смысл компонент тензора напряжений Пиолы-Кирхгофа}

Напомним, что данный тензор вводиться для некоторой площадки $d\mathring{\Sigma}$
в отсчётной конфигурации, которая переходит в $d\Sigma$. Тогда:
$\mathbf{n} \cdot T d\Sigma = \mathbf{t}_n \cdot d\Sigma =
d\mathbf{\mathcal{F}}
= \mathring{\mathbf{t}}_n d\mathring{\Sigma} = \mathring{\mathbf{n}} \cdot P \cdot d\mathring{\Sigma},$
где тензор $P = \sqrt{\dfrac{g}{\mathring{g}}} F^{-1} \cdot T$ -- тензор Пиолы-Кирхгофа.

Рассмотрим в $\mathring{\mathcal{K}}$ элементарный объем в виде куба, 
ориентированного по векторам базиса $\hat{\mathring{\mathbf{r}}}_\alpha$ --
ортонормированный базис, в актуальной конфигурации этот объём перейдет в
параллелограм образованный векторами $\mathbf{r}_\alpha$.
На каждую грань $d\Sigma_\alpha$ этого параллелограма действует сила
$d\mathbf{\mathcal{F}}_\alpha$, тогда:
$\mathring{\mathbf{n}}_\alpha \cdot P = \dfrac{d\mathbf{\mathcal{F}}}{d\mathring{\Sigma}_\alpha}$.
Запишем $\mathring{\mathbf{n}}_\alpha$, $P$, $d\mathbf{\mathcal{F}}$ в базисе
$\hat{\mathring{\mathbf{r}}_\alpha}$:
\[
  \mathring{n}_\alpha = \hat{\mathring{n}}^i_\alpha \cdot \mathring{\mathbf{r}}_i
  \quad
  \dots
  \quad
  \dots
  \quad
  d\mathbf{\mathcal{F}}_\alpha = d\hat{\mathbf{\mathcal{F}}}^i_\alpha \cdot \hat{\mathbf{r}}_i
\]
% TODO дописать

\[
  \hat{\mathring{n}}_{\alpha k} \cdot \hat{\mathring{P}}^{kj} = \dfrac{d\hat{\mathring{\mathcal{F}}}^i_\alpha}{d\mathring{\Sigma}_\alpha}.
\]

Так как $d\mathring{V}$ -- куб, то $\mathring{n}_{\alpha k } = \delta_{\alpha k}$,
поэтому:
\[
  \hat{\mathring{P}}^{\alpha k} = \dfrac{d\hat{\mathring{\mathcal{F}}}^i_\alpha}{d\mathring{\Sigma}_\alpha}.
\]

Аналогично тому, как мы рассматривали тензор Коши, рассмотрим сначала диагональные
элементы:
\[
  \hat{\mathring{P}}^{\alpha\alpha} = \dfrac{d\hat{\mathring{F}}^{\alpha}_\alpha}{d\mathring{\Sigma}_\alpha}
\]
-- $d\hat{\mathring{\mathcal{F}}}^\alpha_\alpha$ -- взяли прообраз вектора силы
$d\mathbf{\mathcal{F}}_\alpha$ в отсчетной конфигурации, и его спроецировали на 
нормаль к прообразу площадки, на которую он действует, поэтому диагональные
компоненты называются нормальными напряжениями.

Аналогично получаем, что 
\[
  \begin{cases}
    \hat{\mathring{P}}^{\alpha\beta} = \dfrac{d\hat{\mathring{F}}^\beta_\alpha}{d\mathring{\Sigma}_\alpha}, \\
    \hat{\mathring{P}}^{\alpha\gamma} = \dfrac{d\hat{\mathring{F}}^\gamma_\alpha}{d\mathring{\Sigma}_\alpha}.
  \end{cases}
\]
-- касательные напряжения.


\subsection{Уравнения движения}
Рассмотрим снова уравнение изменения количества движения в интегральной форме:
\[
  \dfrac{d}{dt} \underbrace{\int\limits_V \rho \mathbf{v} dV}_{\equiv I} = 
  \int\limits_V \rho \mathbf{f} dV + \int\limits_\Sigma \underbrace{\mathbf{t}_n d\Sigma}_{= \int\limits_{\Sigma} d\mathbf{\mathcal{F}}_\Sigma}.
\]
Согласно правилу дифференцирования интеграла по подвижному объёму:
\[
  \int\limits_V \rho \left( \dfrac{d\mathbf{v}}{dt} - \mathbf{f} \right) dV
  - \int\limits_\Sigma \mathbf{n} \cdot T d\Sigma = 0,
\]
по формуле Гаусса-Остроградского:
\[
  \int\limits_V \rho \left( \dfrac{d\mathbf{v}}{dt} - \mathbf{f} \right) dV
  - \int\limits_V \nabla \cdot T \, dV = 0,
  \Rightarrow
  \int\limits_V \left(
    \rho \left( \dfrac{d\mathbf{v}}{dt} - \mathbf{f} \right)
  - \nabla \cdot T \right) \, dV = 0,
\]

Тогда, благодаря произвольности объёма интегрирования,
\begin{equation}\label{lec_8:eq_dv_euler}
  \rho \dfrac{d\mathbf{v}}{dt}
  = \nabla \cdot T + \rho \cdot \mathbf{f}
\end{equation}
-- уравнение движения в Эйлеровом описании (в полных дифференциалах).


\begin{remark}

  \begin{enumerate}
    \item Если вдруг тензор напряжений нулевой, или, например, $T \equiv a E, a = const$,
      т.е. $\nabla \cdot T \equiv 0$, тогда:
      \[
        \rho \dfrac{d\mathbf{v}}{dt} + \rho \mathbf{f}.
      \]

    \item Если вдруг $\mathbf{v} \equiv 0$ (квазистатические задачи), то
      \[
        \nabla \cdot T = \rho \mathbf{f} = 0
      \]
      --  уравнение равновесия.
  \end{enumerate}
\end{remark}

Т.к. $\dfrac{d\mathbf{v}}{dt} = \dfrac{\partial \mathbf{v}}{\partial t} (x^i, t) + \mathbf{v} \cdot \nabla \otimes \mathbf{v}$ -- полная производная по времени в эйлеровом
описании, то \eqref{lec_8:eq_dv_euler} принимает вид:
\[
  \rho \dfrac{\partial \mathbf{v}}{\partial t} + \rho \mathbf{v} \cdot \nabla \otimes \mathbf{v}
  + \mathbf{v} \cdot
  \underbrace{\left( \dfrac{\partial \rho}{\partial t} + \nabla \cdot \rho\mathbf{v} \right)}_{=0 \text{ уравнение неразрывности}}
  = \nabla \cdot T + \rho \mathbf{f}.
\]
\[
  \dfrac{\partial \rho \mathbf{v}}{\partial t} + \nabla \cdot \left( \rho \mathbf{v}\otimes \mathbf{v} \right) = \nabla \cdot T + \rho \mathbf{f}
\]
-- дивергентная форма уравнения движения в Эйлеровом описании (такая форма,
что перед всеми производными коэффициент единица). Эту форму чаще используют в 
газовой динамике.

\paragraph{Уравнение движения в Лагранжевом (материальном) описании}

Переход от Эйлерова описания к Лагранжеву описанию (и наоборот) осуществляется
через интегральную форму.
\[
  \int\limits_V \rho \mathbf{a} \, dV
  = \int\limits_V \mathring{\rho} \left( \dfrac{\rho}{\mathring{\rho}} \right) \mathbf{a} \, dV
  = \int\limits_V \mathring{\rho} \sqrt{\dfrac{g}{\mathring{g}}} \mathbf{a} \, dV
  = \int\limits_{\mathring{V}} \mathring{\rho} \mathbf{a} \, d\mathring{V}.
\]
\[
  \int\limits_{\mathring{V}} \mathring{\rho} \left( \dfrac{d\mathbf{v}}{dt} -\mathbf{f} \right) \, d\mathring{V} = \int\limits_{\mathring{\Sigma}} \mathbf{n} \cdot P \, d\mathring{\Sigma}.
\]

% TODO дописать одно уравнение тут

\[
  \int\limits_{\mathring{V}} \mathring{\rho} \left( \dfrac{d\mathbf{v}}{dt} - \mathbf{f} - \mathring{\nabla} \cdot P \right) \, d\mathring{V} = 0
\]
в силу произвольности $\mathring{V}$:
\[
  \mathring{\rho} \dfrac{d\mathbf{v}}{dt} = \mathring{\rho} \mathbf{f} + \mathring{\rho} \mathring{\nabla} \cdot P
\]
-- уравнение движения в Лагранжевом описании.


\subsection{Закон изменения момента количества движения}
Для импульса мы получили такую аналогию с системами материальных точек:
\[
  \dfrac{d}{dt} \sum_i m_i \mathbf{v}_i = \sum_i \mathbf{f}_i
  \rightarrow
  \dfrac{d}{dt} \int\limits_V \rho d\mathbf{v} = \int_V \mathbf{f}\, dV + \int_V \mathbf{t}_n \, d\Sigma.
\]

Для системы точек закон изменения момента количества движения формулируется так:
$\mathbf{x}_i \times \mathbf{f}_i$ -- момент импульса (момент количества движения).
\[
  \dfrac{d}{dt} \sum_{j=1} \mathbf{x}_i \times m_i \mathbf{v}_i = \sum_i \mathbf{x}_i \times \mathbf{f}_i
\]
-- уравнение изменения момента количества движения систем точек.

\begin{definition}
  $\mathbf{k}' \equiv \int\limits_V \mathbf{x} \times \mathbf{v} \, dm$ -- момент
  количества движения сплошной среды.
  Преобразуя интеграл, можно получить также:
  $\mathbf{k}' \equiv \int\limits_V \rho \mathbf{x} \times \mathbf{v} \, dV$.
\end{definition}

\begin{definition}
  $\mathbf{\mu}_m' \equiv \int\limits_V \mathbf{x} \times d\mathbf{\mathcal{F}}
  = \int\limits_V \rho \mathbf{x} \times \mathbf{f} \, dV$
  -- вектор момента массовых сил.

  $\mathbf{\mu}_\Sigma' \equiv \int\limits_\Sigma \mathbf{x} \times d\mathbf{\mathcal{F}}
  = \int_\Sigma \mathbf{x} \times \mathbf{t}_n \, d\Sigma$ 
  -- вектор момента поверхностных сил.

  $\mathbf{\mu}' \equiv \mathbf{\mu}'_m + \mathbf{\mu}'_\Sigma$
  -- суммарный вектор моментов.
\end{definition}

\paragraph{Аксиома 6} (Закон изменения момента количества движения).
Для всякой сплошной среды существуют две аддитивные векторные функции:
  $\mathbf{k}''$ -- вектор собственного момента количества движения,
  $\mathbf{\mu}''$ -- вектор собственных моментов,
которые удовлетворяют следующему уравнению:
\[
  \dfrac{d\mathbf{k}}{dt} = \mathbf{\mu},
\]
где $\mathbf{k} = \mathbf{k}' + \mathbf{k}''$ -- полный вектор момента количества движения;
$\mathbf{\mu} = \mathbf{\mu}' + \mathbf{\mu}''$ -- полный вектор моментов.

\[
  \dfrac{d\mathbf{k}'}{dt} = \mathbf{\mu}'
\]


\subsection{Интегральная формулировка закона изменения момента количества движения СС}
% TODO кусок лекции



\begin{definition}
  $\mathbf{k}_m \equiv \dfrac{d\mathbf{k}''}{dm}$ -- плотность собственного момента количества движения.

  $\mathbf{h}_m \equiv \dfrac{d \mathbf{\mu}''}{dm}$ -- плотность собственного массового момента.
  
  $\mathbf{h}_\Sigma \equiv \dfrac{d \mathbf{\mu}''}{d\Sigma}$ -- плотность собственного поверхностного момента.
\end{definition}

Тогда $\mathbf{k} = \int\limits_V d\mathbf{k}'' = \int\limits_V \rho \, d\mathbf{k}_m$,
и т.д., и т.п.
Используем введённые величины для преобразования закона сохранения момента количества движения:
\[
  \dfrac{d}{dt}\int\limits_V \underbrace{\left( \mathbf{x}\times\rho\mathbf{v}+\rho\mathbf{k}_m \right)}_{\mathbf{k}' + \mathbf{k}''} \, dV
  = \int\limits_V \left( \mathbf{x}\times\rho\mathbf{f}+\rho\mathbf{h}_m \right) \, dV
  + \int\limits_\Sigma \left( \mathbf{x}\times\mathbf{t}_n + \mathbf{h}_\Sigma \right) \, d\Sigma
\]
-- интегральная форма закона сохранения момента количества движения сплошных сред.


\subsection{Дифференциальная формулировка закона изменения момента количества движения СС}  

\paragraph{Обобщённая теорема Коши}
Если
\[
  \dfrac{d}{dt} \int\limits_V \rho A \, dV 
  = \int\limits_V \rho C \, dV + \int\limits_\Sigma B_n \, d\Sigma
\]
то применяя этот закон к элементарному объёму $dV$ -- тетраэдру и стягивая этот
тетраэдр $dV \to 0$, то интегралы по $V$ -- $O(h^3)$, а интеграл по $\Sigma$ -- $O(h^2)$,
следовательно, 
\[
  \sum_{\alpha=1}^4 B_{n\alpha} = 0.
\]

Применяя полученную теорему к ЗСМКД, получаем, что:
\[
  \mathbf{h}_\Sigma + \mathbf{x} \times \mathbf{t}_n = \mathbf{n} \cdot \tilde M, \quad
  \mathbf{t}_n = \mathbf{n} \cdot T.
\]

Таким образом, согласно обобщенной теореме Коши, $\exists$ тензор $\tilde M$ -- тензор момента напряжения.


  % \chapter{Модуль 2}

	% 
% TODO пропущено поллекции

\subsection{Закон сохранения энергии (первый закон термодинамики)}

\begin{definition}
  Введём следующие величины:
  \begin{enumerate}
    \item Кинетической энергией называется величина:
      \[
        K = \int\limits_V \dfrac{v^2}{2} \, dm = \int\limits_V \dfrac{\rho v^2}{2} \, dV;
      \]
    \item Мощностями массовых и поверхностных сил называются величины:
      \[
        W_m = \int\limits_V \mathbf{f} \cdot \mathbf{v} \, dm
        = \int\limits_V \rho \mathbf{f} \cdot \mathbf{v} \, dm; \quad
        W_\Sigma = \int\limits_\Sigma \mathbf{t}_n \cdot \mathbf{v} \, d\Sigma
      \]
  \end{enumerate}
\end{definition}

\paragraph{Аксиома 7. Закон сохранения энергии (первый закон термодинамики)}[формулировка К.Трусдела]
Всякая сплошная среда в актуальной конфигурации $K$ в любой момент времени $t \geqslant 0$
обладает двумя скалярными аддитивными функциями:
$U(K, t)$ -- внутренняя энергия сплошной среды, $Q(K, t)$ -- скорость нагрева сплошной среды;
такими, что
\[
  \dfrac{dE}{dt} = W + Q,
\]
где $E = U + K$ -- полная энергия;
$W = W_m + W_\Sigma$ -- мощность внешних сил.

	% \subsection{Дифференциальная форма закона СЭ}

\[
  \dfrac{\partial \rho \varepsilon}{\partial t}  + \nabla \cdot \left( \rho \mathbf{v} \varepsilon - T \mathbf{v} + \mathbf{q} \right) = \rho \mathbf{f} \cdot \mathbf{v} + \rho q,
\]
где $\varepsilon = e + \dfrac{v^2}{2}$ -- плотность полной энергии; $e = \dfrac{dU}{dm}$, 
$K = \int_V \dfrac{\rho \mathbf{v}^2}{2} \, dV = \int_V \dfrac{V^2}{2} \, dm$.

$\mathbf{a} = \rho \mathbf{v} \varepsilon - T \mathbf{v} + \mathbf{q} \equiv a^i \mathbf{r}_i$.
$a^i = \rho v^i \varepsilon - \tensor{T}{^i_j} v^j + q^i$.

\[
  \rho \dfrac{d\varepsilon}{dt} = \nabla \cdot (T \mathbf{v} - \mathbf{q}) + \rho \mathbf{f} \cdot \mathbf{v} + \rho q_n
\]
-- уравнение энергии в полных дифференциалах.

\begin{theorem}[<<живых сил>>]
  <<Живой силой>> называется кинетическая энергия.
  Уравнение движения сплошной среды в полных дифференциалах:
  \[
    \rho \dfrac{d\mathbf{v}}{dt} = \nabla \cdot T + \rho \mathbf{f},
  \]
  домножим его скалярно на $\mathbf{v}$:
  \[
    \rho \dfrac{d}{dt} \left( \dfrac{v^2}{2} \right) = \nabla \cdot T \cdot \mathbf{v} + \rho \mathbf{f} \cdot \mathbf{v}.
  \]

  % TODO много дописать
\end{theorem}

\begin{remark}
  Введём обозначения:
  \begin{enumerate}
    \item $K = \int_V \rho \dfrac{v^2}{2} \, dV$ -- кинетическая энергия сплошной среды;
    \item $W_m = \int_V \rho \mathbf{f} \cdot \mathbf{v} \, dV$ -- мощность внешних массовых сил;
    \item $W_{(i)} = - \int_V T \cddot \nabla \otimes \mathbf{v}^T \, dV$ -- мощность внутренних поверхностных сил;
    \item $\omega_{(i)} \equiv - T \cddot \nabla \otimes \mathbf{v}^T$ -- удельная мощность внутренних поверхностных сил, причём если $T$ -- симметричный, то
      \[
        \omega_{(i)} = \dfrac{1}{2} \left(  T \cddot \nabla \otimes \mathbf{v}^T + T \cddot \nabla \mathbf{v}^T \right) = T \cddot D,
      \]
      где тензор $D$ называется \emph{тензором скоростей деформации}.
  \end{enumerate}
  С учётом введённых выше обозначений, закон СЭ принимает вид:
  \[
    \dfrac{dK}{dt} = W_\Sigma + W_{(i)} + W_m
  \]
  -- теорема <<живых сил>> (теорема о кинетической энергии).
\end{remark}

В случае абсолютно твёрдого тела, мощность внутренних сил равна нулю: $W_{(i)} \equiv 0$.



\subsection{Уравнение притока тепла}

Вычтем из уравнения для полной энергии уравнение локальной теоремы о кинетической энергии:
\[
  \rho \dfrac{de}{dt} = - \nabla \mathbf{q} + \rho q_n + T \cddot \nabla \mathbf{v}^T
\]
-- \emph{локальное уравнение притока тепла или уравнение изменения плотности внутренней энергии}.
Проинтегрируем его:
\[
  \int_V \rho \dfrac{de}{dt} \, dV = - \int_V \nabla \cdot \mathbf{q} \, dV + \int_V \rho q_m \, dV
  + \int_V T \cddot \nabla \otimes \mathbf{v}^T \, dV
\]
С учётом наших обозначений, получаем интегральную формулировку уравнения притока тепла:
\[
  \dfrac{dU}{dt} = Q_\Sigma + Q_m - W_{(i)}.
\]
иногда обозначают $Q = Q_\Sigma + Q_m$.

\begin{align*}
  dU = Q dt - W_{(i)} dt; \\
  dU = \delta Q - \delta A_{(i)}; \\
  \delta Q = dU + \delta A_{(i)}.
\end{align*}


\subsection{Закон сохранения энергии в начальной конфигурации}

Переход от материального (лагранжевого) описания к пространственному (эйлерову) описанию
осуществляется с помощью интегральной формулировки.

Напомним интегральную формулировку закона сохранения энергии:
\[
  \dfrac{d}{dt} \int_V \rho \left( e + \dfrac{v^2}{2} \right) \, dV
  = - \int_\Sigma \mathbf{n} \cdot \mathbf{q} \, d\Sigma
  + \int_V \rho q_m \, dV
  + \int_V \rho \mathbf{f} \cdot \mathbf{v} \, dV
  + \int_\Sigma \mathbf{n} \cdot \left( T \cdot \mathbf{v} \right) \, d\Sigma.
\]

Так как:
\begin{align*}
  &\int_V \rho \left( e + \dfrac{v^2}{2} \right) \, dV
  = \int_{\mathring{V}} \mathring{\rho} \left( e + \dfrac{v^2}{2} \right) \, d\mathring{V}; \\
  &\int_\Sigma \mathbf{n} \cdot T \cdot \mathbf{v} \, d\Sigma
  = \int_\Sigma \mathbf{v}^T \left( \mathbf{n} \cdot T \right) \, d\Sigma
  = \int_\Sigma \mathbf{v} \cdot d\mathcal{\mathbf{F}}
  = \int_\Sigma \mathbf{v} \cdot \mathring{\mathbf{t}}_n \, d\mathring{\Sigma}
  = \int_{\mathring{\Sigma}} \mathbf{v} \cdot \mathring{\mathbf{n}} \cdot P \, d\mathring{\Sigma} \\
  &\int_\Sigma \mathbf{n} \cdot \mathbf{q} \, d\Sigma
  = \left|\, \begin{aligned} \mathring{q}_n = \dfrac{dQ_\Sigma}{d\mathring{\Sigma}} \\ \mathring{q}_n = - \mathring{\mathbf{n}} \cdot \mathring{\mathbf{q}} \end{aligned} \,\right| 
  = \int_{\mathring{\Sigma}} \mathring{\mathbf{n}} \cdot \mathring{\mathbf{q}} \, d\mathring{\Sigma}
\end{align*}

Тогда получаем представление уравнения изменения энергии в отсчётной конфигурации:
\[
  \dfrac{d}{dt} \int_{\mathring{V}} \mathring{\rho} \left( e+\dfrac{v^2}{2} \right) \, d\mathring{V}
  = \int_{\mathring{\Sigma}} \mathring{\mathbf{n}} \cdot \mathbf{q} \, d\mathring{\Sigma}
  + \int_{\mathring{V}} \rho q_m \, d\mathring{V}
  + \int_\mathring{V} \mathring{\rho} \mathbf{f} \cdot \mathbf{v} \, d\mathring{V}
  - \int_{\mathring{\Sigma}} \mathring{\mathbf{n}} \cdot P \cdot \mathbf{v} \, d\mathring{\Sigma}
\]
-- интегральный закон сохранения энергии в материальном описании.

По формуле Остроградского-Гаусса:
\[
  \int_{\mathring{V}} \left(  \rho \dfrac{d}{dt} \left( e+\dfrac{v^2}{2} \right) + \mathring{\nabla} \cdot \mathring{\mathbf{q}} - \mathring{\rho} q_m - \mathring{\rho} \mathbf{f} \cdot \mathbf{v} + \mathring{\nabla} \cdot \left( P\cdot\mathbf{v} \right)  \right) \, d\mathring{V} = 0
\]
в силу произвольности $V$:
\[
  \rho \dfrac{d}{dt} \left( e+\dfrac{v^2}{2} \right) + \mathring{\nabla} \cdot \mathring{\mathbf{q}} - \mathring{\rho} q_m - \mathring{\rho} \mathbf{f} \cdot \mathbf{v} + \mathring{\nabla} \cdot \left( P\cdot\mathbf{v} \right) = 0
\]
-- уравнение энергии (локальная формулировка) в материальном (лагранжевом) описании.

все функции здесь зависят от $(X^i, t)$ или $(\mathring{x}^i, t)$, причём
\[
  \dfrac{d}{dt} \left( e+\dfrac{v^2}{2} \right) = \dfrac{\partial }{\partial t} \left( e+\dfrac{v^2}{2} \right) |_{x^i}
\]
-- полная производная совпадает с частной.



\subsection{Нулевой закон термодинамики}

\paragraph{Аксиома 8 (Закон существования абсолютной температуры)}
Для всякой материальной точки $M$ любой сплошной среды для любого момента времени существует
скалярная положительная функция, обозначаемая $\Theta(X^i, t) > 0$, которая называется
\emph{абсолютной температурой}.

\begin{remark}
  Этот закон локальный.
\end{remark}
\begin{remark}
  Температура $\Theta$ в законе сохранения не повляется явным образом.
\end{remark}
\begin{remark}
  $\exists \Theta_{min} = 0$
\end{remark}



\subsection{Второй закон термодинамики}

\paragraph{Аксиома 9 (Второй закон термодинамики)}
Для всякой сплошной среды в любой момент времени существуют 2 скалярные аддитивные
функции:
\begin{enumerate}
  \item $H$ -- энтропия;
  \item $\bar{Q}^* \geqslant 0$ -- производство энтропии за счёт внутренних источников.
\end{enumerate}
Которые удовлетворяют следующему дифференциальному уравнению:
\[
  \dfrac{dH}{dt} = \bar{Q} + \bar{Q}^*,
\]
где $\bar{Q} = \bar{Q}_m + \bar{Q}_\Sigma$,
а $\bar{Q}^i \geqslant 0$ -- неравенство Планка.

\begin{remark}
  $\dfrac{dH}{dt} \geqslant \bar{Q}$ -- неравенство Клаузиуса.
\end{remark}


\subsection{Интегральная формулировка 2-го закона термодинамики}

В силу аддитивности $H$, можно записать $H = \int_V dH$, где $dH$ -- энтропия элементарного
объёма $dV$. Аналогично, $\bar{Q}^* = \int_V d\bar{Q}^*$, где $d\bar{Q}^*$ -- производство
энтропии за счёт внешних источников в $dV$.

\begin{definition}
  Введём величины:
  \begin{enumerate}
    \item $\eta = \dfrac{dH}{dm}$ -- плотность энтропии;
    \item $q^* = \theta \dfrac{d\bar{Q}^*}{dm} \geqslant 0$
      -- плотность внутреннего производства энтропии.
  \end{enumerate}
\end{definition}

Тогда можно получить интегральную формулировку второго закона термодинамики:
\[
  \dfrac{d}{dt} \int_V \rho \eta \, dV = \int_V \dfrac{\rho (q_m + q^*)}{\theta} \, dV
  - \int_\Sigma \dfrac{\mathbf{n} \cdot \mathbf{q}}{\theta} \, d\Sigma.
\]


\subsection{Дифференциальная формулировка 2-го закона термодинамики}

\[
  \rho \dfrac{d\eta}{dt} = - \nabla \cdot \left( \dfrac{\mathbf{q}}{\theta} \right) + \rho \dfrac{q_m + q^*}{\theta}
\]


Так как $\nabla \left( \dfrac{\mathbf{q}}{\theta} \right) = \dfrac{\nabla \mathbf{q}}{\theta} + \mathbf{q} \cdot \dfrac{1}{\theta^2} \nabla \theta$, то:
\begin{align*}
  \rho \dfrac{d\eta}{dt} =
  - \dfrac{\nabla \mathbf{q}}{\theta}
  + \dfrac{1}{\theta^2} \mathbf{q} \nabla \theta
  + \rho \dfrac{q_m + q^*}{\theta} \\
  \rho \theta \dfrac{d\eta}{dt} =
  - \nabla \mathbf{q}
  + \rho (q_m + q^*)
  + \dfrac{1}{\theta} \mathbf{q} \nabla \theta
\end{align*}

\begin{definition}
  Функция $\omega^* = \rho q^* + \dfrac{1}{\theta} \mathbf{q} \nabla \theta$ называется
  \emph{функцией диссипации (функцией рассеивания энергии)}
\end{definition}

\paragraph{Аксиома 9a} Для любой точки $M$ в любой сплошной среде в любой момент времени всегда выполнается неравенство Фурье: $\mathbf{q} \cdot \nabla \theta \leqslant 0$.

В терминах функции диссипации $\rho q^* = \omega^* - \dfrac{1}{\theta} \mathbf{q} \nabla \theta \geqslant 0$. Производство энтропии за счёт внутренних источников $q^*$ состоит из двух частей
(две причины внутреннего производства энтропии):
\begin{enumerate}
  \item За счёт функции диссипации энергии, которая определяется тензором деформации, температурой,
    но не зависит от $\nabla \alpha$: $\omega^* = \omega^* ( F, \theta )$ (будет доказано 
    позже). Эта функция определяет внутреннее производство энтропии за счёт внутреннего трения.
    \begin{itemize}
      \item Примером этого может послужить резкое сгибание твёрдого тела -- линия сгиба заметно
        нагревается;
      \item Вторым примером может послужить резкое движение бруска по поверхности -- поверхность 
        соприкосновения нагревается;
      \item Сварка трением;
      \item Консольно закреплённая балка, при отклонении от положения равновесия и отпускании,
        амплитуда колебаний будет уменьшаться как раз за счёт диссипации.
    \end{itemize}
    
  \item Если тело не движеться, то градиент деформации $F = E$, а $w^* \equiv 0$; но есть нагрев
    тела, причём предположим, что $\theta = \theta (X^i, t)$ и $\nabla \theta \neq 0$. Тогда
    появляется внутреннее производство энтропии.
\end{enumerate}

Если $\theta = \theta(t)$, т.е. $\nabla \theta \equiv 0$, тогда 
\[
  \rho q^* = \omega^* \geqslant 0
\]
-- неравенство диссипации.

\subsection{Тепловые машины и КПД}

\begin{definition}
  \emph{Тепловые машины} -- устройства (совокупность СС), которые преобразуют тепло в механическую
  работу (это различные двигатели), или наоборот (нагревательные или холодильные устройства).
  Или машины, которые включают процессы в обе стороны (это $\dots$)
\end{definition}

Тепловые машины можно разделить на три типа:
\begin{enumerate}
  \item Область $V$ (рабочее тело) состоит из одних и тех же материальных частиц.
    Примером таких машин служит поршень, газ под которым состоит из одних и тех же мат частиц.
  \item Рабочее тело $V$ не меняется во времени, но в этой области находится различные
    материальные точки. В качестве примера камера с горючим, в которое поступает горючее,
    уходят продукты горения.
  \item $V$ -- переменная, материальные точки разные. Двигатели, в которых под поршень поступает
    горючее, уходят продукты горения.
\end{enumerate}

Для простоты рассмотрим только тепловые машины первого типа.
Запишем 1-ый закон термодинамики (уравнение притока тепла):
\[
  \dfrac{dU}{dt} = Q - W_{(i)} \Leftrightarrow 
  U(t) = U(t_1) + \int\limits_{t_1}^t Q(\tau) \, d\tau - \int\limits_{t_1}^t W_{(i)}(\tau)\, d\tau
\]
Обозначим $\Delta U = U(t) - U(t_1)$ -- изменение внутренней энергии,
$C = \int\limits_{t_1}^t Q(\tau) \, d\tau$ -- суммарное тепло за время, 
$A_{(i)} = \int\limits_{t_1}^t W_{(i)}(\tau) \, d\tau$ -- работа внутренних сил.

Тогда получим: $\Delta U + A_{(i)} = C$.

Проинтегрируем аналогично второй закон термодинамики:
\[
  \Delta U = H(t) - H(t_1), \Delta \geqslant \bar{C},
\]
где $\bar{C} = \int\limits_{t_1}^t \bar{Q} \, d\tau$ -- суммарное производство энтропии внешних источников за время.

	% \paragraph{Рассмотрим следующий вопрос:}
при рассмотрении кинематики (то есть движения сплошной среды)
мы предполагали существование закона движения сплошной среды $\vec{x} = \vec{x} (X^i, t)$, 
причём $\vec{x}(X^i, t)$ -- непрерывно-дифференцируемая (иногда дважды), а при $t=0$ она ещё
и взаимно-однозначная. Возможна ли такая ситуация, когда в $\mathcal{K}$ радиус-вектор
перестанет быть взаимно-однозначной функцией.

Цель этого раздела: вывести дифференциальные уравнения, которые гарантируют, если они
выполнены, взаимную однозначность радиус-вектора материальных точек. Иначе говоря, мы хотим
получить уравнения, которые гарантируют сохранение сплошности тела в $\mathcal{K}$.

\begin{definition}
  Необходимые и достаточные условия существования однозначной функции или, что тоже самое,
  вектора перемещений $\vec{u} = \vec{x} - \mathring{\vec{x}}$ называются
  \emph{условиями соместности деформаций} сплошной среды.
\end{definition}

\begin{utv}
  Условия совместности деформации сплошной среды $\Leftrightarrow$
  сплошная среда принадлежит точечно-евклидову пространству $\mathring{\varepsilon}_3^a$.
\end{utv}
\begin{proof}
  \begin{enumerate}
    \item необходимость. Пусть точка $M \in \mathring{\varepsilon}_3^a$ в $\mathcal{K}$,
      тогда по условию евклидовости для точки М существует радиус-вектор, но это и
      означает, что выполнены условия совместности.
    \item достаточность. Аналогично.
  \end{enumerate}
\end{proof}

Из этого утверждения следует, что если условия совместности деформаций не выполнены, то такая
сплошная среда не принадлежит точечно-евклидовому пространству.



\subsection{Условия интегрируемости дифференциальной формы}

\begin{definition}
  Дифференциальной формой называется выражение вида $\sum_{\alpha=1}^3 A_\alpha dX^\alpha$,
  где $A_\alpha$ -- некоторая функция $A_\alpha = A_\alpha(X^i)$.
\end{definition}

\begin{theorem}
  Функция $A$, 
\end{theorem}


\subsection{Первая форма условий совместности деформаций}

В формулах для дифференциальных форм вместо $A_\alpha$ могут быть и вектор-функции.
Тогда:
\[
  d\vec{x} = \vec{r}_i \, dX^i
\]

\begin{utv}
  Условия совместности деформаций мыполнены тогда и только тогда, когда в $\mathcal{K}$
  существуют локальные векторы базиса $\vec{r}_i$, то есть система вектор-функций
  $\vec{r}_i = \vec{r}_i(X^j, t)$, обладающие следующим свойствам:
  \begin{enumerate}
    \item линейно независимы;
    \item однозначны и гладкие в рассматриваемой области;
    \item 
  \end{enumerate}
\end{utv}
\begin{proof}
  \begin{enumerate}
    \item необходимость. Пусть существуют локальные векторы базиса, удовлетворяющие свойствам.
      Тогда можно вычислить производную от $r_\alpha$ по $X^\beta$:
      \[
        \dfrac{\partial r_\alpha}{\partial X^\beta} = \dfrac{\partial^2 \vec{x}}{\partial X^\alpha \partial X^\beta} = \dfrac{\partial r_\beta}{\partial X^\alpha},
      \]
      иначе говоря, выполнены условия взаимности, тогда $\dots$ $\vec{x} = \vec{x}(X^i, t)$ --
      взаимно-однозначная, то есть выполнены условия совместности деформаций.

    \item Пусть существует взаимно-однозначная функция $\vec{x} = \vec{x}(X^i, t)$,
      тогда последовательно вычисляем и проверяем вектора локального базиса.
  \end{enumerate}
\end{proof}

У этом утверждении условие взаимности имеет вид: $ \dfrac{\partial \vec{r}_\alpha}{\partial X^\beta} = \dfrac{\partial \vec{r}_\beta}{\partial X^\alpha} $.

\subsection{Вторая форма условия совместности деформаций}

Выберем теперь в качестве функции $A$ локальные векторы базиса.
% TODO дописать

\begin{utv}
  УСД выполнены тогда и только тогда, когда существуют символы Кристоффеля $\Gamma_{ij}^k$,
  удовлетворяющие уравнению $\tensor{R}{_n_i_j^k} = 0$.
\end{utv}

% TODO здесь ещё чото

\begin{definition}
  Тензор $\tensor[^4]{R} = R_{nijk} r^n \otimes r^i \otimes r^j \otimes r^k$ называется
  тензором \emph{Римана-Кристоффеля}.
\end{definition}
\begin{theorem}
  Тензор Римана-Кристоффеля -- действительно тензор 4-го ранга.
\end{theorem}
\begin{proof}
  Необходимо доказать, что $R'_{i_1 i_2 i_3 i_4} = R_{j_1 j_2 j_3 j_4} \tensor{Q}{^{j_1}_{i_1}}
  \tensor{Q}{^{j_2}_{i_2}} \tensor{Q}{^{j_3}_{i_3}} \tensor{Q}{^{j_4}_{i_4}}$, но доказать
  это будет достаточно трудоёмко. Поэтому применим \emph{косвенный признак тензора}:
  если в некотором выражении учавствуют некоторые тензоры и наш неизвестный индексный объект,
  например, $a_i = b_i + T_{ik} c^k$, где $a, b, c$ -- тензоры, тогда $T_{ik}$ -- тензор по этому
  признаку.

  Расммотрим $\forall \vec{a} = a^k \vec{r}_k$ -- вектор, его вторая ковариантная производная:
  \[
    \nabla_j \left( \nabla_i a^k \right) = \dots
  \]
  Тогда:
  \[
    \nabla_j (\nabla_i a^k) - \nabla_i (\nabla_j a^k) = \left(\dfrac{\partial \Gamma^k_{is}}{\partial X^j} - \dfrac{\partial \Gamma^k_{js}}{\partial X^i} + \Gamma^{k}_{jm} \Gamma^{m}_{is} - \Gamma^k_{im} \Gamma^m_{js}\right) a^s = \tensor{R}{_{jis}^k} a^s.
  \]

  Выводы:
  \begin{enumerate}
    \item т.к. $\tensor{R}{_{jis}^k} \equiv 0$, то
      $\nabla_j (\nabla_i a^k) - \nabla_i (\nabla_j a^k) = 0$ --
      \emph{переставимость ковариантных производных при повторном дифференцировании};
    \item $\tensor{R}{_{jis}^k} \equiv 0$ -- условие евклидовости пространства $\mathcal{E}_3^a$;
    \item так как в выражении
      $\nabla_j (\nabla_i a^k) - \nabla_i (\nabla_j a^k) = \tensor{R}{_{jis}^k} a^s$
      повторные ковариантные производные являются компонентами тензоров 3-го ранга, $a$ -- вектор,
      тогда $\tensor{R}{_{jis}^k}$ -- компоненты тензора 4-го ранга.
  \end{enumerate}
\end{proof}

  % \section{Полная система законов сохранения}
Введём классификацию форм записи законов сохранения
\begin{table}[]
\centering
\label{tab:my-table}
\begin{tabular}{|ccccc|}
\hline
\multicolumn{5}{|c|}{Формы записи законов сохранения} \\ \hline
\multicolumn{3}{|c|}{\begin{tabular}[c]{@{}c@{}}В эйлеровом \\ (пространственном) описании\end{tabular}} &
  \multicolumn{2}{c|}{\begin{tabular}[c]{@{}c@{}}В лагранжевом \\ (материальном) описании\end{tabular}} \\ \hline
\multicolumn{1}{|c|}{В полных $ \mathbf{d} $} &
  \multicolumn{1}{c|}{В дивергентной форме} &
  \multicolumn{1}{c|}{$ \int $ форма} &
  \multicolumn{1}{c|}{В полных $ \mathbf{d} $} &
  $ \int $ форма \\ \hline
\end{tabular}
\caption{}
\end{table}

\begin{remark} 
  Рассматриваем только неполярные среды, то есть  
  \[
    T = T^{\mathsf T}.
  \]
  При этом исключаем у рассматриваемого закона иззменения моментов количества
  движения, то есть $ T = T^{\mathsf{T}} $ --- это оно и есть.
\end{remark}

\begin{remark}
  Все законы сохранения записываются в единой универсальной для всех всех форме.
\end{remark}

%TODO
\paragraph{1. Эйлерово описание, форма в полных дифференциалах.}
\[
    \rho \frac{d\bar A_\alpha}{dt} = \nabla \cdot B_\alpha + \rho C_\alpha, \quad
    (\alpha = 1, \ldots, 6 \text{ --- номер группы уравнений}).
\]
Здесь $ \bar A_\alpha $, $ B_\alpha $, $ C_\alpha $ --- обобщённые координатные
столбцы (векторами лучше не называть). Выпишем их: 
\begin{gather*}
  \bar{A}_\alpha = (1/\rho, \mathbf{v}, \varepsilon, \eta, \mathbf{u},
  F^{\mathsf T})^{\mathsf T}, \quad B_\alpha = (\mathbf{v}, T, T\cdot
  \mathbf{v}-\mathbf{q}, -\mathbf{q}/\theta, 0, \rhoF^{\mathsf T}\otimes
  \mathbf{v} )^{\mathsf T}, \\ C_\alpha = (0, \mathbf{f}, \mathbf{f}\cdot
  \mathbf{v} + q_m, (q_m + q^\ast)/\theta, \mathbf{v}, 0)^{\mathsf T}.
\end{gather*}

При $ \alpha = 1 $ --- уравнение неразрывности. 
\[
  \rho \frac{d(1/\rho)}{dt} = \nabla \cdot \mathbf{v}.
\]
Проверим. 
\begin{align*}
  - \frac{1}{\rho} \frac{d\rho}{dt} &= \nabla \mathbf{v},\\
  \frac{d\rho}{dt} + \rho\nabla \cdot \mathbf{v} &= 0 &\Rightarrow \frac{\partial
\rho}{\partial t} + \mathbf{v}\nabla\rho + \rho\nabla\mathbf{v} &= 0,\\
                                                                && \frac{\partial
                                                                \rho}{\partial
                                                                t}
                                                                +\nabla(\rho\mathbf{v})
                                                                &= 0.
\end{align*}



При $ \alpha = 2 $ --- уравнение движения. %TODO

При $ \alpha =3 $ --- уравнение энергии ($ \varepsilon = e + v^2/2$) 
\begin{equation}\label{eq:9}
  \rho = \frac{d \varepsilon}{dt} = \nabla \cdot (T \cdot \mathbf{v} -
  \mathbf{q}) + \rho \mathbf{f} \cdot \mathbf{v} + \rho q_m.
\end{equation}

При $ \alpha = 4 $ --- уравнение баланса энтропии. 
\begin{equation}\label{eq:10}
  \rho \frac{d\eta}{d t} = -\nabla (\mathbf{q}/\theta) + \rho \frac{q_m +
  q^\ast}{\theta}.
\end{equation}

При $ \alpha = 5 $ --- кинематическое соотношение (раньше не встречалось!) 
\[
  \rho \frac{d\mathbf{u}}{dt} = \rho \mathbf{v}.
\]
Действительно $ \frac{d\mathbf{u}}{dt} = \mathbf{v} $ --- следствие из
определения $ \mathbf{v} $ и $ \mathbf{u} $: 
\begin{align*}
  \mathbf{v} &= \frac{d \mathbf{x}}{dt} \big|_{x^i} = \frac{\partial
  \mathbf{x}}{\partial t} + \mathbf{v} \cdot \nabla \otimes \mathbf{x},\\
  %TODO %11
\end{align*}
\eqref{eq:10} $ \to $ \eqref{eq:9}:
\[
  \mathbf{v} = \frac{d}{dt}(\mathbf{u} + \mathring{\mathbf{x}}) =
  \frac{d\mathbf{u}}{dt}.
\]

При $ \alpha = 6 $ --- динамическое уравнение совместности деформации 
\[
  \rho \frac{dF^{\mathsf T}}{dt} = \nabla\cdot (\rho F^{\mathsf T}\otimes
  \mathbf{v}).
\]
Нигде не используется --- самая современная! Мало кто владеет полностью
математическим аппаратом, а мы (теперь) владеем!

\paragraph{2. Дивергентная форма законов сохранения в эйлеровом описании.}
\[\boxed{
  \frac{\partial \rho A_\alpha}{\partial t} + \nabla \cdot (\rho \mathbf{v}
\otimes A_\alpha - B_\alpha) = \rho C_\alpha, \quad (\alpha = 1, \ldots, 6).
}
\]
Заметим, что, как и выше, неизвестные константы могут быть и скалярами, и
векторами и т.\,п. 

Почему дивергентная? Потому что оператор дивергенции и потому что коэффициент
при ней не единичен (?).  
\[
  A_\alpha = (1, \mathbf{v}, \varepsilon, \eta, \mathbf{u}, F^{\mathsf T})
\]
Остальные коэффициенты те же, отличие только между $ A_1 $ и $ \bar{A}_1 $.

При $ \alpha = 1 $ имеем  
\[
  \frac{\partial \rho}{\partial t} + \nabla \cdot \rho \cdot \mathbf{v},
\]
где, например, $ A_1 \to \rho \mathbf{v} \otimes = \rho \mathbf{v} $ (такое соглашение).
Аналогично $ A_2 \to \rho \mathbf{v} \otimes \mathbf{v} $, \ldots

Остальное выписывать не будем.

Дивергентная форма хороша в численных методах, потому что можно перейти в
интегральную форму. А всё в численных методах решается интегральной формой,
никто по старинке не пользуется методом конечно-разностных сумм, не строит
прямоугольную сетку.

\paragraph{3. Интегральная форма в эйлеровом описании.}  
\[
  \frac{d}{dt} \int\limits_{V}^{} \rho A_\alpha\,d V =
  \int\limits_{\Sigma}^{}\mathbf{n} \cdot B_\alpha\,d\Sigma +
  \int\limits_{V}^{}\rho C_\alpha\,dV.
\]
(Второе слагаемое --- поток). 

Есть мнение, что интегральное исчисление более фундаментально, чем
дифференциальное. Действительно --- в том смысле, что имеет больше физических
приложений.

Например, при $ \alpha = 1 $ 
\[
    \frac{d}{dt}\int\limits_{V}^{}\rho\,dV = 0 \Rightarrow \frac{dm}{dt} = 0.
\]
В данном случае первый интеграл --- это масса.

\paragraph{4. ...}
%TODO:

При $ \alpha = 1 $  
\[
  \mathring{\rho} \frac{d}{dt} \left( \frac{\rho}{\mathring{\rho}}\det F \right)
  = 0 
\]
Но в $ \mathring{\mathcal K} $ $ \rho = \mathring{\rho} $, $ F = E $ %TODO
 
\[
    \boxed{
    \frac{\mathring{\rho}}{\rho} = \det F.}
\]


При $ \alpha = 6 $  
\[
  \mathring{\rho} \frac{dF^{\mathsf T}}{dt} = \mathring{\nabla}\cdot (\rho E
  \otimes \mathbf{v}).
\]
Так как $ \mathring{\rho} = \mathrm{const} $, то  
\begin{align*}
  \frac{dF^{\mathsf T}}{dt} &= \mathring{\nabla} \cdot (E \otimes \mathbf{v}) =
  \mathring{\nabla} \cdot \mathbf{v},\\
\end{align*}

%TODO

\paragraph{5.}
%TODO

 
\[
  \frac{d}{dt} \int\limits_{\mathring{V}}^{}\mathring{\rho}
  \mathring{A}_\alpha\,dV = \int\limits_{\mathring\Sigma}^{}\mathring n\cdot
  \mathring{B}_\alpha\,d\mathring{\Sigma} +
  \int\limits_{\mathring{V}}^{}\mathring{\rho} C\,d\mathring V.
\]
Фундаментальный закон мира: любое изменение разделяется на изменение от
внутренних источников и от внешних. Так и в квантовой физике, и в экономике, и
вообще везде.

\dimus{Современная экономика, я уж себе позволю, совсем ещё не развита. На уровне законов Ньютона
--- на 300 лет отстаёт от остальных наук. В МСС присутствуют все законы
экономики, если хотите заниматься серьёзной экономикой, занимайтесь МСС.}

При $ \alpha = 1 $  
\[
  \frac{d}{dt} \int\limits_{\mathring{V}}^{}\mathring{\rho}\,d\mathring{V},
\]


  % \input{lection13}
  % \input{lection14}
  % \input{lection15}

\end{document}
